\documentclass[cp]{copernicus}

\begin{document}

\title{How precipitation intermittency sets an optimal sampling distance for
  temperature reconstructions from Antarctic ice cores}

\Author[1]{Thomas}{M\"{u}nch}
\Author[2]{Martin}{Werner}
\Author[1,3]{Thomas}{Laepple}

\affil[1]{Alfred-Wegener-Institut Helmholtz-Zentrum f{\"u}r Polar- und
Meeresforschung, Research Unit Potsdam, Telegrafenberg A45, 14473 Potsdam,
Germany}
\affil[2]{Alfred-Wegener-Institut Helmholtz-Zentrum f{\"u}r Polar- und
Meeresforschung, Bussestra{\ss}e 24, 27570 Bremerhaven, Germany}
\affil[3]{University of Bremen, MARUM~--~Center for Marine Environmental
  Sciences and Faculty of Geosciences, 28334 Bremen, Germany}

\correspondence{Thomas M\"unch (thomas.muench@awi.de)}
\runningtitle{Optimal ice-core arrangement}
\runningauthor{T. M\"unch et al.}

\received{}
\pubdiscuss{}
\revised{}
\accepted{}
\published{}

\firstpage{1}

\maketitle

\begin{abstract} Many palaeoclimate proxies share one challenging property: they
are not only driven by the climatic variable of interest, e.g., temperature, but
they are also influenced by secondary effects which cause, among other things,
increased variability, frequently termed \emph{noise}. Noise in individual proxy
records can be reduced by averaging the records, but the effectiveness of this
approach depends on the correlation of the noise between the records and
therefore on the spatial scales of the noise-generating processes. Here, we
review and apply this concept in the context of Antarctic ice-core isotope
records to determine which core locations are best suited to reconstruct
local- to-regional-scale temperatures. Using data from a past-millennium climate
model simulation equipped with stable isotope diagnostics we intriguingly find
that even for a local temperature reconstruction the optimal sampling strategy
is to combine a local ice core with a more distant core $\sim500$--$1000$\,km
away. A similarly large distance between cores is also optimal for
reconstructions that average more than two isotope records. We show that these
findings result from the interplay of the two spatial scales of the correlation
structures associated with the temperature field and with the noise generated by
precipitation intermittency. Our study helps to maximize the usability of
existing Antarctic ice cores and to optimally plan future drilling campaigns. It
also broadens our knowledge of the processes that shape the isotopic record and
their typical correlation scales. Finally, many palaeoclimate reconstruction
efforts face the similar challenge of spatially correlated noise, and our
presented method could directly assist further studies in also determining
optimal sampling strategies for these problems.
\end{abstract}

\introduction

The oxygen and hydrogen isotopic composition of firn and ice recovered from
polar ice cores is a key proxy for past near-surface atmospheric temperature
changes \citep{Dansgaard1964,Lorius1969,Masson-Delmotte2008,Sjolte2011}.
Although the physical mechanisms that link local changes in temperature to the
isotopic composition of precipitated snow are generally well understood
\citep{Dansgaard1964,Craig1965,Jouzel1984} and can be modelled with general
circulation models
\citep{Joussaume1984,Werner2011,Werner2016,Sjolte2011,Goursaud2018}, the
quantitative interpretation of ice-core isotope variability, in terms of
temperature variability, is complicated by second-order processes that influence
the isotopic record, adding noise \citep{Munch2018a}.

Specifically, the isotopic record that is derived from an ice core is the result
of a chain of processes: (1) atmospheric temperature changes along with (2)
isotopic fractionation during the pathway from atmospheric moisture to
precipitation, (3) the effect of variable and intermittent precipitation and
finally (4) local depositional and post-depositional effects. As we outline in
the following, each element of this chain can be associated with a typical
spatial length scale over which it is correlated.

Atmospheric temperature variations drive the isotopic composition fractionation
of the atmospheric moisture along its pathway to the final stage of
precipitation \citep{Dansgaard1964,Jouzel1984}. The spatial coherence of the
temperature-related isotopic signal in precipitation is hence determined by the
spatial coherence of the variations of the atmospheric temperature field
itself. Typical spatial decorrelation scales for temperature anomalies are on
the order of $\gtrsim1000$\,km \citep{Jones1997}, which implies that ice cores
distributed on spatial scales below $\sim 1000$\,km should typically record a
similar, i.e. correlated, temperature signal. However, the temporal variability
of the isotopic composition in the local atmospheric moisture also depends on
the variability of the atmospheric circulation, since different air masses may
exhibit different source regions and distillation pathways \citep{Schlosser2004,
Sodemann2008a,Birks2009,Kuttel2012}. In addition, the isotopic composition
profile across a deposited layer of snow will not directly reflect the temporal
variability of the atmospheric isotopic signal due to the intermittent nature of
precipitation \citep{Schleiss2015}. By this, the initial isotope signal is
weighted with the amount of precipitation, which introduces bias
\citep{Steig1994,Laepple2011a} and adds additional variability to the isotopic
record \citep{Persson2011,Casado2020}. The latter two processes are linked to
atmospheric dynamics, and their typical spatial scales range from the mesoscale
(i.e. tens of kilometres), driven by topography and orographic effects, to
synoptic scales of hundreds of kilometres associated with cyclonic activity and
the movement of high- and low-pressure systems. Finally, in polar conditions,
the precipitated snow does not directly settle but is constantly eroded, blown
away, and redeposited. These depositional processes have been shown to give rise
to stratigraphic noise in the isotopic record
\citep{Fisher1985,Munch2016,Laepple2016}, which exhibits a small-scale
decorrelation scale of a few metres \citep{Munch2016}. We further note that the
final isotopic record is also influenced by potential exchange processes at the
surface and by densification and diffusion within the snow and ice, which are,
however, not within the scope of this article.

Both the effect of precipitation intermittency and stratigraphic noise
constitute a significant relative contribution to the overall isotopic
variability in the form of noise: around a deep drilling site in Dronning Maud
Land, East Antarctica, stratigraphic noise was shown to amount to approximately
50\,\% of the total variance at the seasonal timescale \citep{Munch2016}, but
quantitative estimates for other Antarctic regions are still missing. A
similarly high relative contribution is expected from precipitation
intermittency \citep{Laepple2018}, which probably has a larger impact further
inland than compared to coastal regions \citep{Casado2020,Hatvani2017}.

The hierarchy of the different spatial scales of the processes influencing an
isotope record determines the effectiveness of reducing the overall noise, since
a reduction in the noise level by averaging records will depend on the spatial
correlation scale of the different noise sources. For example, if an isotope
record were only shaped by temperature variations and stratigraphic noise, it
would be sufficient to average records spaced only tens of metres apart, as this
would ensure highly correlated temperature signals but uncorrelated
stratigraphic noise between the records. However, comparing the
correlation-based signal-to-noise ratios derived from nearby isotope records
\citep{Munch2016,Munch2017} with the signal-to-noise ratios estimated from
analysing the records' temporal variability \citep{Laepple2018} shows that the
reproducibility on a local scale does not necessarily imply a climatic, i.e.
temperature-driven, origin. Instead, the additional noise sources from
circulation variability and precipitation intermittency are likely to exhibit
larger decorrelation lengths than the stratigraphic noise
\citep{Laepple2018,Munch2018a}. Taking this into account, we expect there to be
an optimal length scale which lies between the decorrelation scales of the local
noise and of the temperature and which results in a trade-off between averaging
out atmospheric circulation and precipitation intermittency effects, while also
ensuring a sufficient coherence in the recorded temperature signal.

The aim of the present study is to use data from a climate model equipped with
stable isotope diagnostics to systematically study the different typical process
scales~--~including those from atmospheric temperature variations, circulation
variability, precipitation intermittency, and the isotope--temperature
relationship~-- to determine the optimal spatial arrangement of ice-core
locations which maximizes the correlation with temperature at a specific target
site. To address this problem we focus on target sites on the East Antarctic
Plateau. Our results show that the average of multiple ice-core isotope records
yields a higher degree of correlation with temperature when the sampled
locations are spread across distances of $1000$\,km or more from the target site
than when they are all located close ($<250$\,km) to the target site. While
these results may seem counterintuitive at first, we qualitatively explain their
general features with a simple analytical model that uses the typical spatial
correlation structures associated with the temperature and isotope fields, as
well as with the noise generated by precipitation intermittency.

\hack{\clearpage}

\section{Data and methods}\label{methods}

\subsection{Climate model data}\label{methods:data}

We use data from the past-millennium simulation (800--1999\,CE;
\citealp{Sjolte2018}) of the fully coupled ECHAM5/MPI-OM-wiso atmosphere--ocean
general circulation model equipped with stable isotope diagnostics
\citep{Werner2016}. This simulation is forced by greenhouse gases, volcanic
aerosols, total solar irradiance, land use changes, and changes in the Earth's
orbital parameters. The model's atmospheric component ECHAM5-wiso is run with a
T31 spectral resolution ($3.75\degree\times3.75\degree$) and with $19$ vertical
levels \citep{Sjolte2018}. Compared to observations, the climatological
relationship between temperature and the precipitation isotopic composition is
reproduced well by the model, but it is biased towards warm temperatures in the
T31 setup and its isotopic composition is not depleted enough over Antarctica
\citep{Werner2011}. These issues can be improved upon by using a higher spatial
resolution \citep{Werner2011}; however, such a higher-resolution model is not
needed for our study, since we are mainly interested in the relative variability
between sites and not in the absolute temperature or isotope values. The full
atmosphere--ocean model was compared to observational data and palaeoclimate
records for two equilibrium simulations under pre-industrial and Last Glacial
Maximum conditions \citep{Werner2016}, and the past-millennium simulation was
used to reconstruct North Atlantic atmospheric circulation in combination with
ice-core isotope data \citep{Sjolte2018}.

In this study, we use the 1200-year ECHAM5/MPI-OM-wiso time series of 2\,m
surface air temperature ($T_{2\mathrm{m}}$), precipitation ($p$), and oxygen
isotopic composition in precipitation (the relative abundance of oxygen-18 to
oxygen-16 isotopes, denoted as $\delta^{18}\mathrm{O}$) extracted from the total
number of $442$ model grid cells that are available for the Antarctic continent
\citep{Munch2020}.

\subsection{Data processing}\label{methods:prc}

The model simulation output has a monthly temporal resolution, while ice-core
isotope records typically exhibit an annual (or even lower) resolution. The
latter is commonly achieved by averaging the isotopic data across annual layers
of snow and ice, which are determined through a dating approach. The resulting
annual isotopic composition data therefore include a weighting effect due to the
intra-annual variability in the amount of precipitation. To account for this, we
produce two versions of annual data from the monthly model output
\citep{Munch2020}: (1) the 2\,m temperature and oxygen isotopic composition data
averaged to an annual resolution without any weighting (denoted as
$T_{2\mathrm{m}}$ and $\delta^{18}\mathrm{O}$ in the following) and (2) the
respective monthly data averaged to an annual resolution including the weighting
by the monthly precipitation amount (denoted as precipitation-weighted data
$T_{2\mathrm{m}}^{\mathrm{(pw)}}$ and $\delta^{18}\mathrm{O}^{\mathrm{(pw)}}$).

In extremely dry areas with very little precipitation or high evaporation,
numerical instabilities can occur for the modelled isotopic composition in
precipitation, resulting in anomalously strong positive or negative spikes in
the isotope time series, which is also observed for the Antarctic data in our
model simulation. We set a threshold of $4$ times the interquartile range of a
time series, above or below which data points are regarded as outliers, and
apply it to every grid cell in order to filter outliers in the
$\delta^{18}\mathrm{O}$ and $\delta^{18}\mathrm{O}^{\mathrm{(pw)}}$ time
series. This approach removes $443$ anomalous annual values ($<0.1$\,\%), out of
which $435$ anomalies occur for the model year $970$\,CE.

\subsection{Data analyses}\label{methods:main}

\subsubsection{General approach}\label{methods:general}

The overarching aim of this study is to determine a set of locations from which
the averaged model data optimally reconstruct the $T_{2\mathrm{m}}$ temperature
time series at a \emph{target site}, i.e. a specified model grid cell of
interest. The optimal reconstruction is assessed by maximizing the Pearson
correlation coefficient ($r$) with the target site temperature. To define a
spatial set, we combine a given number, $N_{\ell}$, of model grid cells and vary
$N_{\ell}$ and the distances of these locations relative to the target site. To
derive implications for actual ice-core studies, we use the
$\delta^{18}\mathrm{O}^{\mathrm{(pw)}}$ time series at the locations as a
surrogate for ice-core isotope records. We thus neglect stratigraphic noise and
any further depositional or post-depositional effects on the isotopic record,
and therefore our results represent an upper limit of the extent to which ice
cores can reconstruct the climatic temperature signal in the atmosphere. In
order to learn how the different underlying processes affect the results and to
isolate their contributions, we compare the results obtained for
$\delta^{18}\mathrm{O}^{\mathrm{(pw)}}$ with those obtained for
$T_{2\mathrm{m}}$, $T_{2\mathrm{m}}^{\mathrm{(pw)}}$, and
$\delta^{18}\mathrm{O}$. In addition to using only a single target site, we
analyse several adjacent target sites in a given region to derive results that
are relevant on local to regional spatial scales. In the next section, we
present the two main methods that we use to assess the optimal reconstructions.

\subsubsection{Assessing optimal reconstructions}\label{methods:opt.sampling}

\subsubsection*{Selecting optimal sites}

In a first approach, we select an optimal set of ice-core locations to
reconstruct a target site's $T_{2\mathrm{m}}$ time series by sampling without
replacement $N_{\ell}$ grid cells that lie within a selection circle of
$2000$\,km radius around the target site and then correlating the average
$\delta^{18}\mathrm{O}^{\mathrm{(pw)}}$ time series from these $N_{\ell}$ grid
cells with the target site temperature. We perform this for different $N_{\ell}$
and determine the optimal set of cores for each value of $N_{\ell}$ from the
maximum correlation value across all selection trials. For this, we either
sample all possible combinations of grid cell locations within the selection
circle, if the number of possibilities does not exceed $10^7$, which effectively
applies to all $N_{\ell}\le3$, or we randomly sample $10^7$ times from all the
possible combinations.\newline

%f01
\begin{figure*}[t]%
\centering
\includegraphics[width=12.5cm]{../plots/main/fig_01.pdf}
\caption[Conceptual approach]{%
  Conceptual sketch of the ring sampling approach. Around a given Antarctic
  target site (black crosses in \textbf{a} and \textbf{b}) we define
  consecutive rings of $250$\,km radial width (red lines in \textbf{a} and
  \textbf{b}). From the array of available model grid cells (grey points in
  \textbf{b}), we choose sets of grid cells which consist of $N_{\ell}$ cells
  and which are drawn from $N_{\ell}$ radial bins determined by a selected
  combination of rings. As an example for $N_{\ell}=2$, possible grid cell sets
  are shown for the cases of (i) combining the innermost ring with itself (grid
  cells marked black), (ii) combining the innermost ring with the second ring
  (grid cells marked blue), and (iii) combining the third and the fourth ring
  (grid cells marked orange). Also shown in (\textbf{a}) are our main study
  regions (black polygons) around the EDML (upward-pointing triangle) and Vostok
  (downward-pointing triangle) ice-core sites. The ring width of $250$\,km is
  chosen as a trade-off between high spatial resolution and the requirement that
  a sufficient number of grid cells lie inside each ring. Note that for
  aesthetic reasons, only four rings are displayed instead of the actually used
  nine rings and that the model grid is shown simplified as a regular grid in
  space.}
\label{fig:concept}%
\end{figure*}%

\subsubsection*{Optimal sampling structure}

In order to learn about the typical spatial scales associated with the processes
that contribute to the overall temperature--isotope relationship, we aim to
investigate how the reconstruction quality depends on the radial distances
between the target site and the locations of the ice-core network only,
neglecting their angular positions. To do so, one could use the above random
selection trials and bin them according to the distances of the selected
locations relative to the target site. However, for $N_{\ell}>1$ the number of
possible grid cell combinations quickly becomes much larger than the actual
number of grid cells. In combination with the limited computation time, such an
approach would likely result in uneven sample sizes for the available distance
combinations for larger $N_{\ell}$, especially for distances farther away from
the target site due to the radially increasing number of grid cells.

Here, we instead use a second more general approach that ensures constant
sampling of the entire available space of radial distance combinations and which
also reduces local effects in the climate model data and provides more stable
correlation results. For a given target site, we define as sampling regions nine
concentric rings around the target site with increasing radius in steps of
$250$\,km (Fig.~\ref{fig:concept}) and identify all grid cells that lie within
each of these rings. The sampling of $N_{\ell}$ grid cells is then implemented
in the following two-step process: first, we determine all possible combinations
of selecting $N_{\ell}$ rings with replacement. For every ring combination, we
then apply the following second step: we sample one individual grid cell from
each of the $N_{\ell}$ rings (see the examples in Fig.~\ref{fig:concept}b for an
illustration), extract from this grid cell set the time series for a studied
model variable, average them, and compute the degree of correlation of this
average record with the target site temperature. This second step is iterated
over the available number of grid cell sets and we report the mean correlation
across all analysed grid cell sets. For the iteration, we identify all possible
grid cell sets until $N_{\ell}=2$; for $N_{\ell}\geq3$, we resort to Monte Carlo
sampling of the grid cell sets due to computational reasons, for which we
estimated $10^5$ iterations to provide sufficient convergence of the results.

This approach provides insight into the average spatial structure of the
correlation with the target site temperature for sampling $N_{\ell}$ locations
from the model field depending on the radial distances of the locations, as
given by the respective ring midpoint radii. We denote this quantity as the
\emph{sampling correlation structure}. Note that in the one-dimensional case
($N_{\ell}=1$), the sampling correlation structure is identical to what is often
called the spatial correlation structure, i.e. the average correlation as a
function of radial distance.

\subsubsection{Study regions}\label{methods:regions}

To derive sampling correlation structures which are representative on a regional
scale, we conduct the above analysis for specific regions by successively using
each model grid cell in the region as a target site and then averaging the
resulting sampling correlation structures across these target sites.

We make use of this approach for two subregions of the East Antarctic Plateau:
the Dronning Maud Land (DML) region in the Atlantic sector of the plateau and
the Vostok region in the Indian Ocean sector, both of which include existing
deep ice-core drilling sites and large arrays of shallower ice and firn
cores. We define the DML region as the area of $\pm17.5\degree$ longitude and
$\pm5\degree$ latitude around the European Project for Ice Coring in Antarctica
(EPICA) DML site (EDML; $75\degree$\,S, $0\degree$\,E; Fig.~\ref{fig:concept}a),
consisting of $26$ model grid cells. This region encompasses the site of the
deep EDML ice core \citep{EPICAcommunitymembers2006,awi2016} and $>50$ firn and
shallow ice cores \citep{Altnau2015}. For the Vostok region, we choose an
identical latitudinal and longitudinal coverage with respect to the Vostok
station ($78.47\degree$\,S, $106.83\degree$\,E; Fig.~\ref{fig:concept}a),
covering $30$ model grid cells and encompassing the sites of the deep Vostok and
Dome~C ice cores, several shallower cores \citep{Stenni2017}, and the new deep
drilling project (``Little Dome C'') where an ice core extending back more than
$1$ million years is envisaged \citep{Passalacqua2018}.

\section{Results}\label{results}

\subsection{Spatial scale of the temperature anomalies and the local
  temperature--isotope relationship}
\label{results:t2m-iso}

First, we assess the extent to which a single ice-core record, i.e. the annual
isotope time series of an individual grid cell in the model simulation, is
representative of the local- and regional-scale variability of the near-surface
atmospheric temperature.

The temperature field over Antarctica in the climate model exhibits large-scale
coherent variations (Fig.~\ref{fig:correlation.maps}a) with a clear two-part
structure, which is roughly divided by the Transantarctic Mountains: for most
parts of the East Antarctic Plateau, the temperature field shows typical
decorrelation lengths between $\sim1500$ and $2500$\,km, while the decorrelation
lengths are notably lower, with values ranging from $\sim500$ to $1500$\,km, for
larger parts of the West Antarctic Ice Sheet and for the Antarctic Peninsula.
Still, for perfect ice cores, i.e. assuming an ideal temperature proxy record
that is only governed by local temperature variations, a single ice core would
capture the temperature variability in both East and West Antarctic regions
across hundreds of kilometres.

%f02
\begin{figure*}[t]%
\centering
\includegraphics[width=14cm]{../plots/main/fig_02.pdf}
\caption{%
  Temperature decorrelation lengths and local temperature--isotope relationship
  across Antarctica. (\textbf{a}) The temperature decorrelation lengths ($\tau$,
  in kilometres) for each Antarctic model grid cell estimated by fitting an
  exponential model to the correlation--distance relationship (see
  Eq.~\ref{eq:t2m.decorr}) obtained from correlating the local annual
  near-surface $T_{2\mathrm{m}}$ time series with the respective temperature
  time series from all other grid cells. Note that only the continental grid
  cells were used for the fit. Although the decorrelation lengths show a strong
  partition between East and West Antarctica, they are larger than $1000$\,km at
  most locations. (\textbf{b}, \textbf{c}) The local correlations at each model
  grid cell between the annual time series of precipitation-weighted oxygen
  isotope composition and of (\textbf{b}) near-surface temperature and
  (\textbf{c}) precipitation-weighted near-surface temperature. The difference
  between the maps (\textbf{d}) clearly demonstrates that precipitation
  intermittency is a major limiting factor for the temperature--isotope
  relationship.}
\label{fig:correlation.maps}%
\end{figure*}%

%f03
\begin{figure*}[t]%
\centering
\includegraphics[width=12cm]{../plots/main/fig_03.pdf}
\caption{%
  Spatial correlation with the temperature at the EDML and Vostok target
  sites. Shown are the correlations of the $T_{2\mathrm{m}}$ time series at the
  target sites EDML (\textbf{a}--\textbf{d}) and Vostok (\textbf{e}--\textbf{h})
  with the spatial fields of temperature (\textbf{a}, \textbf{d}),
  precipitation-weighted temperature (\textbf{b}, \textbf{f}), oxygen isotope
  composition (\textbf{c}, \textbf{g}), and precipitation-weighted oxygen
  isotope composition (\textbf{d}, \textbf{h}). The target sites are marked with
  a black cross; black lines indicate correlation contour lines incremented in
  steps of $0.1$.}
\label{fig:t2m.spatial.correlation}%
\end{figure*}%

However, as simulated by the isotope-enabled climate model, actual single
Antarctic ice-core isotope records only explain a low portion of the variations
in the local temperature fields: correlating the annual precipitation-weighted
field of modelled $\delta^{18}\mathrm{O}^{\mathrm{(pw)}}$ with the annual
$T_{2\mathrm{m}}$ time series at the same grid cell results in generally low
correlations (mean of $0.38$), which across all analysed grid cells range from
$\sim0.1$ up to $\sim0.57$, with $\sim60\,\%$ of the correlations $\leq0.4$
(Fig.~\ref{fig:correlation.maps}b). The correlations are overall improved when
the $T_{2\mathrm{m}}^{\mathrm{(pw)}}$ time series is used instead of the
$T_{2\mathrm{m}}$ time series (mean correlation of $0.53$, range $\sim0.1$ to
$0.77$; Fig.~\ref{fig:correlation.maps}c) but with unaffected correlation values
mostly in the coastal regions (Fig.~\ref{fig:correlation.maps}d). This shows
that precipitation intermittency is a major limiting factor for the local
temperature--isotope correlation on the continental plateau but is less
important near the coasts due to higher and more regular snowfall amounts there
\citep{Casado2020}.

\subsection{Spatial correlation with local temperature}
\label{results:t2m.spatial.correlation}

In the next step, we investigate how a local temperature record correlates in
space with the temperature itself and with the oxygen isotope composition. For
this, we choose the EDML and Vostok drilling sites as target sites and correlate
the annual $T_{2\mathrm{m}}$ time series at these target sites with the spatial
fields of annual temperature and of annual $\delta^{18}\mathrm{O}$, both
unweighted and weighted by the precipitation amount
(Fig.~\ref{fig:t2m.spatial.correlation}).

We find that the correlation patterns with the temperature field itself are
largely radially symmetric with respect to the target sites and decay uniformly
with distance within the first couple of hundred kilometres from the target
(Fig.~\ref{fig:t2m.spatial.correlation}a, e). However, for
$\delta^{18}\mathrm{O}$, and also partly through the effect of the precipitation
weighting, radial asymmetry in the correlation patterns occurs. This is
particularly striking for the EDML target site. Here, the maximum in correlation
with the $\delta^{18}\mathrm{O}$ field is not centred on the target site but
displaced by $\sim1200$\,km towards the southeast
(Fig.~\ref{fig:t2m.spatial.correlation}c, d). Some spatial displacement in
maximum correlation is also visible for the Vostok target site and the
$T_{2\mathrm{m}}^{\mathrm{(pw)}}$, $\delta^{18}\mathrm{O}$, and
$\delta^{18}\mathrm{O}^{\mathrm{(pw)}}$ fields
(Fig.~\ref{fig:t2m.spatial.correlation}f--h), but in different directions
between $T_{2\mathrm{m}}^{\mathrm{(pw)}}$ and the oxygen isotope fields and much
smaller than in the case of EDML. We also note that the correlation patterns for
the $T_{2\mathrm{m}}^{\mathrm{(pw)}}$, $\delta^{18}\mathrm{O}$, and
$\delta^{18}\mathrm{O}^{\mathrm{(pw)}}$ fields still contain radially symmetric
contributions with respect to the target sites, which are more pronounced for
the Vostok than for the EDML target site.

\subsection{Selecting optimal ice-core sites for temperature reconstructions}
\label{results:picking}

The above analyses have shown firstly that isotope records from single ice cores
likely only capture a small portion of the local interannual temperature
variability, suggesting that additional processes, such as precipitation
intermittency, influence the isotopic signal and decrease the degree of
correlation with the local temperature record. Interpreting these additional
processes as noise raises the question of whether the correlation with
temperature can be improved upon by averaging isotope records across space. In
addition, we have seen that the correlation of an oxygen isotope composition
record with a local temperature record is not necessarily maximal at the
location of the temperature recording, posing the question of how locations of
isotope records should be spatially arranged with respect to the location of the
temperature record in order to get the best correlation. To address these
questions, we assume an ideal world in which the climate model data are a
perfect surrogate for the true climate and proxy variations at each site, and we
set up the simple experiment of selecting and averaging
$\delta^{18}\mathrm{O}^{\mathrm{(pw)}}$ records from grid cells within a
$2000$\,km circle around a target site (see Sect.~\ref{methods:opt.sampling} for
details) to determine what spatial array of $N_{\ell}$ ice cores optimizes the
temperature correlation with the target site.

%f04
\begin{figure*}[t]%
\centering
\includegraphics[width=17.5cm]{../plots/main/fig_04.pdf}
\caption[Picking optimal sites]{%
  Selecting ice-core locations that optimally reconstruct interannual
  temperatures at the EDML and Vostok drilling sites. The maps show the
  correlation coefficient in the climate model data between the annual
  temperature time series at the target sites (black crosses) EDML
  (\textbf{a}--\textbf{c}) and Vostok (\textbf{b}--\textbf{f}) with the time
  series fields of precipitation-weighted oxygen isotope composition
  ($\delta^{18}\mathrm{O}^{\mathrm{(pw)}}$). Filled black points denote
  grid cells that yield the maximum correlation between the target site
  temperature and the $\delta^{18}\mathrm{O}^{\mathrm{(pw)}}$ time series from
  either selecting a single grid cell ($N_{\ell}=1$; \textbf{a}, \textbf{d}) or
  from averaging across $N_{\ell}=3$ (\textbf{b}, \textbf{e}) or $N_{\ell}=5$
  (\textbf{c}, \textbf{f}) grid cells, obtained from iteratively selecting sets
  of $N_{\ell}$ grid cells from within a selection circle of $2000$\,km radius
  around the target site indicated by the black radial lines (see
  Sect.~\ref{methods:opt.sampling} for details). Interestingly, non-local
  ice-core locations systematically show the strongest relationship with the
  target site temperature.}
\label{fig:picking}%
\end{figure*}%

For our specific model simulation and specifying the EDML drilling site as the
target site, we already know from {Fig.~\ref{fig:t2m.spatial.correlation}a} that
the optimal location for a single ice core is not the local grid cell at the
target site but should be a $\sim1200$\,km southeastward site. Choosing this
more distant site increases the correlation with the target temperature from an
$r$ value of $0.30$ for the local EDML site to a value of $0.44$
(Fig.~\ref{fig:picking}a). Even more intriguingly, when we analyse the maximum
correlations with the EDML target temperature for an average of three or five
cores chosen from the $2000$\,km selection circle (Fig.~\ref{fig:picking}b--c),
we find optimal locations that in both cases are scattered at significant
distances around the target and which yield an even further increase in
correlation ($r=0.50$ for $N_{\ell}=3$, $r=0.52$ for $N_{\ell}=5$). We obtain
comparable results when the Vostok drilling site is specified as the target
(Fig.~\ref{fig:picking}d--f). The optimal single core would be at a location
$\sim190$\,km west of Vostok ($r=0.49$ compared to the local correlation of
$r=0.46$), and the optimal locations for averaging three or five cores again all
lie scattered around the target without including it and result in a significant
further increase in correlation ($r=0.60$ for $N_{\ell}=3$, $r=0.63$ for
$N_{\ell}=5$).

We generalize these findings by considering each Antarctic model grid cell as a
target site and determining in each case the ice-core location that results in
an optimal correlation with the target site. As in the above EDML case study,
about half of the optimal locations for a single ice core are situated at
distances between $500$ and $1500$\,km from the respective target sites, while
only about $10$\,\% lie within $500$\,km from the targets. We note that this
distribution might be affected by the number of available sampling points (i.e.
model grid cells) per distance bin which increase with increasing distance from
the target site. However, after weighting the distance distribution with the
average inverse number of available grid cells per distance bin, still only
about one-fifth of the optimal distances lie within $500$\,km from the targets.

\subsection{Optimal ice-core sampling structures}
\label{results:optim-spacing}

The approach for choosing optimal ice-core locations yields straightforward and
instructive results. However, it might be doubtful as to whether these results
can be directly applied to the real world, since they might depend on the
specific simulated climate state, depend on the specific climate model and model
isotope scheme used, or result from statistical overfitting. We therefore adapt
our approach in a next step to learn more about the general spatial arrangement
of the optimal ice-core locations which yield the maximum correlation with
temperature. This is done by applying our concept of \emph{sampling correlation
structures} (see Sect.~\ref{methods:opt.sampling} and the illustration in
Fig.~\ref{fig:concept}), which studies the correlation patterns only as a
function of radial distance from the target site by averaging across $250$\,km
radial bins and across the angular positions, thereby reducing local variability
in the model data. Additionally, we apply the approach to all target sites in
our DML and Vostok study regions (Sect.~\ref{methods:regions}) and average the
results across these sites to obtain regional estimates. Finally, we analyse
each of the model variables to highlight the differences between the individual
fields.

%f05
\begin{figure*}[t]%
\centering
\includegraphics[width=13cm]{../plots/main/fig_05.pdf}
\caption{%
  Average sampling correlation structures with temperature for the DML and
  Vostok regions in the case of sampling single locations. Shown as a
  function of distance is the average correlation between the interannual
  near-surface temperature ($T_{2\mathrm{m}}$) at a target site and the spatial
  fields of $T_{2\mathrm{m}}$ (black), oxygen isotope composition
  ($\delta^{18}\mathrm{O}$, green), and precipitation-weighted oxygen isotope
  composition ($\delta^{18}\mathrm{O}^{\mathrm{(pw)}}$, blue). Averaging was
  performed in two steps: first, for a given target site, the correlations with
  the target site temperature were averaged across grid cells lying within
  $250$\,km wide consecutive rings around the given target site. Secondly, this
  analysis was conducted for all target sites in the DML (\textbf{a}) and Vostok
  (\textbf{b}) region, and the results were averaged across the respective
  region (see Sects.~\ref{methods:opt.sampling} and \ref{methods:regions} for
  details). Shading denotes $\pm1$ standard deviations of the correlation
  results across the different target sites in each region. The black dashed
  lines indicate an exponential fit to the $T_{2\mathrm{m}}$ data.}
\label{fig:avg.cor.structure}%
\end{figure*}%

When we sample only a single location ($N_{\ell}=1$), the sampling correlation
structure is conceptually equivalent to the average correlation with distance,
and it therefore simply gives the spatial decorrelation in the case of sampling
from the $T_{\mathrm{2m}}$ field itself. The average sampling correlation
structures for $T_{\mathrm{2m}}$ across the DML and Vostok regions
(Fig.~\ref{fig:avg.cor.structure}) can be described by an exponential decay with
a length scale of $\sim1900$\,km in both cases, consistent with the estimated
spatial temperature decorrelation lengths for the individual grid cells in these
regions (Fig.~\ref{fig:correlation.maps}a). In accordance with the general
expectation, the maximum average correlation with the target site temperature is
thus obtained from sampling the innermost ring only.

When we compare these results to the average sampling correlation structure for
the $\delta^{18}\mathrm{O}$ field, we find for the DML region a much lower
average correlation with the target site temperature as a function of distance
(Fig.~\ref{fig:avg.cor.structure}a). The average correlation for the innermost
ring ($<250$\,km) is $\sim0.4$, but it decreases only slightly within the first
$\sim800$\,km, followed by a slightly steeper decrease and nearly constant
levels of $r\lesssim0.2$ for distances $\gtrsim1600$\,km. For the Vostok region
(Fig.~\ref{fig:avg.cor.structure}b), the average sampling correlation structure
for $\delta^{18}\mathrm{O}$ exhibits a nearly linear decrease from an initial
value of $r\sim0.6$ to $r\sim0.1$ in the final ring ($>2000$\,km). When we
analyse the $\delta^{18}\mathrm{O}^{\mathrm{(pw)}}$ fields we find that
precipitation weighting reduces the correlation values in both regions but that
it does not have a large effect on the shape of the sampling correlation
structures itself.

Extending this analysis to the two-dimensional case of sampling and averaging
$N_{\ell}=2$ locations offers the possibility of investigating the average
correlation not only as a function of distance from the target site but also
implicitly as a function of distance between the two sampled locations
(Fig.~\ref{fig:two-core-correlation}). The difference in the average sampling
correlation structure between the fields of $T_{\mathrm{2m}}$ and
$\delta^{18}\mathrm{O}^{\mathrm{(pw)}}$ is even more pronounced for $N_{\ell}=2$
than for $N_{\ell}=1$. The maximum average correlation for $T_{\mathrm{2m}}$ is
still found when both sampling locations lie inside the innermost ring, as shown
for the DML region (Fig.~\ref{fig:two-core-correlation}a). However, for
$\delta^{18}\mathrm{O}^{\mathrm{(pw)}}$ the optimal arrangement of two locations
is to sample one location from within the innermost ring but the second location
from within the fifth ring, i.e. between $\sim1000$ and $1250$\,km from the
target site (Fig.~\ref{fig:two-core-correlation}c). Part of this structure is
related to the effect of precipitation intermittency, which can be seen from the
average sampling correlation structure of the $T_{\mathrm{2m}}^{\mathrm{(pw)}}$
field (Fig.~\ref{fig:two-core-correlation}b). Here, in contrast to
$T_{\mathrm{2m}}$, the correlation is about as high when we combine the
innermost ring and one ring further away as when we sample both locations from
within the innermost ring.

%f06
\begin{figure*}[t]%
\centering
\includegraphics[width=17.5cm]{../plots/main/fig_06.png}
\caption{%
  Average sampling correlation structure with temperature for the DML and Vostok
  regions in the two-dimensional case of sampling two locations. Shown is the
  mean correlation of all possible single correlations between the target site
  temperature and the average of two grid cells of (\textbf{a}, \textbf{d})
  $T_{\mathrm{2m}}$, (\textbf{b}, \textbf{e}) $T_{\mathrm{2m}}^{\mathrm{(pw)}}$,
  and (\textbf{c, \textbf{f}}) $\delta^{18}\mathrm{O}^{\mathrm{(pw)}}$ time
  series sampled from the same ring or from two different rings. This analysis
  was conducted for every target site in the DML region (panels
  \textbf{a}--\textbf{c}) and in the Vostok region (panels
  \textbf{d}--\textbf{f}), and the results were then averaged across the
  respective region. For each plot, the axes display the distance from the
  target site; the $x$ ($y$) axis represents the first (second) sampled
  ring, with the results being mirrored along the diagonal for aesthetic
  reasons. The tick marks indicate the border distances of the rings. Note the
  marked difference in the locations of the correlation maxima between
  $T_{\mathrm{2m}}$ and $\delta^{18}\mathrm{O}^{\mathrm{(pw)}}$ for the DML
  region, and also for the Vostok region the -- albeit marginal -- correlation
  maximum for $\delta^{18}\mathrm{O}^{\mathrm{(pw)}}$ is achieved by combining
  the innermost ring with the ring between $500$ and $750$\,km.}
\label{fig:two-core-correlation}%
\end{figure*}%

Analysing the Vostok study region leads to comparable results
(Fig.~\ref{fig:two-core-correlation}d--f), with a similar difference in average
sampling correlation structure between $T_{\mathrm{2m}}$ and
$T_{\mathrm{2m}}^{\mathrm{(pw)}}$ as for the DML region and a similar structure
of $T_{\mathrm{2m}}^{\mathrm{(pw)}}$ and $\delta^{18}\mathrm{O}^{\mathrm{(pw)}}$
for distances $\lesssim1000$\,km.  However, the results for the
$\delta^{18}\mathrm{O}^{\mathrm{(pw)}}$ field
(Fig.~\ref{fig:two-core-correlation}f) do not display such a pronounced maximum
correlation when one location is sampled from within the innermost ring and the
second one from inside a ring further away as is observed for the DML
region. This suggests that the regional differences in the spatial correlation
structure of the $\delta^{18}\mathrm{O}$ field
(Fig.~\ref{fig:avg.cor.structure}) have an influence here.

The general feature of the optimal $\delta^{18}\mathrm{O}^{\mathrm{(pw)}}$
sampling arrangement is robust throughout Antarctica, despite the above regional
differences. When we analyse all available Antarctic target sites and fix the
first core location to lie inside the innermost ring, in $\sim82\,\%$ of all
cases the optimal second core location is at least the second ring ($>250$\,km),
and in $\sim63\,\%$ of the cases it is the second to fourth ring
($250$--$1000$\,km).

%f07
\begin{figure*}[t]%
\centering
\includegraphics[width=15cm]{../plots/main/fig_07.pdf}
\caption{%
  The optimal arrangement for averaging three or five
  $\delta^{18}\mathrm{O}^{\mathrm{(pw)}}$ ice cores to reconstruct the target
  site temperature at the EDML (\textbf{a}, \textbf{c}) and Vostok (\textbf{b},
  \textbf{d}) drilling sites. Displayed are subsets of the sampling correlation
  structures for $N_{\ell}=3$ and $5$, showing along the vertical axis the
  optimal five of all possible combinations of rings (best denoted as rank 1,
  fifth best as rank 5), i.e. those which exhibit the five highest mean
  correlation values across $10^5$ random trials of averaging $N_{\ell}=3$
  (\textbf{a}, \textbf{b}) or $N_{\ell}=5$ (\textbf{c}, \textbf{d}) grid cells
  from these rings. The ring bin borders are marked by thin vertical lines with
  their distances from the target site given on the horizontal axes; the
  selected optimal ring combinations are marked as black dots. Systematically,
  arrangements which combine ice cores from the innermost ring with ice cores
  further away are found to be optimal, with larger distances for the EDML
  target site.}
\label{fig:binning}%
\end{figure*}%

%f08
\begin{figure*}[t]%
\centering
\includegraphics[width=14cm]{../plots/main/fig_08.pdf}
\caption{%
  Gain in correlation and risk of adverse sampling. (\textbf{a}) The average
  correlation with the target temperature at the EDML (red) and Vostok (blue)
  sites depending on the number of locations, $N_{\ell}$, used for averaging the
  $\delta^{18}\mathrm{O}^{\mathrm{(pw)}}$ time series. Sampling is performed
  either locally from the innermost ring only (dashed lines) or from all
  possible individual combinations of locations for the respective optimal ring
  combination determined for each $N_{\ell}$ (solid lines). Compared to the local
  samples which show virtually no or only a small increase with the number of
  sampled locations, the correlation increases markedly with $N_{\ell}$ when
  sampling from the optimal rings, as highlighted by the shaded
  area. (\textbf{b}) Histogram of individual correlations for sampling from the
  optimal ring combination when averaging $N_{\ell}=3$ locations compared to the
  correlation (vertical lines) for sampling from the innermost ring only,
  displayed for the EDML (red) and Vostok (blue) target sites. In both cases,
  the correlation is higher than the local value for more than $93\,\%$ of the
  optimal ring combination samples.}
\label{fig:cor.increase.risk}%
\end{figure*}%

We also obtain similar results when averaging $N_{\ell}=3$ or $5$ locations of
the $\delta^{18}\mathrm{O}^{\mathrm{(pw)}}$ field to reconstruct the target site
temperature (Fig.~\ref{fig:binning}). For computational reasons, we only analyse
single target sites here. When EDML is set as the target site, the optimal
sampling configuration is such that one to two core locations lie in the
innermost ring, while the others are distributed at distances mostly between
$\sim750$ and $1500$\,km from the target. For reconstructing the Vostok target
site temperature, the optimal core locations combine the innermost ring with
locations distributed mostly across the second to third ($250$--$750$\,km) ring.

In summary, averaging the $\delta^{18}\mathrm{O}^{\mathrm{(pw)}}$ time series
across the optimal locations clearly increases the average correlation with the
target site temperature more strongly with the number of locations compared to
sampling all core locations only locally close to the target site, i.e. from the
grid cells that lie within the innermost ring
(Fig.~\ref{fig:cor.increase.risk}a). While the local correlation for the EDML
target site stays constant around $0.31$, the optimal correlation rises to
$0.35$ for $N_{\ell}=2$ and to $0.43$ for $N_{\ell}=10$, which is equivalent to
nearly a doubling in the explained variance. For the Vostok target site, we
observe a nearly concurrent increase in correlation between the local and
optimal sampling up until $N_{\ell}=2$ from $0.45$ to $\sim0.50$, but for larger
$N_{\ell}$ the optimal correlation also increases more strongly and reaches
$0.58$ for $N_{\ell}=10$, which is a $\sim1.7$-fold higher explained variance
compared to $N_{\ell}=1$.

These results are the mean value from averaging across many possible
combinations of individual locations. In reality, any new drilling campaign or
reanalysis of existing ice cores only represents one single combination of
locations. Therefore, we further assess the risk of an ``adverse optimal
sampling'', i.e. the probability of choosing by chance a specific sampling
realization from the optimal ring combination which yields a lower correlation
than the correlation for sampling locally. For this purpose, we compare the
distribution of individual correlations from sampling the optimal ring
combination with the value obtained from sampling only the local sites which lie
in the innermost ring. Overall we find the risk of adverse optimal sampling to
be low, since more than $93\,\%$ of all individual correlation values in the
example of $N_{\ell}=3$ are actually larger than the respective local
correlation (Fig.~\ref{fig:cor.increase.risk}b).

\section{Discussion}\label{discussion}

\subsection{Dependence on radial distance}
\label{discussion:radial.dependence}

Oxygen isotope records derived from ice cores are commonly interpreted to
reflect local temperature changes at the ice-core drilling site. Here we have
shown that while there is local isotope--temperature correlation
(Fig.~\ref{fig:correlation.maps}b), this correlation can be increased
considerably by averaging isotope records across space
(Fig.~\ref{fig:cor.increase.risk}a) following a distinct radial pattern which
combines the local target site with locations between a few hundred kilometres
to up to $\sim1000$\,km from the target site
(Figs.~\ref{fig:two-core-correlation}c, \ref{fig:two-core-correlation}f, and
\ref{fig:binning}). These results are based on a method which investigates the
spatial correlation structure only as a function of radial distance by averaging
across the azimuthal component. The motivation for this approach is that from
physical arguments we expect the first-order spatial correlation patterns to be
invariant against rotation. Such radial symmetry is indeed observed as the
leading component of the spatial correlation structure of the temperature field
and as a second-order component of the oxygen isotope field
(Fig.~\ref{fig:t2m.spatial.correlation}). We interpret these symmetric
contributions as a general feature of the underlying atmospheric processes
compared to individual, local correlation maxima which are more due to the
actual dynamics. Therefore, we expect that our results obtained from the radial
sampling correlation structures should be largely independent of the climate
state, or the climate model used, and thus serve as valid recommendations for
real-world applications. In the next section, we substantiate this
interpretation by showing that a simple conceptual model can predict the
sampling correlation structure from the basic processes which shape the isotopic
composition time series, modelled only as a function of radial distance.
Finally, we will discuss the relevance of our results to actual ice-core
studies.

\subsection{Conceptual model of the optimal sampling structure}
\label{discussion:concept.model}

For a conceptual model of the sampling correlation structure, we focus on the
three main atmospheric processes that influence the oxygen isotope records in
ice cores: (i) temperature variations, (ii) precipitation intermittency, and
(iii) the temperature--isotope relationship. We statistically model the
associated fields of $T_{\mathrm{2m}}$, $T_{2\mathrm{m}}^{\mathrm{(pw)}}$, and
$\delta^{18}\mathrm{O}^{\mathrm{(pw)}}$ separately in order to understand the
influence of each process (see Appendix~\ref{app:concept.model} for details),
and we assess, for comparable results, the predicted average sampling
correlation structure with the target site temperature in the two-dimensional
case of averaging two locations in the same manner that we analysed the climate
model data.

To model the atmospheric temperature field, we assume an isotropic exponential
decay of the spatial correlation with a constant decorrelation length
(Appendix~\ref{app:concept.model.t2m}). Such an exponential temperature
decorrelation is a commonly observed feature \citep{Jones1997} and also
confirmed by our climate model data (Figs.~\ref{fig:correlation.maps}a,
\ref{fig:t2m.spatial.correlation}a, e, and \ref{fig:avg.cor.structure}). Given
this relationship, we find good agreement for the two-dimensional sampling
correlation structure between the conceptual model and the climate model data
regarding both absolute correlation values and the spatial pattern
(Fig.~\ref{fig:conceptual.model}a). We emphasize that the maximum correlation
with the target site temperature naturally occurs in the case of an isotropic
correlation decay when the averaged two (or $N_{\ell}$) locations are close to
the target site, as any location which is further away will contribute a
temperature signal that is less similar to the other locations.

To elucidate the role of precipitation intermittency, we follow the simplest
assumption, which is that this process can be described by partly aliasing the
original temperature signal into temporal white noise
\citep{Laepple2018,Casado2020}. We further assume that this noise is not
independent between sites but that it follows the spatial scale of precipitation
events, which we describe as an exponential decorrelation in space with a second
length scale (Appendix~\ref{app:concept.model.t2m.pw}). This intermittency
length scale is related to the atmospheric processes that deliver precipitation,
e.g. synoptic systems, and is hence assumed to be smaller than the length scale
of the temperature anomalies. The introduction of this second length scale into
our conceptual model generally explains the optimal sampling structure we
obtained from the climate model data. Qualitatively, close locations exhibit a
strong correlation in temperature but also in the noise from precipitation
intermittency; therefore, this noise cannot be reduced by averaging the
locations, yielding an overall low signal-to-noise ratio. However, with
increasing distance between the locations, the intermittency noise decorrelates
faster than the temperature field due to the different decorrelation scales,
resulting in an optimal distance of maximum signal-to-noise ratio. This is also
reflected in our conceptual model (Figs.~\ref{fig:conceptual.model.illustration}
and ~\ref{fig:conceptual.model}b, e): when fixing the position of one core to
the innermost location and varying only the distance from the target site of the
second core location, the correlation with the target site temperature first
increases with increasing distance of the second location and then maximizes at
an optimal distance before it decays with a further increase in distance. In the
climate model data, we observed a similar feature for the precipitation-weighted
temperature (Fig.~\ref{fig:two-core-correlation}), though it was not as clear as
in the conceptual model. This mismatch could be related to the assumed isotropy
in the conceptual model and the according azimuthal averaging done in the
climate model data analysis, which potentially smears the intermittency effect
in the climate model data due to slight differences in the decorrelation lengths
between the different horizontal directions.

In order to incorporate the $\delta^{18}\mathrm{O}^{\mathrm{(pw)}}$ field into
the conceptual model, we need to account for the spatial temperature--isotope
relationship. To accomplish this, we parameterize the spatial dependence of the
correlation between temperature and the oxygen isotope composition with a simple
isotropic linear model based on the climate model data results
(Fig.~\ref{fig:avg.cor.structure} and
Appendix~\ref{app:concept.model.oxy.pw}). In addition, we assume that the same
effect of precipitation intermittency that we adopted for the temperature field
is also applicable to the oxygen isotope field. With these simple assumptions,
we obtain good qualitative agreement for the DML region between the conceptual
model and the climate model data results (see Figs.~\ref{fig:conceptual.model}c
and \ref{fig:two-core-correlation}c). In addition, when we change the
parameterized isotope--temperature relationship such that it more closely
resembles the Vostok region data (Fig.~\ref{fig:avg.cor.structure}b), the
sampling correlation structure in the conceptual model
(Fig.~\ref{fig:conceptual.model}f) is more similar to the observed correlation
structure (Fig.~\ref{fig:two-core-correlation}f). However, in general the
conceptual model fails for $\delta^{18}\mathrm{O}^{\mathrm{(pw)}}$ to reproduce
the actual range in correlations as it produces much lower values than expected.

In summary, our conceptual model provides a quantitative understanding of the
spatial correlation of the temperature in the climate model data and at least a
qualitative understanding of the processes that affect the correlation between
temperature and the $\delta^{18}\mathrm{O}^{\mathrm{(pw)}}$ field, i.e.
precipitation intermittency and the spatial temperature--isotope
relationship. The deficiencies in the conceptual model may be attributed to its
simplicity. For the governing processes, we assumed spatially constant and
isotropic length scales, neglecting local and direction-related differences in
e.g. temperature decorrelation lengths (see Fig.~\ref{fig:correlation.maps}a) or
the spatial extent of the coherence of precipitation intermittency. Instead of
being constant, the latter may differ depending on the type of precipitation,
e.g. synoptic versus stratiform precipitation, and may exhibit directional
dependencies related to topography. Furthermore, we assumed constant variance of
all time series, thereby ignoring potential weighting effects on the
correlations for the spatial average of several locations due to different
variabilities between them.

\subsection{Relevance for ice-core studies}
\label{discussion:relevance}

Our results from analysing the climate model data provide guidance on where to
drill or from where to analyse $N_{\ell}=1, 2, 3$, or more ice cores in order to
optimally reconstruct the atmospheric temperature signal for a certain target
site. For this, our analysis highlights two distinct approaches.

The first possibility is to follow the recommendations obtained from directly
choosing the specific locations which maximize the correlation with the target
site temperature (Fig.~\ref{fig:picking}). This is straightforward; however, for
applications such locations would need to be derived for every target site in
question. In addition, as outlined above, it is unclear whether the results can
be one-to-one transferred to the real world, since they might be due to
unaccounted model deficiencies or depend on dynamical processes in the
atmosphere which could differ between climate states or depend on initial
conditions. One indication for this is that we obtain different optimal single
core locations for more than half of all investigated Antarctic target sites
when we analyse only the first or only the second half of the respective climate
model time series compared to the full 1200 years.

We have argued above that the sampling correlation structures, obtained from
averaging the individual correlations across space for combinations of
concentric rings around the target site, are a more general quantity, and we
have shown with our conceptual model that they are on average governed by the
interplay of the different underlying correlation length scales. We expect the
latter to vary less between different climate periods or states or between
regions. This is substantiated by the fact that the sampling correlation
structures for two cores (Fig.~\ref{fig:two-core-correlation}) are much more
robust against analysing only the first or the second half of the model time
series, which is different to the results from directly choosing optimal
locations. Thus, the sampling correlation structure offers a general approach
for finding an optimal ice-core network, but with the downside that it informs
us only about the relative radial distances of the optimal network around the
target site.

Using the sampling correlation structures we arrive at the following
recommendations for optimal ice-core sampling configurations. If it is only
possible to drill or analyse a single ice core, our results show that it is
always best to sample locally, i.e. to place this core near the target site of
interest. This is also common practice, given the usual interpretation of
ice-core isotope records as a proxy for local temperatures. However, due to the
effect of precipitation intermittency modulated by the spatial coherence of the
temperature--isotope relationship, it is no longer optimal in the case of
drilling two ice cores to collect both cores near the target site, but instead
to drill one core at the target site and one at least $500$\,km away. Where
three or more ice cores will be drilled or analysed, we expect the optimal
spatial configuration to be more dependent on the study region, but our results
indicate that it is still likely best to place one core near the target site and
distribute the others across several hundred kilometres.

These inferences are based on data from a single climate model simulation
together with a simple statistical conceptual model, which should be tested
against observations. As a proof of concept, we thus need to create an isotope
record that is in first order only governed by temperature variations and
precipitation intermittency and remove the impact of small-scale stratigraphic
noise from the actual measured records (assuming that any further processes in
the pre-depositional to depositional phase contribute negligibly to the local
isotopic variations). To accomplish this, one possible strategy would be to use
trench sampling campaigns (see \citealp{Munch2016,Munch2017}, for the EDML
site). Then, one test of our optimal sampling configurations could be to combine
one trench record, e.g. one from EDML, with another trench sampled at the
optimal distance based on our results for $N_{\ell}=2$ and correlate the average
of these two trench records with the instrumental temperature data set available
for EDML. Based on the results in this study we would expect a higher degree of
correlation in this case compared to using only one local trench record from
EDML. We acknowledge that such an approach would be challenging due to the small
amount of available instrumental data ($\sim20$\,years for EDML) and the
inevitable dating uncertainties between the two trench records.

Finally, we note that our implications concerning optimal ice-core sampling
configurations might in reality be affected by two further processes we have
neglected here. Firstly, clear-sky precipitation (``diamond dust'') is a common
phenomenon in Antarctica, especially in the drier regions of the Antarctic
Plateau, which potentially occurs more regularly than convective-type or
stratiform precipitation. Diamond dust formation is not explicitly simulated by
the ECHAM5 model, so it is possible that the precipitation intermittency
modelled in our simulation is partly offset in reality by a stronger relative
contribution of diamond dust to the total precipitation amount. Secondly,
surface--atmosphere vapour exchange between precipitation events might
constitute a second process which imprints an atmospheric temperature signal
into the surface snow, next to precipitation
\citep[e.g.][]{Steen-Larsen2014,Madsen2019}. This process could hence also
partly counteract the impact of precipitation intermittency, depending on its
relative importance for the isotopic composition of the surface snow. However,
there is no clear consensus in the recent literature on this question, and
ultimately we need quantitative estimates of the importance of vapour exchange
processes across temporal scales. In any case, these considerations do not
affect our general notion that the optimal ice-core sampling configuration
depends on the differences in spatial decorrelation scales of the processes
which shape the isotopic records.

\conclusions

In this study we assessed the spatial sampling configuration of ice cores to
optimally reconstruct the annual near-surface temperature at a specific target
site. This problem was motivated by the expectation that the major processes
influencing the isotopic records of ice cores operate on different spatial
scales.

Indeed, by analysing the temperature and isotope data of an isotope-enabled
atmosphere--ocean climate model simulating the climatic history over the last
millennium in Antarctica, we showed that while in the optimal setup a single ice
core should be placed close to the target site of interest, a second core should
be located far ($>500$\,km) from the first core. While this may seem surprising
at first glance, it can be straightforwardly explained by the interplay of two
different correlation lengths in space: one for the temperature anomalies and
one parameterizing the spatial coherence of the effect of precipitation
intermittency, as demonstrated by a simple conceptual model. Despite the fact
that these results were specifically obtained for two regions of the East
Antarctic Plateau, we expect similar results to hold for other parts of
Antarctica and potentially also for other large-scale ice-coring regions such as
Greenland, as long as our simplified assumptions of nearly isotropic exponential
decorrelation length scales are also valid there. However, we also suggest
verifying our results with a different isotope-enabled climate model in order to
rule out any dependence on the specific atmospheric model and isotope scheme
applied in the simulation used here.

Overall, our study explicitly improves the planning of drilling or analysis
campaigns for spatial networks of ice-core isotope records. In addition, it
provides a strategy to analyse an optimal configuration of sampling locations
for any proxy which is influenced by two or more processes that exhibit
different spatial correlation scales. This likely applies to various marine and
terrestrial proxy types, and our strategy might thus offer a step forward in the
best use of sampling and measurement capacity for quantitative climate
reconstructions, which needs to be investigated in further studies.

\hack{\clearpage}

\appendix

\section{Conceptual model of sampling correlation structures}
\label{app:concept.model}

\subsection{General model}
\label{app:concept.model.general}

We set up a conceptual model for the correlation between a target temperature
time series and a spatial average based on a set of locations sampled from a
climatic field (sampling correlation structure). Our model assumes simple
isotropic and exponential decorrelation structures for the involved climatic
fields and is based on previous work which suggests that precipitation
intermittency can be described by partly aliasing the original temperature
signal into white noise \citep{Laepple2018}.

In the model, we consider a temperature time series $T_0$ at some target site
$\vec{r}_0$ and a scalar field $x$ of a given climate variable. From this field,
we select $N_{\ell}$ time series $x_i$ at the locations $\vec{r}_i$,
$i=1,\dotsc,N_{\ell}$ and denote the spatial average of these time series by
$\overline{x}=\frac{1}{N_{\ell}}\sum_{i=1}^{N_{\ell}}{x_i}$. The distances of
the $N_{\ell}$ locations from the target site and the distances between the
locations are given by $r_i=|\vec{r}_i-\vec{r}_0|$ and by
$d_{ij}=|{\vec{r}_i-\vec{r}_j}|$, respectively. The correlation between $T_0$
and $\overline{x}$ follows from
%
\begin{equation}
\label{eq:corr.general}
\mathrm{cor}\left(T_0,\overline{x}\right)=\frac
{\mathrm{cov}\left(T_0,\overline{x}\right)}
{\sqrt{\mathrm{var}\left(T_0\right)\mathrm{var}\left(\overline{x}\right)}},
\end{equation}
%
and it is governed by the covariance between the temperature at the target site
and the climate field at the sampling locations $\vec{r}_i$,
%
\begin{equation}
\label{eq:cov.general}
\mathrm{cov}\left(T_0,\overline{x}\right)=
\frac{1}{N_{\ell}}\sum_{i=1}^{N_{\ell}}{\mathrm{cov}\left(T_0,x_i\right)},
\end{equation}
%
as well as by the covariance between the sampling locations through the variance
of their spatial average,
\begin{equation}
\label{eq:var.general}
\mathrm{var}\left(\overline{x}\right)=
\frac{1}{N_{\ell}^2}\left(
\sum_{i=1}^{N_{\ell}}{\mathrm{var}(x_i)} +
2\sum_{i=1}^{N_{\ell}-1}\sum_{j=i+1}^{N_{\ell}}{\mathrm{cov}\left(x_i,x_j\right)}
\right).
\end{equation}
%
In our model, these quantities depend on the distance between sites and on the
correlation structure of the respective field $x$, as we show in the following
and as illustrated in Fig.~\ref{fig:conceptual.model.illustration}.

%fA01
\begin{figure}[t]%
\centering
\includegraphics[width=6cm]{../plots/main/fig_A01.pdf}
\caption{%
  Illustration of the decorrelation lengths in the conceptual model. Shown as a
  function of distance are the correlation between two temperature time series
  (black), between the intermittency noise (purple), between a temperature and a
  precipitation-weighted temperature time series (dashed black), between two
  precipitation-weighted temperature time series (green), and between a target
  temperature time series and the average of two precipitation-weighted
  temperature time series (orange) when one is located at the target site and
  the other one is located away from the target site at a distance as indicated
  on the x axis. Model parameters are taken from the DML region. The
  decorrelation curve of the precipitation-weighted temperature time series is
  simply the superposition of the temperature decorrelation and the
  decorrelation of the intermittency noise, depending on the intermittency
  factor $\xi$.}
\label{fig:conceptual.model.illustration}%
\end{figure}%

\subsection{Temperature}
\label{app:concept.model.t2m}

For the near-surface temperature field, $x \equiv T$, we assume a spatially
constant variance, $\mathrm{var}(T_0)=\mathrm{var}(T_i)\equiv\sigma_T^2$, and an
isotropic decorrelation following an exponential decay with a decorrelation
length $\tau$; i.e. the covariance between sites is (see black line in
Fig.~\ref{fig:conceptual.model.illustration})
%
\begin{align}
\label{eq:t2m.decorr}
\mathrm{cov}\left(T_0,T_i\right)&=\sigma_T^2\exp{\left(-\frac{r_i}{\tau}\right)},\\
\mathrm{cov}\left(T_i,T_j\right)&=\sigma_T^2\exp{\left(-\frac{d_{ij}}{\tau}\right)}.
\end{align}
%
The correlation between the target site temperature and the spatial average of
$N_{\ell}$ temperature time series is then obtained from
%
\begin{equation}
\label{eq:t2m.corr}
\mathrm{cor}\left(T_0,\overline{T}\right)=
\frac{\sum_{i=1}^{N_{\ell}}\exp{\left(-\frac{r_i}{\tau}\right)}}
{\sqrt{N_{\ell}+2\sum_{i=1}^{N_{\ell}-1}
\sum_{j=i+1}^{N_{\ell}}{\exp{\left(-\frac{d_{ij}}{\tau}\right)}}}}.
\end{equation}

\subsection{Precipitation-weighted temperature}
\label{app:concept.model.t2m.pw}

To model the effect of precipitation intermittency, we follow
\citet{Laepple2018} and assume that precipitation intermittency redistributes
the energy of the temperature time series constantly across frequencies,
i.e. creating temporal white noise without changing the total variance. Then,
the precipitation-weighted temperature time series at location $\vec{r}_i$
arises from $T_i$ as
%
\begin{equation}
\label{eq:precip.weighting}
T_i^{\mathrm{(pw)}}=
\left(1-\xi\right)^{1/2}T_i + \xi^{1/2} \sigma_T \varepsilon_i(0,1),
\end{equation}
%
where $\varepsilon_i(0,1)$ represents independent and normally distributed
random variables with a mean of zero and a standard deviation of $1$. The
parameter $0\leq\xi\leq1$ determines the fraction of the input temperature time
series which is aliased into white noise.

The covariance between the target site temperature and a precipitation-weighted
temperature time series is then
%
\begin{equation}
\label{eq:t2m.pw.decorr}
\mathrm{cov}\left(T_0,T_i^{\mathrm{(pw)}}\right)=
\left(1-\xi\right)^{1/2}\sigma_T^2\exp{\left(-\frac{r_i}{\tau}\right)},
\end{equation}
%
which implies that the spatial correlation structure between $T_0$ and the
precipitation-weighted temperature follows the same exponential decay as in
Eq.~\eqref{eq:t2m.decorr}, only scaled by the factor $(1-\xi)^{1/2}$ (see dashed
black line in Fig.~\ref{fig:conceptual.model.illustration}). The factor $\xi$
can be estimated from the climate model data by analysing the local correlation,
i.e. at the same grid cell, between the temperature and the
precipitation-weighted temperature.

We further assume that the effect of precipitation intermittency is not
independent between sites but is related to the spatial coherence of the
precipitation fields, for which we assume an exponential decorrelation structure
with a decay length $\tau_{\mathrm{pw}}$. Based on these assumptions, the
spatial covariance between sites of the white noise terms induced by the effect
of precipitation intermittency has the form (see purple line in
Fig.~\ref{fig:conceptual.model.illustration})
%
\begin{equation}
\label{eq:noise.cov}
\mathrm{cov}\left(\varepsilon_i,\varepsilon_j\right)=
\exp{\left(-\frac{d_{ij}}{\tau_{\mathrm{pw}}}\right)}.
\end{equation}
%
Then, the correlation between the target site temperature and the spatial
average of $N_{\ell}$ precipitation-weighted temperature time series is governed
by the intermittency factor $\xi$ and by the two spatial length scales $\tau$
and $\tau_{\mathrm{pw}}$,
%
\begin{equation}
\label{eq:t2m.pw.corr}
\begin{split}
&\mathrm{cor}\left(T_0,\overline{T}^{\mathrm{(pw)}}\right)=\\
&\quad\frac
{\sqrt{1-\xi}\sum_{i=1}^{N_{\ell}}\exp{\left(-\frac{r_i}{\tau}\right)}}
{\sqrt{N_{\ell} + 2\sum_{i=1}^{N_{\ell}-1}\sum_{j=i+1}^{N_{\ell}}
  g\left(d_{ij}; \tau, \tau_{\mathrm{pw}}, \xi\right)}},
\end{split}
\end{equation}
%
with
\begin{equation}
\label{eq:exp.fun}
\begin{split}
g\left(d_{ij}; \tau, \tau_{\mathrm{pw}}, \xi\right)&:=
\left(1-\xi\right)\exp{\left(-\frac{d_{ij}}{\tau}\right)}\\ & \phantom{{}={}}\hspace{0.2em} +
\xi\exp{\left(-\frac{d_{ij}}{\tau_{\mathrm{pw}}}\right)}.
\end{split}
\end{equation}
%
An example of the correlation according to Eq.~(\ref{eq:t2m.pw.corr}) for
$N_{\ell}=2$ and $r_1=0$ is shown as a function of $r_2$ in
Fig.~\ref{fig:conceptual.model.illustration}.

\subsection{Precipitation-weighted oxygen isotope composition}
\label{app:concept.model.oxy.pw}

For the precipitation-weighted oxygen isotope composition field, $x \equiv
\delta^{\mathrm{(pw)}}$, we assume the same effect of precipitation
intermittency as for the temperature field. Furthermore, an analysis of the
climate model data suggests that the oxygen isotope field largely exhibits an
exponential decorrelation structure in space (not shown). Hence, the correlation
between the target site temperature and the spatial average of $N_{\ell}$
$\delta^{\mathrm{(pw)}}$ time series is obtained in a similar manner as for
$T^{\mathrm{(pw)}}$, i.e.
%
\begin{equation}
\label{eq:oxy.pw.corr}
\begin{split}
&\mathrm{cor}\left(T_0,
  \overline{\delta}^{\mathrm{(pw)}}\right)=\\
&\quad\frac
{\sqrt{1-\xi}\sum_{i=1}^{N_{\ell}}\mathrm{cor}\left(T_0,\delta_i\right)}
{\sqrt{N_{\ell} + 2\sum_{i=1}^{N_{\ell}-1}\sum_{j=i+1}^{N_{\ell}}
  g\left(d_{ij}; \tau_{\delta}, \tau_{\mathrm{pw}}, \xi\right)}},
\end{split}
\end{equation}
%
where $\tau_{\delta}$ is the decorrelation length of the $\delta$ field and the
only difference to Eq.~\eqref{eq:t2m.pw.corr} is the unknown spatial correlation
structure between the temperature at the target site and the oxygen isotope
field, $\mathrm{cor}\left(T_0,\delta_i\right)$.  Based on our climate model
results (Fig.~\ref{fig:avg.cor.structure}), we parameterize this function with a
simple linear decay of the form
%
\begin{equation}
\label{eq:t2m.oxy.corr}
\mathrm{cor}\left(T_0,\delta_i\right)=
\begin{cases}
  c_0 - \gamma d, & d \le d_0,\\
  0, & d > d_0,
\end{cases}
\end{equation}
%
where $\gamma=c_0/d_0$, and $d_0$ is some threshold distance above which
the correlation is zero.

\subsection{Model parameter estimation and model results}
\label{app:concept.model.estimation}

Overall, our model is governed by three decorrelation lengths ($\tau$,
$\tau_{\delta}$, $\tau_{\mathrm{pw}}$), the intermittency factor $\xi$, and two
parameters describing the temperature--isotope correlation ($c_0$, $d_0$).

We estimate $\tau$ from the climate model data for the DML and Vostok regions
(Fig.~\ref{fig:avg.cor.structure}) and find for both regions values of
$\tau=1900$\,km. In the same way we estimate a value of $\tau_{\delta}=1100$\,km
for both regions. The intermittency factor $\xi$ is derived from the local
correlation between temperature and precipitation-weighted temperature
(Eq.~\ref{eq:t2m.pw.decorr}). We find an average value for the DML region of
$\xi_{\mathrm{DML}}=0.73$, which is close to the average value across all of
Antarctica ($\xi_{\mathrm{Ant.}}=0.71$), while the intermittency is stronger for
the Vostok region ($\xi_{\mathrm{Vostok}}=0.82$). We parameterize the
temperature--isotope correlation in the DML region with $c_0=0.4$ and
$d_0=6000$\,km and in the Vostok region with $c_0=0.6$ and $d_0=2500$\,km
(Fig.~\ref{fig:avg.cor.structure}). The only unconstrained parameter is the
decorrelation length of the effect of precipitation intermittency,
$\tau_{\mathrm{pw}}$, since it is unclear by which precipitation variable it is
mainly governed (total annual amount, seasonal amount, or its distribution). An
investigation with reanalysis data yielded scales between $\sim300$ and
$500$\,km for different precipitation variables \citep{Munch2018a}, while our
model data exhibit an average decorrelation length of $\sim600$\,km for the
annual precipitation amount. Here, for the conceptual model we choose a value of
$500$\,km.

%fA02
\begin{figure*}[t]%
\centering
\includegraphics[width=17.5cm]{../plots/main/fig_A02.png}
\caption{%
  Two-dimensional sampling correlation structures with temperature as predicted
  from our conceptual model using the model parameters from the DML
  (\textbf{a}--\textbf{c}) and Vostok (\textbf{d}--\textbf{f}) regions. Shown is
  the mean correlation of all possible single correlations for the average of
  two time series sampled from a pair of concentric rings around the target site
  for the fields of (\textbf{a}, \textbf{d}) $T_{\mathrm{2m}}$, (\textbf{b},
  \textbf{e}) $T_{\mathrm{2m}}^{\mathrm{(pw)}}$, and (\textbf{c}, \textbf{f})
  $\delta^{18}\mathrm{O}^{\mathrm{(pw)}}$. Note that the plots (\textbf{a}) and
  (\textbf{d}) are based on the same parameters and therefore identical.}
\label{fig:conceptual.model}%
\end{figure*}%

We can test our assumption for the effect of intermittency based on using the
estimated values of $\tau$ and $\xi$ to predict the spatial decorrelation
between temperature and precipitation-weighted temperature
(Eq.~\ref{eq:t2m.pw.decorr}). Indeed, this yields a comparably good fit to the
data as an independent fit (root mean square deviation of $\sim0.03$ between
data and fit in both cases), supporting our assumption that intermittency can be
parameterized by a partial conversion of the time series into white noise.

Similarly to analysing the climate model data, we now use our conceptual model
to predict the two-dimensional ($N_{\ell}=2$) sampling correlation structures
for the different model fields of $T_{\mathrm{2m}}$,
$T_{\mathrm{2m}}^{\mathrm{(pw)}}$, and $\delta^{18}\mathrm{O}^{\mathrm{(pw)}}$
(Eqs.~\ref{eq:t2m.corr}, \ref{eq:t2m.pw.corr}, and \ref{eq:oxy.pw.corr}). Since
our model space is continuous, we sample from locations placed \emph{on}
concentric rings around the target site. We either sample the two locations from
the same ring or from two different rings using ring radii from $0$ to
$2000$\,km in increments of $10$\,km and calculate the average correlation for a
specific ring combination. To obtain meaningful expectation values, we choose
$36$ locations distributed uniformly across each ring in steps of $10\degree$,
combine these locations one by one for each ring combination, and average across
the correlations for each location pair. With the model parameters from the DML
and Vostok regions we obtain the results displayed in
Fig.~(\ref{fig:conceptual.model}), which are discussed and compared to the
estimated results from the climate model data in the main text.

\noappendix

\hack{\clearpage}

\codedataavailability{
  The climate model data used in this study are freely available from the Zenodo
  database under \url{https://doi.org/10.5281/zenodo.4001565}
  \citep{Munch2020}. Software to run the analyses and produce the figures is
  available as R code hosted in the public Git repository at
  \url{https://github.com/EarthSystemDiagnostics/optimalcores} (last access:
  28~July 2021); a snapshot of this software code at the time of publication is
  archived on the Zenodo database under
  \url{https://doi.org/10.5281/zenodo.5075439} \citep{Munch2021}.
}

\authorcontribution{TM and TL designed the research and developed the
  methodology. MW contributed by providing the climate model data and with his
  expertise on the modelling of precipitation isotopic composition. TM processed
  the model data, coded the analysis software, performed the analyses, and wrote
  the first version of the paper. All authors contributed to the interpretation
  of the results and to the preparation and revision of the final paper.}

\competinginterests{The authors declare that they have no conflict of interest.}

\disclaimer{Publisher's note: Copernicus Publications remains neutral with
  regard to jurisdictional claims in published maps and institutional
  affiliations.}

\begin{acknowledgements}
We thank Jesper Sjolte (Lund University) for performing the ECHAM5/MPI-OM-wiso
past1000 model simulation as well as Mathieu Casado, Rapha\"{e}l H\'{e}bert, and
Torben Kunz (AWI) for their helpful comments on this project and the paper. All
plots and numerical analyses in this paper were carried out using the
open-source software R: A Language and Environment for Statistical Computing. We
are grateful to the two reviewers, Lenneke Jong and Dmitry Divine, and to
Istv{\'a}n Hatvani and Zolt{\'a}n Kern, whose comments helped to significantly
improve the first version of this paper. Finally, we thank Nerilie Abram for
editing the paper.
\end{acknowledgements}

\financialsupport{This research has been supported by funding from the European
  Research Council (ERC) under the European Union's Horizon 2020 research and
  innovation programme (grant agreement no.~716092) and by Helmholtz funding
  through the Polar Regions and Coasts in the Changing Earth System (PACES)
  programme of the Alfred Wegener Institute. \newline\newline The article
  processing charges for this open-access\newline publication were covered by
  the Alfred Wegener Institute, \newline Helmholtz Centre for Polar and Marine
  Research (AWI).}

\reviewstatement{This paper was edited by Nerilie Abram and reviewed by Dmitry Divine and Lenneke Jong.}

\bibliographystyle{copernicus}
\bibliography{muench_etal_optimalcores}

\end{document}
