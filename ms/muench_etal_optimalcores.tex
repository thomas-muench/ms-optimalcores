\documentclass[draft]{agujournal2019}
\usepackage{url}
\usepackage{lineno}
\usepackage{soul}
\usepackage{textcomp}
\usepackage{amsmath}
\usepackage{amssymb}
\linenumbers

\draftfalse

\journalname{JGR: Atmospheres}

\begin{document}

\title{How Precipitation Intermittency Sets an Optimal Spatial Sampling
  Configuration for Antarctic Ice Cores}

\authors{Thomas M\"{u}nch\affil{1}, Martin Werner\affil{2}, and Thomas
  Laepple\affil{1,3}}

\affiliation{1}{Alfred-Wegener-Institut Helmholtz-Zentrum f{\"u}r Polar- und
Meeresforschung, Research Unit Potsdam, Telegrafenberg A45, 14473 Potsdam,
Germany}
\affiliation{2}{Alfred-Wegener-Institut Helmholtz-Zentrum f{\"u}r Polar- und
Meeresforschung, Bussestra{\ss}e 24, 27570 Bremerhaven, Germany}
\affiliation{3}{University of Bremen, MARUM~--~Center for Marine Environmental
  Sciences and Faculty of Geosciences, 28334 Bremen, Germany}

\correspondingauthor{Thomas M\"unch}{thomas.muench@awi.de}

\begin{keypoints}
\item We analyse Holocene climate model data to find the optimal locations where
  to sample Antarctic ice cores
\item The spatial correlation scales of the temperature variations and of
  precipitation intermittency set an effective sampling length scale
\item A single core should be located at the target site of the temperature
  reconstruction, a second one optimally lies more than 500 km away
\end{keypoints}

\begin{abstract} Many palaeoclimate proxies share one challenging property: they
are not only driven by the climatic variable of interest, e.g., temperature, but
they are also influenced by secondary effects which cause, among other things,
increased variability, frequently termed \emph{noise}. Noise in individual proxy
records can be reduced by averaging the records, but the effectiveness of this
approach depends on the correlation of the noise between the records and
therefore on the spatial scales of the noise-generating processes. Here, we
review and apply this concept in the context of Antarctic ice-core isotope
records to determine which core locations are best suited to reconstruct
local-to-regional-scale temperatures. Using data from a past-millennium climate
model simulation equipped with stable isotope diagnostics we intriguingly find
that even for a local temperature reconstruction the optimal sampling strategy
is to combine a local ice core with a more distant core $\sim500$--$1000$\,km
away. A similarly large distance between cores is also optimal for
reconstructions that average more than two isotope records. We show that these
findings result from the interplay of the two spatial scales of the correlation
structures associated with the temperature field and with the noise generated by
precipitation intermittency. Our study helps to maximize the usability of
existing Antarctic ice cores and to optimally plan future drilling campaigns. It
also broadens our knowledge on the processes that shape the isotopic record and
their typical correlation scales. Finally, the presented method can be directly
extended to determine optimal sampling strategies for other palaeoclimate
reconstruction problems.
\end{abstract}

\section{Introduction}\label{intro}

The oxygen and hydrogen isotopic composition of firn and ice recovered from
polar ice cores is a key proxy for past near-surface atmospheric temperature
changes \cite{Dansgaard1964,Lorius1969,Masson-Delmotte2008,Sjolte2011}.
Although the physical mechanisms that link local changes in temperature to the
isotopic composition of precipitated snow are generally well understood
\cite{Dansgaard1964,Craig1965,Jouzel1984} and can be modeled with general
circulation models
\cite{Joussaume1984,Werner2011,Werner2016,Sjolte2011,Goursaud2018}, the
quantitative interpretation of ice-core isotope variability, in terms of
temperature variability, is complicated by second-order processes that
influence the isotopic record, adding noise \cite{Munch2018a}.

Specifically, the isotopic record that is derived from an ice core is the result
of a chain of processes: (1) atmospheric temperature changes along with (2) isotopic
fractionation during the pathway from atmospheric moisture to precipitation, (3)
the effect of variable and intermittent precipitation and finally (4) local
depositional and post-depositional effects. Each element of this chain can be
associated with a typical spatial length scale over which it is correlated, as
will be outlined in the following.

Atmospheric temperature variations drive the isotopic composition fractionation
of the atmospheric moisture along its pathway to the final stage of
precipitation \cite{Dansgaard1964,Jouzel1984}. The spatial coherence of the
temperature-related isotopic signal in precipitation is hence determined by the
spatial coherence of the variations of the atmospheric temperature field
itself. Typical spatial decorrelation scales for temperature anomalies are on
the order of $\gtrsim1000$\,km \cite{Jones1997}, which implies that ice cores
distributed on spatial scales below $\sim 1000$\,km should typically record a
similar, i.e., correlated, temperature signal. However, the temporal variability
of the isotopic composition in the local atmospheric moisture also depends on
the variability of the atmospheric circulation, since different air masses may
exhibit different source regions and distillation pathways \cite{Schlosser2004,
Sodemann2008a,Birks2009,Kuttel2012}. In addition, the isotopic composition
profile across a deposited layer of snow will not directly reflect the temporal
variability of the atmospheric isotopic signal due to the intermittent nature of
precipitation \cite{Schleiss2015}. By this, the initial isotope signal is
weighted with the amount of precipitation, which introduces bias
\cite{Steig1994,Laepple2011a} and adds additional variability to the isotopic
record \cite{Persson2011,Casado2020}. The latter two processes are linked to
atmospheric dynamics and their typical spatial scales range from the mesoscale
(i.e., tens of kilometers), driven by topography and orographic effects, to
synoptic scales of hundreds of kilometers, associated with cyclonic activity and
the movement of high and low pressure systems. Finally, in polar conditions, the
precipitated snow does not directly settle but is constantly eroded, blown away,
and redeposited. These depositional processes have been shown to give rise to
stratigraphic noise in the isotopic record
\cite{Fisher1985,Munch2016,Laepple2016}, which exhibits a small-scale
decorrelation scale of a few meters \cite{Munch2016}.%
\footnote{We note that the final isotopic record is also influenced by
  potential snow--atmosphere vapour exchange at the surface and by
  post-depositional processes within the snow and ice matrix, such as
  densification and diffusion. These processes are, however, not within the
  scope of this article.}

This hierarchy of the different spatial scales of the processes influencing an
isotope record determines the effectiveness of reducing the overall noise, since
a reduction in the noise level by averaging records will depend on the spatial
correlation scale of the different noise sources. For example, if an isotope
record were only shaped by temperature variations and stratigraphic noise, it
would be sufficient to average records spaced only tens of meters apart, as this
would ensure highly correlated temperature signals but uncorrelated
stratigraphic noise between the records. However, comparing correlation-based
signal-to-noise ratios derived from nearby isotope records
\cite{Munch2016,Munch2017} with the signal-to-noise ratios estimated from
analysing the records' temporal variability \cite{Laepple2018} shows that
reproducibility on a local scale does not necessarily imply a climatic, i.e.,
temperature-driven, origin. Instead, circulation variability and precipitation
intermittency act as additional noise sources which are likely to exhibit larger
decorrelation lengths than the stratigraphic noise
\cite{Laepple2018,Munch2018a}. Taking this into account, we expect there to be
an optimal length scale, which lies in between the local and the temperature
decorrelation scales and which results in a trade-off between averaging out
atmospheric circulation and precipitation intermittency effects, while also
ensuring a sufficient coherence in the recorded temperature signal.

The aim of the present study is to use data from a climate model equipped with
stable isotope diagnostics to systematically study the different typical process
scales~--~including those from atmospheric temperature variations, circulation
variability, precipitation intermittency and the isotope--temperature
relationship~--, to determine the optimal spatial arrangement of ice-core
locations, which maximizes the correlation with temperature at a specific target
site. To address this problem we focus on target sites on the East Antarctic
Plateau. Our results show that the average of multiple ice-core isotope records
yields a higher degree of correlation with temperature when the sampled
locations are spread across distances of $1000$\,km or more from the target
site, than when they are all located close ($<250$\,km) to the target
site. While these results may seem counterintuitive at first, we qualitatively
explain their general features with a simple analytical model that uses the
typical spatial correlation structures associated with the temperature and
isotope fields, and with the noise generated by precipitation intermittency.

\section{Data and Methods}\label{methods}

\subsection{Climate Model Data}\label{methods:data}

We use data from the past-millennium simulation
\cite<800--1999\,CE;>{Sjolte2018} of the fully coupled ECHAM5/MPI-OM-wiso
atmosphere--ocean general circulation model equipped with stable isotope
diagnostics \cite{Werner2016}. This simulation is forced by greenhouse gases,
volcanic aerosols, total solar irradiance, land use changes, and changes in the
Earth's orbital parameters. The model's atmospheric component ECHAM5-wiso is run
with a T31 spectral resolution ($3.75$\textdegree$\times3.75$\textdegree) and
with $19$ vertical levels \cite{Sjolte2018}. Compared to observations, the
climatological relationship between temperature and the precipitation isotopic
composition is reproduced well by the model, but it is biased towards warm
temperatures in the T31 setup and its isotopic composition is not depleted
enough over Antarctica \cite{Werner2011}. These issues can be improved upon by
using a higher spatial resolution \cite{Werner2011}; however, such a
higher-resolution model is not needed for our study, since we are mainly
interested in the relative variability between sites and not in the absolute
temperature or isotope values. The full atmosphere--ocean model was compared to
observational data and palaeoclimate records for two equilibrium simulations
under pre-industrial and Last Glacial Maximum conditions \cite{Werner2016}, and
the past-millennium simulation was used to reconstruct North Atlantic
atmospheric circulation in combination with ice-core isotope data
\cite{Sjolte2018}.

In this study, we use the 1200-year ECHAM5/MPI-OM-wiso time series of two-meter
surface air temperature ($T_{2\mathrm{m}}$), precipitation ($p$), and oxygen
isotopic composition in precipitation (the relative abundance of oxygen-18 to
oxygen-16 istopes, denoted as $\delta^{18}\mathrm{O}$) extracted from all model
grid cells on the Antarctic continent ($N_{\mathrm{grid}}=442$).

\subsection{Data Processing}\label{methods:prc}

The model simulation output has a monthly temporal resolution, while typically
ice-core isotope records exhibit an annual (or even lower) resolution. The
latter is commonly achieved by averaging the isotopic data across annual layers
of snow and ice, which are determined through a dating approach. The resulting
annual isotopic composition data therefore include a weighting effect due to the
intra-annual variability in the amount of precipitation. To account for this, we
produce two versions of annual data from the monthly model output: (1) the
two-meter temperature and oxygen isotopic composition data are averaged to an
annual resolution without any weighting (denoted as $T_{2\mathrm{m}}$ and
$\delta^{18}\mathrm{O}$ in the following), and (2) the respective monthly data
are averaged to an annual resolution including the weighting by the monthly
precipitation amount (denoted as precipitation-weighted data
$T_{2\mathrm{m}}^{\mathrm{(pw)}}$ and $\delta^{18}\mathrm{O}^{\mathrm{(pw)}}$).

\subsection{Data Analyses}\label{methods:main}

\subsubsection{General Approach}\label{methods:general}

We investigate the relationships among the model variables by assessing the
Pearson correlation coefficient ($r$). To derive implications for actual
ice-core studies, we use the $\delta^{18}\mathrm{O}^{\mathrm{(pw)}}$ time series
at the model grid cells as a surrogate for ice-core isotope records. We thus
neglect stratigraphic noise and any further depositional or post-depositional
effects on the isotopic record, since we are interested in the upper limit of
the extent to which ice cores can reconstruct the climatic temperature signal in
the atmosphere. Our analyses are conducted relative to specified grid cells of
interest (target sites; $\mathbf{r}_0$) to obtain results that are relevant on
local-to-regional spatial scales.

\subsubsection{Picking Optimal Sites}\label{methods:picking}

To determine an optimal set of ice-core locations to reconstruct
$T_{2\mathrm{m}}$ at a given target site we first randomly pick without
replacement a number $N$ of the grid cells that lie within a circle of
$2000$\,km radius around the target site and then correlate the average
$\delta^{18}\mathrm{O}^{\mathrm{(pw)}}$ time series from these $N$ grid cells
with the temperature at the target site. The optimal set of cores for each $N$
is then determined from the maximum correlation value across all trials: For
$N=1$, we can directly pick the optimal location from the maximum correlation
value within the circle without random sampling; for $N>1$, we set the maximum
number $n$ of picking trials to $10^5$ to ensure stable results.

%f01
\begin{figure}[t]%
\centering
\includegraphics[width=6.5cm]{../plots/main/fig_01.pdf}
\caption[Conceptual approach]{%
  Conceptual sketch of the general approach. Around a given Antarctic target
  site (black cross), we define consecutive rings (red lines) of $250$\,km
  radial width and analyse all model grid cells that fall within each of the
  rings. Also shown are our main study regions (black polygons) around the EDML
  (upward pointing triangle) and Vostok (downward pointing triangle) ice-core
  sites.}
\label{fig:concept}%
\end{figure}%

\subsubsection{Optimal Sampling Structure}\label{methods:opt.sampling}

To learn about the typical spatial scales associated with the processes
contributing to the overall temperature--isotope relationship we use a more
general approach that reduces local effects in the climate model data. We choose
a given target site and define consecutive rings around this site with a
$250$\,km radial width until a maximum distance of $2000$\,km is achieved
(Figure~\ref{fig:concept}). Then, we identify all the grid cells that fall into
each of these rings and randomly sample $N$ grid cells from out of these
rings. This is implemented in a two-step process: (1) we determine all possible
combinations of selecting $N$ rings with replacement, and then (2) for each
ring combination we identify the possibilities of combining grid cells by
selecting an individual grid cell from each ring. For each of these grid-cell
combinations, we average the time series for a studied model variable
($T_{2\mathrm{m}}$, $T_{2\mathrm{m}}^{\mathrm{(pw)}}$,
$\delta^{18}\mathrm{O}^{\mathrm{(pw)}}$) and compute the degree of correlation
with the target site temperature. Finally, we report the mean correlation for
every ring combination by averaging across all correlations of the analysed
grid-cell combinations. This provides insight into the average spatial structure
of the correlation with the target site temperature for sampling $N$ locations
from the model field depending on the distances between the locations. We denote
this quantity as the \emph{sampling correlation structure}. Note that in the
one-dimensional case ($N=1$), the sampling correlation structure is identical to
what is often called the spatial correlation structure, i.e., the average
correlation as a function of distance.

In the second step from the above two-step process, it is computationally
feasible to identify all possible grid-cell combinations until $N=2$. For
$N\geq3$ we resort to Monte Carlo sampling instead, for which we estimated the
required number of Monte Carlo iterations from comparing the Monte Carlo
sampling solution for $N=2$ with its exact solution, yielding sufficient
convergence for $10^4$ iterations. Based on this, we choose $10^5$ iterations
for sampling $N\geq3$ locations, since this larger number of locations involves
a larger number of possible ring combinations and thus many more possible
grid-cell combinations.

\subsubsection{Study Regions}\label{methods:regions}

We focus our analyses predominantly on two subregions of the East Antarctic
Plateau, the Dronning Maud Land (DML) region and the Vostok region, both of
which include existing deep ice-core drilling sites as well as large arrays of
shallower ice and firn cores.

For the DML region, we choose all model grid cells ($N_{\mathrm{grid}}=26$)
within a range of $\pm17.5$\textdegree\ longitude and $\pm5$\textdegree\
latitude around the European Project for Ice Coring in Antarctica (EPICA) DML
site (EDML; $-75$\textdegree\,S, $0$\textdegree\,E;
Figure~\ref{fig:concept}). This region encompasses the site of the deep EDML ice
core \cite{EPICAcommunitymembers2006,awi2016} and $>50$ firn and shallow ice
cores \cite{Altnau2015}. For the Vostok region, we choose an identical
latitudinal and longitudinal coverage ($N_{\mathrm{grid}}=30$) with respect to
the Vostok station ($-78.47$\textdegree\,S, $106.83$\textdegree\,E;
Figure~\ref{fig:concept}), encompassing the deep Vostok and Dome C ice cores,
several shallower cores \cite{Stenni2017}, and the new deep drilling site
(``Little Dome C'') where an ice core extending back more than one million years
is envisaged \cite{Passalacqua2018}.

\section{Results}\label{results}

\subsection{Spatial Scale of the Temperature Anomalies and the Local
  Temperature--Isotope Relationship}
\label{results:t2m-iso}

First, we asses the extent to which a local ice-core record, i.e., the annual
isotope time series of a single grid cell in the model simulation, is
representative of the local and regional scale variability of the near-surface
atmospheric temperature.

The temperature field over Antarctica in the climate model exhibits large scale
coherent variations (Figure~\ref{fig:t2m.decorrelation.map}) with a clear two-part
structure, which is roughly divided by the Transantarctic Mountain range: For
most parts of the East Antarctic Plateau, the temperature field shows typical
decorrelation lengths between $\sim1500$ and $2500$\,km, while the decorrelation
lengths are significantly lower with values $\lesssim1000$\,km for larger parts
of the West Antarctic Ice Sheet and for the Antarctic Peninsula.  Still, for
perfect ice cores, i.e., assuming an ideal temperature proxy record that is only
governed by local temperature variations, a single ice core would capture the
temperature variability in both East and West Antarctic regions across hundreds
of kilometers.

%f02
\begin{figure*}[t]%
\centering
\includegraphics[width=8.5cm]{../plots/main/fig_02.pdf}
\caption{%
  Temperature decorrelation lengths across Antarctica. The temperature
  decorrelation lengths ($\tau$, in km) for each Antarctic model grid cell were
  estimated by fitting an exponential model to the correlation--distance
  relationship (cf.\ equation~\ref{eq:t2m.decorr}) obtained from correlating the
  local annual near-surface $T_{2\mathrm{m}}$ time series with the respective
  temperature time series from all other grid cells. Note that only the
  continental grid cells were used for the fit. Although the decorrelation
  lengths show a strong partition between East and West Antarctica, they
  are larger than $1000$\,km at most locations.}
\label{fig:t2m.decorrelation.map}%
\end{figure*}%

%f03
\begin{figure*}[t]%
\centering
\includegraphics[width=16cm]{../plots/main/fig_03.pdf}
\caption{%
  The local temperature--isotope relationship across Antarctica. Shown are the
  local correlations for each model grid cell between the annual time series of
  precipitation-weighted oxygen isotope composition and of (\textbf{a})
  near-surface temperature and (\textbf{b}) precipitation-weighted near-surface
  temperature. The difference between the maps clearly demonstrates that
  precipitation intermittency is a major limiting factor for the
  temperature--isotope relationship.}
\label{fig:t2m.oxy.correlation.maps}%
\end{figure*}%

However, as simulated by the isotope-enabled climate model, actual single
Antarctic ice-core isotope records only explain a low portion of the variations
in the local and regional temperature fields: Correlating the annual
precipitation-weighted field of modeled
$\delta^{18}\mathrm{O}^{\mathrm{(pw)}}$, the model variable which most closely
mimics a real ice-core record, locally with the annual $T_{2\mathrm{m}}$ time
series results in generally low correlations (mean of $0.36$), which across all
analysed grid cells range from $<0.1$ up to $\sim0.53$ with $\sim70\,\%$ of the
correlations $\leq0.4$ (Figure~\ref{fig:t2m.oxy.correlation.maps}a). The
correlations are improved when the $T_{2\mathrm{m}}^{\mathrm{(pw)}}$ time series
is used instead of the $T_{2\mathrm{m}}$ time series with a mean local
correlation of $0.51$ (range $\sim0$ to $0.77$;
Figure~\ref{fig:t2m.oxy.correlation.maps}b). This shows that precipitation
intermittency is a major limiting factor for the temperature--isotope
correlation. In the following sections, we assess the extent to which the
correlation with temperature can be increased and how this relates to the
spatial scales studied.

\subsection{Choosing Optimal Ice-Core Sites for Temperature Reconstructions}
\label{results:picking}

The above analysis shows that isotope records from single ice cores likely only
capture a small portion of the local interannual temperature variability. This
suggests that additional processes, such as precipitation intermittency,
influence the isotopic signal and decrease the degree of correlation with the
local temperature record. Interpreting these additional processes as noise
raises the question of whether the correlation with temperature can be improved
upon by averaging isotope records across space. To address this question, we
first assume an ideal world in which the climate model data are a perfect
surrogate for the true climate and proxy variations at each site. We then set up
a simple experiment where $\delta^{18}\mathrm{O}^{\mathrm{(pw)}}$ grid cells are
randomly picked and averaged to determine what spatial array of $N$ ice cores
optimizes the temperature correlation with a target site.

For our specific model simulation and specifying the EDML drilling site as the
target site, we obtain the interesting result that the optimal location for a
single ice core is not the local grid cell as one might expect, but a site
$\sim1100$\,km away from the target towards the southeast
(Figure~\ref{fig:picking}a). Choosing this more distant site increases the
correlation with the target temperature from an $r$ value of $0.26$ for the
local EDML site to a value of $0.44$. Furthermore, by analysing the maximum
correlations with the EDML target temperature for an average of three or five
cores (Figure~\ref{fig:picking}b--c) we find optimal locations that in both cases
are scattered at significant distances around the target and which yield an even
further increase in correlation ($r=0.49$ in both cases). We obtain comparable
results when the Vostok drilling site is specified as the target
(Figure~\ref{fig:picking}d--f). The optimal single core would be at a location
$\sim420$\,km north of Vostok ($r=0.45$, compared to the local correlation of
$r=0.34$), and the optimal locations for averaging three or five cores all lie
again scattered around the target without including it, and, as for EDML, result
in a significant increase in correlation for $N=3$ ($r=0.57$) but in no further
increase for $N=5$ ($r=0.56$).

%f04
\begin{figure*}[t]%
\centering
\includegraphics[width=16cm]{../plots/main/fig_04.pdf}
\caption[Picking optimal sites]{%
  Choosing ice-core locations that optimally reconstruct interannual
  temperatures at the EDML and Vostok drilling sites. The maps show the
  correlation coefficient in the climate model data between the annual
  temperature time series at the target sites (black crosses) EDML
  (\textbf{a}--\textbf{c}) and Vostok (\textbf{d}--\textbf{f}) with the time
  series fields of precipitation-weighted oxygen isotope composition. Filled
  black circles denote grid cells that maximize the correlation between the
  target site temperature and either a single grid cell ($N=1$; \textbf{a},
  \textbf{d}) or for an average across $N=3$ (\textbf{b}, \textbf{e}) or $N=5$
  (\textbf{c}, \textbf{f}) grid cells. Interestingly, non-local ice-core
  locations systematically show the strongest relationship with the target site
  temperature.}
\label{fig:picking}%
\end{figure*}%

We generalize these findings by considering each Antarctic model grid cell as a
target site and determining in each case the ice core location that results in
an optimal correlation with the target site. Similarly to the above case
studies, the majority ($\sim67$\,\%) of optimal locations for a single ice core
are situated at distances between $400$ and $1000$\,km from the respective
target sites, while only about $20$\,\% lie within $400$\,km from the
targets. We note that this distribution might be affected by the number of
available sampling points (i.e., model grid cells) per distance bin which
increase with increasing distance from the target site.

\subsection{Optimal Ice-Core Sampling Structures}
\label{results:optim-spacing}

The approach for choosing optimal ice-core locations yields straightforward and
instructive results. However, it might be doubtful as to whether these results
can be directly applied to the real world, since they might depend on the
specific simulated climate state or result from statistical overfitting. Thus,
as a next step, we adapt our approach to learn more about the general spatial
arrangement of the optimal ice-core locations which yield the maximum
correlation with temperature. To address this issue, we compute the mean of
correlation results obtained between a target site temperature and individual
grid cells in order to reduce local variability in the model data. We perform
this averaging step across several combinations of $250$\,km wide concentric
rings with a target site at the centre (``sampling correlation structure'';
Figure~\ref{fig:concept} and section~\ref{methods:opt.sampling}) to derive
results which are only a function of radial distance. Additionally, if
applicable, we average the obtained results across the target sites within our
defined DML and Vostok regions (section~\ref{methods:regions}) to get regional
estimates. Finally, we analyse each of the model variables to highlight the
differences between the individual fields.

%f05
\begin{figure*}[t]%
\centering
\includegraphics[width=12cm]{../plots/main/fig_05.pdf}
\caption{%
  Sampling correlation structures with temperature for the DML and Vostok
  regions in the case of sampling single locations. Shown is the average
  correlation as a function of distance between the interannual near-surface
  temperature ($T_{2\mathrm{m}}$) at a target site and the spatial fields of
  $T_{2\mathrm{m}}$ (black), oxygen isotope composition
  ($\delta^{18}\mathrm{O}$, green) and precipitation-weighted oxygen isotope
  composition ($\delta^{18}\mathrm{O}^{\mathrm{(pw)}}$, blue). Averaging was
  performed in two steps: first, correlations were averaged across grid cells
  falling within $250$\,km wide consecutive rings around a given target site,
  and secondly, the results were averaged across all respective target sites in
  the DML (\textbf{a}) and Vostok (\textbf{b}) region (see Methods). The black
  dashed lines indicate an exponential fit to the $T_{2\mathrm{m}}$ data.}
\label{fig:avg.cor.structure}%
\end{figure*}%

The sampling correlation structure from this approach is, when we sample only a
single location ($N=1$), conceptually equivalent to the average correlation with
distance, and it therefore simply gives the spatial decorrelation in the case of
sampling from the $T_{\mathrm{2m}}$ field itself. Indeed, the sampling
correlation structures for $T_{\mathrm{2m}}$ in the DML and Vostok region
(Figure~\ref{fig:avg.cor.structure}) can be described by an exponential decay with
a length scale of $\sim1900$\,km in both cases, consistent with the estimated
spatial decorrelation lengths on the local scale
(Figure~\ref{fig:t2m.decorrelation.map}). We note that these results show that
the maximum average correlation with the target site temperature is obtained
from sampling the innermost ring only, consistent with the general expectation.

When we compare these results to the sampling correlation structure for the
$\delta^{18}\mathrm{O}$ field, we find in the DML region a much lower average
correlation with the target site temperature as a function of distance
(Figure~\ref{fig:avg.cor.structure}a). The average local ($<250$\,km) correlation
is $\sim0.4$, but decreases only slightly within the first $\sim1000$\,km,
followed by a little steeper decrease and near constant levels of $r\lesssim0.2$
for distances $\gtrsim1700$\,km. For the Vostok region
(Figure~\ref{fig:avg.cor.structure}b), the sampling correlation structure for
$\delta^{18}\mathrm{O}$ exhibits a nearly linear decrease from an initial value
of $r\gtrsim0.5$ to $r\sim0.1$ in the final ring ($>2000$\,km). When we analyse
the $\delta^{18}\mathrm{O}^{\mathrm{(pw)}}$ fields we find that precipitation
weighting overall induces even lower correlation values in both regions, but
that it does not have a large effect on the sampling correlation structure
itself.

%f06
\begin{figure*}[t]%
\centering
\includegraphics[width=17cm]{../plots/main/fig_06.png}
\caption{%
  Sampling correlation structure with temperature in the two-dimensional case
  of sampling two locations in the DML region. Shown is the mean correlation of
  all possible single correlations for the average of two grid cells of
  (\textbf{a}) $T_{\mathrm{2m}}$, (\textbf{b}) $T_{\mathrm{2m}}^{\mathrm{(pw)}}$
  and (\textbf{c}) $\delta^{18}\mathrm{O}^{\mathrm{(pw)}}$ time series sampled
  from the same ring or from two different rings, averaged over all target sites
  in the given region. The axes display the distance from the target site, where
  the $x$ ($y$) axis represents the first (second) sampled ring and the tick
  marks indicate the midpoint radii of the rings. Note the marked difference
  in the locations of the correlation maxima between $T_{\mathrm{2m}}$ and
  $\delta^{18}\mathrm{O}^{\mathrm{(pw)}}$.}
\label{fig:two-core-correlation}%
\end{figure*}%

Extending this analysis to the two-dimensional case of sampling and averaging
$N=2$ locations offers the possibility of investigating the average correlation
not only as a function of distance from the target site but also as a function
of distance between the two sampled locations
(Figure~\ref{fig:two-core-correlation}). We find that the difference in the
sampling correlation structure between the fields of $T_{\mathrm{2m}}$ and
$\delta^{18}\mathrm{O}^{\mathrm{(pw)}}$ is even more pronounced for $N=2$ than
for $N=1$. As one would expect, the maximum average correlation for
$T_{\mathrm{2m}}$ is still found when both sampling locations are from the
innermost ring, as shown for the DML region
(Figure~\ref{fig:two-core-correlation}a). However, for
$\delta^{18}\mathrm{O}^{\mathrm{(pw)}}$ the optimal arrangement of two locations
to obtain the maximum average correlation is to sample one location from the
innermost ring and the second location from the fifth ring, i.e., between
$\sim1000$ and $1250$\,km from the target site
(Figure~\ref{fig:two-core-correlation}c). Part of this structure is related to the
effect of precipitation intermittency, which can be seen from the sampling
correlation structure of the $T_{\mathrm{2m}}^{\mathrm{(pw)}}$ field
(Figure~\ref{fig:two-core-correlation}b). In contrast to $T_{\mathrm{2m}}$, the
correlation is about as high when we combine the innermost ring and one ring
further away, as when we sample both locations from the innermost ring.

Analysing the Vostok study region leads to comparable results
(Appendix~\ref{app:vostok.n2}: Figure~\ref{fig:two-core-correlation-vostok}), with
a similar difference in sampling correlation structure between $T_{\mathrm{2m}}$
and $T_{\mathrm{2m}}^{\mathrm{(pw)}}$ as for the DML region, and a similar
structure of $T_{\mathrm{2m}}^{\mathrm{(pw)}}$ and
$\delta^{18}\mathrm{O}^{\mathrm{(pw)}}$ for distances $\lesssim1000$\,km.
However, the results for the $\delta^{18}\mathrm{O}^{\mathrm{(pw)}}$ field
(Figure~\ref{fig:two-core-correlation-vostok}c) do not display such a pronounced
maximum correlation when one location is sampled from the innermost ring and the
second one from a ring further away as is observed for the DML region. This
suggests the regional differences in the spatial correlation structure of the
$\delta^{18}\mathrm{O}$ field (Figure~\ref{fig:avg.cor.structure}) to have an
influence here.

The general feature of the optimal $\delta^{18}\mathrm{O}^{\mathrm{(pw)}}$
sampling arrangement is robust throughout Antarctica, despite the above regional
differences. When we analyse all available Antarctic target sites, setting the
first location to the innermost ring and looking for an optimal ring of the
second location, in which the average correlation with the target site
temperature is maximal, we find that in $\sim77\,\%$ of all cases the optimal
configuration for the second location is at least the second ring ($>250$\,km),
and in $\sim61\,\%$ of the cases it is within the second to fourth ring
($250$--$1000$\,km).

%f07
\begin{figure*}[t]%
\centering
\includegraphics[width=16cm]{../plots/main/fig_07.pdf}
\caption{%
  The optimal arrangement for averaging three or five
  $\delta^{18}\mathrm{O}^{\mathrm{(pw)}}$ ice cores to reconstruct the target
  site temperature at the EDML (\textbf{a}, \textbf{c}) and Vostok (\textbf{b},
  \textbf{d}) drilling sites. Displayed are subsets of the sampling correlation
  structures for $N=3$ and $5$, showing along the vertical axis the optimal five
  of all possible combinations of rings, i.e., those which exhibit the highest
  mean correlation across $10^5$ random trials of averaging $N=3$ (\textbf{a},
  \textbf{b}) or $N=5$ (\textbf{c}, \textbf{d}) grid cells from these rings. The
  ring bin borders are marked by thin vertical lines with their distances from
  the target site given on the horizontal axis; the selected optimal ring
  combinations are marked as black dots. Systematically, arrangements with
  several ice cores sampled at $500$ to $1000$\,km distances are found to be
  optimal.}
\label{fig:binning}%
\end{figure*}%

Furthermore, we obtain similar results also when averaging $N=3$ or $5$
locations of the $\delta^{18}\mathrm{O}^{\mathrm{(pw)}}$ field to reconstruct
the target site temperature (Figure~\ref{fig:binning}). When EDML is set as the
target site, the optimal sampling configuration is such that one location lies
in the innermost ring while the others are distributed at distances between
$\sim500$ and $1500$\,km from the target. For reconstructing the Vostok target
site temperature, the optimal locations are mostly distributed across the second
to third ($250$--$750$\,km) ring. In summary, averaging the
$\delta^{18}\mathrm{O}^{\mathrm{(pw)}}$ time series across several locations
clearly increases the average correlation with the target site temperature, if
this averaging follows an optimal combination of rings, as compared to sampling
all locations only locally (Figure~\ref{fig:cor.increase.risk}a). The increase in
correlation becomes larger by averaging more locations: while the local
correlation stays constant at $0.27$ (EDML) and $0.34$ (Vostok), the optimal
correlation rises for $N=2$ to $0.32$ and $0.40$, respectively, and for $N=10$
to $0.39$ and $0.49$. This is equivalent to nearly a doubling in the explained
variance.

We note that these results are the mean value from averaging across many
possible combinations of individual locations. In reality, any new drilling
campaign or reanalysis of existing ice cores only represents one single
combination of locations. Therefore, we assess the risk of an ``adverse optimal
sampling'', i.e., the probability for choosing by chance a specific sampling
realisation from the optimal ring combination which yields a lower correlation
than the correlation for sampling locally. For this purpose, we compare the
distribution of all individual correlations from sampling the optimal ring
combination with the value obtained from sampling only the local sites which lie
in the innermost ring. Overall we find the risk of adverse optimal sampling to
be low, since more than $92\,\%$ of all individual correlation values in the
example of $N=3$ are actually larger than the respective local correlation
(Figure~\ref{fig:cor.increase.risk}b).

%f08
\begin{figure*}[t]%
\centering
\includegraphics[width=14cm]{../plots/main/fig_08.pdf}
\caption{%
  Gain in correlation and risk of adverse sampling. (\textbf{a}) The average
  correlation with the target temperature at the EDML (red) and Vostok (blue)
  sites depending on the number of locations, $N$, used for averaging the
  $\delta^{18}\mathrm{O}^{\mathrm{(pw)}}$ time series. Sampling is performed
  either locally from the innermost ring only (dashed lines), or from all
  possible individual combinations of locations for the respective optimal ring
  combination determined for each $N$ (solid lines). Compared to the local
  samples which show virtually no increase with the number of sampled locations,
  the correlation increases markedly with $N$ when sampling from the optimal
  rings, as highlighted by the shaded area. (\textbf{b}) Histogram of all
  possible individual correlations for sampling from the optimal ring
  combination when averaging $N=3$ locations compared to the correlation
  (vertical lines) for sampling from the innermost ring only, displayed for the
  EDML (red) and Vostok (blue) target sites. For more than $90\,\%$ of the
  optimal ring combination samples, the correlation is higher than the local
  value.}
\label{fig:cor.increase.risk}%
\end{figure*}%

\section{Discussion}\label{discussion}

Oxygen isotope records derived from ice cores are commonly interpreted to
reflect local temperature changes at the ice-core drilling site. Here, in a
systematic study of analysing the interannual correlation between
precipitation-weighted oxygen isotope composition and near-surface atmospheric
temperature in a climate model, we showed that while there is local
isotope--temperature correlation (Figure~\ref{fig:t2m.oxy.correlation.maps}a),
this correlation can be increased considerably by averaging isotope records
across space following a distinct spatial pattern
(Figure~\ref{fig:cor.increase.risk}a) which combines the local target site with
locations located between a few hundred kilometers to up to $\sim1000$\,km from
the target site (Figs.~\ref{fig:two-core-correlation}c, \ref{fig:binning} and
\ref{fig:two-core-correlation-vostok}c). In the next section, we develop a
qualitative understanding of these results from a conceptual model that predicts
the sampling correlation structure from the processes that shape the isotopic
composition time series, before discussing the relevance of our results to
actual ice-cores studies.

\subsection{Conceptual Model of the Optimal Sampling Structure}
\label{discussion:concept.model}

For a conceptual model of the sampling correlation structure, we focus on three
processes that influence the oxygen isotope records in ice cores: (i)
temperature variations, (ii) precipitation intermittency, and (iii) the
temperature--isotope relationship. We statistically model the associated fields
of $T_{\mathrm{2m}}$, $T_{2\mathrm{m}}^{\mathrm{(pw)}}$ and
$\delta^{18}\mathrm{O}^{\mathrm{(pw)}}$ separately in order to understand the
influence of each process (see Appendix~\ref{app:concept.model} for details),
and we assess, for comparable results, the predicted average sampling
correlation structure with the target site temperature in the two-dimensional
case of averaging two locations in the same manner that we analysed the climate
model data.

To model the atmospheric temperature field, we assume an isotropic exponential
decay of the spatial correlation with a constant decorrelation length
(Appendix~\ref{app:concept.model.t2m}). Such an exponential temperature
decorrelation is a commonly observed feature \cite{Jones1997} and also
confirmed by our climate model data (Figure~\ref{fig:t2m.decorrelation.map} and
Figure~\ref{fig:avg.cor.structure}). Given this relationship, we find a good
agreement for the two-dimensional sampling correlation structure between the
conceptual model and the climate model data, both regarding absolute correlation
values as well as the spatial pattern (Figure~\ref{fig:conceptual.model}a). We
emphasize that the maximum correlation with the target site temperature
naturally occurs, in case of an isotropic correlation decay, when the averaged
two (or $N$) locations are close to the target site, as any location which is
further away will result in a temperature signal that is less similar between
the locations.

To elucidate the role of precipitation intermittency, we follow the simplest
assumption which is that this process can be described by partly aliasing the
original temperature signal into temporal white noise
\cite{Laepple2018,Casado2020}. We further assume that this noise is not
independent between sites but that it follows the spatial scale of precipitation
events, which we describe as an exponential decorrelation in space with a second
length scale (Appendix~\ref{app:concept.model.t2m.pw}). This intermittency
length scale is related to the atmospheric processes that deliver precipitation,
e.g., synoptic systems, and is hence assumed to be smaller than the length scale
of the temperature anomalies. The introduction of this second length scale into
our conceptual model generally explains the optimal sampling structure we
obtained from the climate model data. Qualitatively, close-by locations exhibit
a strong correlation in temperature but also in the noise from precipitation
intermittency; therefore, this noise cannot be reduced by averaging the
locations, yielding an overall low signal-to-noise ratio. However, with
increasing distance between the locations, the intermittency noise decorrelates
faster than the temperature field due to the different decorrelation scales,
resulting in an optimal distance of maximum signal-to-noise ratio. This is also
reflected in our conceptual model (Figure~\ref{fig:conceptual.model}b, e). When
fixing one location to the target site and varying the distance from the target
site of the second location, the correlation with the target site temperature
first increases with increasing distance of the second location and then
maximizes at an optimal distance, before it decays with a further increase in
distance. In the climate model data, we observed a similar feature for the
precipitation-weighted temperature (Figs.~\ref{fig:two-core-correlation} and
\ref{fig:two-core-correlation-vostok}), though it was not as clear as in the
conceptual model. This mismatch could be related to the assumed isotropy in the
conceptual model and the according azimuthal averaging done in the climate model
data analysis, which potentially smears the intermittency effect in the climate
model data due to slight differences in the decorrelation lengths between the
different horizontal directions.

In order to incorporate the $\delta^{18}\mathrm{O}^{\mathrm{(pw)}}$ field into
the conceptual model, we need to account for the spatial temperature--isotope
relationship. To accomplish this, we parameterize the spatial dependence of the
correlation between temperature and the oxygen isotope composition with a simple
isotropic linear model based on the climate model data results
(Figure~\ref{fig:avg.cor.structure} and
Appendix~\ref{app:concept.model.oxy.pw}). In addition, we assume that the same
effect of precipitation intermittency that we adopted for the temperature field
is also applicable to the oxygen isotope field. With these simple assumptions,
we obtain a good qualitative agreement for the DML region between the conceptual
model and the climate model data results (cf. Figs.~\ref{fig:conceptual.model}c
and \ref{fig:two-core-correlation}c). In addition, when we change the
parameterized isotope--temperature relationship such that it more closely
resembles the Vostok region data (Figure~\ref{fig:avg.cor.structure}b), the
sampling correlation structure in the conceptual model
(Figure~\ref{fig:conceptual.model}f) is more similar to the observed correlation
structure (Figure~\ref{fig:two-core-correlation-vostok}c). However, in general the
conceptual model fails for $\delta^{18}\mathrm{O}^{\mathrm{(pw)}}$ to reproduce
the actual range in correlations as it produces much lower values than expected.

In summary, our conceptual model provides a quantitative understanding of the
spatial correlation of the temperature in the climate model data, and, at least,
a qualitative understanding of the processes that affect the correlation between
temperature and the $\delta^{18}\mathrm{O}^{\mathrm{(pw)}}$ field, i.e.,
precipitation intermittency and the spatial temperature--isotope
relationship. The deficiencies in the conceptual model may be attributed to its
simplicity. For the governing processes, we assumed spatially constant and
isotropic length scales, neglecting local and direction-related differences in,
e.g., temperature decorrelation lengths
(cf.~Figure~\ref{fig:t2m.decorrelation.map}) or the spatial extent of the
coherence of precipitation intermittency. Instead of being constant, the latter
may differ depending on the type of precipitation, e.g., synoptic versus
clear-sky precipitation, and may exhibit directional dependencies related to
topography. Furthermore, we assumed constant variance for all time series,
thereby ignoring potential weighting effects on the correlations for the spatial
average of several locations due to different variabilities between them.

\subsection{Relevance for Ice-Core Studies}
\label{discussion:relevance}

Our results which we obtained from analysing the climate model data and
substantiated with our conceptual model provide guidance on where to drill
$N=1, 2, 3$ or more ice cores, or from which locations to analyse them, in order
to optimally reconstruct the atmospheric temperature signal for a certain target
site or region.

The first possibility is to follow the recommendations obtained from directly
choosing the specific locations which maximize the correlation with the target
site temperature (Figure~\ref{fig:picking}). However, it is unclear whether these
results can be one-to-one transferred to the real world, since they might depend
on dynamical processes in the atmosphere which could differ between climate
states or depend on initial conditions. One indication for this is that we
obtain different optimal single core locations for more than half of all
investigated Antarctic target sites, when we analyse only the first or the
second half of the respective climate model time series as compared to the full
1200 years.

Here we argue that the optimal spatial sampling configuration is on average
governed by the interplay of the different underlying correlation length scales,
which we expect to vary less in between different climate periods or states.
This is substantiated by the fact that the sampling correlation structures for
two cores (Figs.~\ref{fig:two-core-correlation} and
\ref{fig:two-core-correlation-vostok}), obtained from averaging the correlations
from individual sampling locations  across concentric rings around the target
site, are much more robust against analysing only the first or the second half
of the model time series, different to the results from directly choosing
optimal locations.

Using the sampling correlation structures we arrive at the following
recommendations for optimal ice core sampling configurations. If it is only
possible to drill or analyse a single ice core, our results show that it is
always best to sample locally, i.e., to place this core near the target site of
interest. This is also common practice, given the usual interpretation of
ice-core isotope records as a proxy for local temperatures. However, due to the
effect of precipitation intermittency, in the case of drilling two ice cores it
is no longer optimal to collect both cores near the target site, but instead to
drill one core at the target site and one at least $500$\,km away. Where three
or more ice cores will be drilled or analysed, we expect the optimal spatial
configuration to be more dependent on the study region. However, our results
indicate that in general it is still likely better to place one core near the
target site and distribute the others across several hundreds of kilometers.

These inferences are based on data from a single climate model simulation
together with a simple statistical conceptual model, which should be tested
against observations. We thus need to create an isotope record that is in first
order only governed by temperature variations and precipitation intermittency,
and remove the impact of local stratigraphic noise from the actual measured
records (assuming that any further processes in the pre-depositional to
depositional phase contribute negligibly to the local isotopic variations). To
accomplish this one possible strategy would be to use trench sampling campaigns
\cite<see>[for the EDML site]{Munch2016,Munch2017}. Then, one test of our
optimal sampling configurations could be to combine one trench record, e.g., one
from EDML, with another trench sampled at the optimal distance based on our
results for $N=2$, and correlate the average of these two trench records with
the instrumental temperature data set available for EDML. Based on the results
in this study we would expect a higher degree of correlation in this case
compared to using only one local trench record from EDML. We acknowledge that
such an approach would be challenging due to the small amount of available
instrumental data ($\sim20$\,years for EDML) and by the inevitable dating
uncertainties between the two trench records.

\section{Conclusions}

In this study we assessed the spatial sampling configuration of ice cores to
optimally reconstruct the annual near-surface temperature at a specific target
site. This problem was motivated by the expectation that the major processes
influencing the isotopic records of ice cores operate on different spatial
scales.

Indeed, by analysing the temperature and isotope data of an isotope-enabled
atmosphere--ocean climate model simulating the climatic history over the last
millennium in Antarctica, we showed that while in the optimal setup a single ice
core should be placed close to the target site of interest, a second core should
be located far ($>500$\,km) from the first core. While this may seem surprising
at first glance, it can be straightforwardly explained by the interplay of two
different correlation lengths in space: one for the temperature anomalies and
one parameterizing the spatial coherence of the effect of precipitation
intermittency, as demonstrated by a simple conceptual model. Despite the fact
that these results were specifically obtained for two regions of the East
Antarctic Plateau, we expect similar results to hold for other parts of
Antarctica, and potentially also for other large-scale ice-coring regions such
as Greenland.

Our study therefore explicitly improves the planning of drilling or analysis
campaigns for spatial networks of ice-core isotope records. In addition, it
provides a strategy to analyse an optimal configuration of sampling locations
for any proxy which is influenced by two or more processes that exhibit
different spatial correlation scales. This likely applies to various marine as
well as terrestrial proxy types, and our strategy thus might offer a step
forward in the best use of sampling and measurement capacity for quantitative
climate reconstructions.

\appendix

\section{Two-Dimensional Sampling Correlation Structure for the Vostok Region}
\label{app:vostok.n2}

In order to reduce the number of figures in the main text, we provide the
results of the average two-dimensional sampling correlation structures ($N=2$)
for the Vostok study region here in Figure~\ref{fig:two-core-correlation-vostok}.

%fA01
\begin{figure*}[t]%
\centering
\includegraphics[width=17cm]{../plots/main/fig_A01.png}
\caption{%
  Sampling correlation structure with temperature in the two-dimensional case of
  sampling two locations in the Vostok region. Shown is the mean correlation of
  all possible single correlations for the average of two grid cells of
  (\textbf{a}) $T_{\mathrm{2m}}$, (\textbf{b}) $T_{\mathrm{2m}}^{\mathrm{(pw)}}$
  and (\textbf{c}) $\delta^{18}\mathrm{O}^{\mathrm{(pw)}}$ time series sampled
  from the same ring or from two different rings, averaged over all target sites
  in the given region. The axes display the distance from the target site, where
  the $x$ ($y$) axis represents for the first (second) sampled ring and the tick
  marks indicate the midpoint radii of the rings. Note that for
  $\delta^{18}\mathrm{O}^{\mathrm{(pw)}}$ the -- albeit marginal -- correlation
  maximum is achieved by combining the innermost ring with the ring between
  $500$--$750$\,km.}
\label{fig:two-core-correlation-vostok}%
\end{figure*}%

\section{Conceptual Model of Sampling Correlation Structures}
\label{app:concept.model}

\subsection{General Model}
\label{app:concept.model.general}

We set up a conceptual model for the correlation between a target temperature
time series and a spatial average based on a set of locations sampled from a
climatic field (sampling correlation structure). Our model assumes simple
isotropic and exponential decorrelation structures for the involved climatic
fields and is based on previous work which suggests that precipitation
intermittency can be described by partly aliasing the original temperature
signal into white noise \cite{Laepple2018}.

In the model, we consider a temperature time series $T_0$ at some target site
$\mathbf{r}_0$ and a field $x$ of a given climate variable. From this field, we
select $N$ time series $x_i$ at the locations $\mathbf{r}_i$, $i=1,\dotsc,N$,
and denote the spatial average of these time series by
$\overline{x}=\frac{1}{N}\sum_{i=1}^{N}{x_i}$. The distances of the $N$
locations from the target site and the distances between the locations are given
by $r_i=|\mathbf{r}_i-\mathbf{r}_0|$ and by
$d_{ij}=|{\mathbf{r}_i-\mathbf{r}_j}|$, respectively. The correlation between
$T_0$ and $\overline{x}$ follows from
%
\begin{linenomath*}
\begin{equation}
\label{eq:corr.general}
\mathrm{cor}(T_0,\overline{x})=\frac
{\mathrm{cov}(T_0,\overline{x})}
{\sqrt{\mathrm{var}(T_0)\mathrm{var}(\overline{x})}},
\end{equation}
\end{linenomath*}
and it is governed by the covariance between the temperature at the target site
and the climate field at the sampling locations $\mathbf{r}_i$,
%
\begin{linenomath*}
\begin{equation}
\label{eq:cov.general}
\mathrm{cov}(T_0,\overline{x})=
\frac{1}{N}\sum_{i}^{N}{\mathrm{cov}(T_0,x_i)},
\end{equation}
\end{linenomath*}
%
and by the covariance between the sampling locations through the variance of
their spatial average,
\begin{linenomath*}
\begin{equation}
\label{eq:var.general}
\mathrm{var}(\overline{x})=
\frac{1}{N^2}\left(
\sum_{i}^{N}{\mathrm{var}(x_i)} +
2\sum_{i}^{N-1}\sum_{j}^{N}{\mathrm{cov}(x_i,x_j)}
\right).
\end{equation}
\end{linenomath*}
%
In our model, these quantities depend on the distance between sites and on the
correlation structure of the respective field $x$, as we show in the following.

\subsection{Temperature}
\label{app:concept.model.t2m}

For the near-surface temperature field, $x \equiv T$, we assume a spatially
constant variance, $\mathrm{var}(T_0)=\mathrm{var}(T_i)\equiv\sigma_T^2$, and an
isotropic decorrelation following an exponential decay with a decorrelation
length $\tau$; i.e., the covariance between sites is
%
\begin{linenomath*}
\begin{align}
\label{eq:t2m.decorr}
\mathrm{cov}(T_0,T_i)&=\sigma_T^2\exp{\left(-\frac{r_i}{\tau}\right)},\\
\mathrm{cov}(T_i,T_j)&=\sigma_T^2\exp{\left(-\frac{d_{ij}}{\tau}\right)}.
\end{align}
\end{linenomath*}
%
The correlation between the target site temperature and the spatial average of
$N$ temperature time series is then obtained from
%
\begin{linenomath*}
\begin{equation}
\label{eq:t2m.corr}
\mathrm{cor}(T_0,\overline{T})=
\frac{\sum_{i=1}^{N}\exp{\left(-\frac{r_i}{\tau}\right)}}
{\sqrt{N+2\sum_{i=1}^{N-1}
\sum_{j=i+1}^{N}{\exp{\left(-\frac{d_{ij}}{\tau}\right)}}}}.
\end{equation}
\end{linenomath*}

\subsection{Precipitation-Weighted Temperature}
\label{app:concept.model.t2m.pw}

To model the effect of precipitation intermittency, we follow
\citeA{Laepple2018} and assume that precipitation intermittency redistributes
the energy of the temperature time series constantly across frequencies,
i.e., creating temporal white noise without changing the total variance. Then,
the precipitation-weighted temperature time series at location $\mathbf{r}_i$
arises from $T_i$ as
%
\begin{linenomath*}
\begin{equation}
\label{eq:precip.weighting}
T_i^{\mathrm{(pw)}}=
\left(1-\xi\right)^{1/2}T_i + \xi^{1/2} \sigma_T \varepsilon_i(0,1),
\end{equation}
\end{linenomath*}
%
where $\varepsilon_i(0,1)$ are independent and normally distributed random
variables with a mean of zero and a standard deviation of $1$. The parameter
$0\leq\xi\leq1$ determines the fraction of the input temperature time series
which is aliased into white noise.

The covariance between the target site temperature and a precipitation-weighted
temperature time series is then
\begin{linenomath*}
\begin{equation}
\label{eq:t2m.pw.decorr}
\mathrm{cov}(T_0,T_i^{\mathrm{(pw)}})=
(1-\xi)^{1/2}\sigma_T^2\exp{\left(-\frac{r_i}{\tau}\right)},
\end{equation}
\end{linenomath*}
%
which implies that the spatial correlation structure between $T_0$ and the
precipitation-weighted temperature follows the same exponential decay as in
equation~\eqref{eq:t2m.decorr}, only scaled by the factor $(1-\xi)^{1/2}$. The
factor $\xi$ can be estimated from the climate model data by analysing the local
correlation, i.e., at the same grid cell, between the temperature and the
precipitation-weighted temperature.

We further assume that the effect of precipitation intermittency is not
independent between sites but is related to the spatial coherence of the
precipitation fields, for which we assume an exponential decorrelation structure
with a decay length $\tau_{\mathrm{pw}}$. Based on these assumptions, the
spatial covariance between sites of the white noise terms induced by the effect
of precipitation intermittency has the form
%
\begin{linenomath*}
\begin{equation}
\label{eq:noise.cov}
\mathrm{cov}(\varepsilon_i,\varepsilon_j)=
\exp{\left(-\frac{d_{ij}}{\tau_{\mathrm{pw}}}\right)}.
\end{equation}
\end{linenomath*}
%
Then, the correlation between the target site temperature and the spatial
average of $N$ precipitation-weighted temperature time series is governed by the
intermittency factor $\xi$ and by the two spatial length scales $\tau$ and
$\tau_{\mathrm{pw}}$,
%
\begin{linenomath*}
\begin{equation}
\label{eq:t2m.pw.corr}
\mathrm{cor}\left(T_0,\overline{T}^{\mathrm{(pw)}}\right)=
\frac
{\sqrt{1-\xi}\sum_{i=1}^{N}\exp{\left(-\frac{r_i}{\tau}\right)}}
{\sqrt{N + 2\sum_{i=1}^{N-1}\sum_{j=i+1}^{N}
  g(d_{ij}; \tau, \tau_{\mathrm{pw}}, \xi)}},
\end{equation}
\end{linenomath*}
%
with
\begin{linenomath*}
\begin{equation}
\label{eq:exp.fun}
g(d_{ij}; \tau, \tau_{\mathrm{pw}}, \xi):=
(1-\xi)\exp{\left(-\frac{d_{ij}}{\tau}\right)} +
\xi\exp{\left(-\frac{d_{ij}}{\tau_{\mathrm{pw}}}\right)}.
\end{equation}
\end{linenomath*}

\subsection{Precipitation-Weighted Oxygen Isotope Composition}
\label{app:concept.model.oxy.pw}

For the precipitation-weighted oxygen isotope composition field, $x \equiv
\delta^{\mathrm{(pw)}}$, we assume the same effect of precipitation
intermittency as on the temperature field. Furthermore, an analysis of the
climate model data suggests that the oxygen isotope field largely exhibits
an exponential decorrelation structure in space (not shown). Hence, the
correlation between the target site temperature and the spatial average of $N$
$\delta^{\mathrm{(pw)}}$ time series is obtained in a similar manner as for
$T^{\mathrm{(pw)}}$, i.e.,
%
\begin{linenomath*}
\begin{equation}
\label{eq:oxy.pw.corr}
\mathrm{cor}\left(T_0,
  \overline{\delta}^{\mathrm{(pw)}}\right)=
\frac
{\sqrt{1-\xi}\sum_{i=1}^{N}\mathrm{cor}\left(T_0,\delta_i\right)}
{\sqrt{N + 2\sum_{i=1}^{N-1}\sum_{j=i+1}^{N}
  g(d_{ij}; \tau_{\delta}, \tau_{\mathrm{pw}}, \xi)}},
\end{equation}
\end{linenomath*}
%
where $\tau_{\delta}$ is the decorrelation length of the $\delta$ field and the
only difference to equation~\eqref{eq:t2m.pw.corr} is the unknown spatial
correlation structure between the temperature at the target site and the oxygen
isotope field, $\mathrm{cor}\left(T_0,\delta_i\right)$.  Based on our climate
model results (Figure~\ref{fig:avg.cor.structure}), we parameterize this
function with a simple linear decay of the form
%
\begin{linenomath*}
\begin{equation}
\label{eq:t2m.oxy.corr}
\mathrm{cor}\left(T_0,\delta_i\right)=
\begin{cases}
  c_0 - \gamma d, & d \le d_0,\\
  0, & d > d_0,
\end{cases}
\end{equation}
\end{linenomath*}
%
where $\gamma=c_0/d_0$ and $d_0$ is some threshold distance above which
the correlation is zero.

\subsection{Model Parameter Estimation and Model Results}
\label{app:concept.model.estimation}

Overall, our model is governed by three decorrelation lengths ($\tau$,
$\tau_{\delta}$, $\tau_{\mathrm{pw}}$), the intermittency factor $\xi$, and two
parameters describing the temperature--isotope correlation ($c_0$, $d_0$).

We estimate $\tau$ from the climate model data for the DML and Vostok regions
(Figure~\ref{fig:avg.cor.structure}) and find for both regions values of
$\tau=1900$\,km. In the same way we estimate a value of $\tau_{\delta}=1100$\,km
for both regions. The intermittency factor $\xi$ is derived from the local
correlation between temperature and precipitation-weighted temperature
(equation~\ref{eq:t2m.pw.decorr}). We find an average value for the DML region
of $\xi_{\mathrm{DML}}=0.73$, which is close to the average value across all of
Antarctica ($\xi_{\mathrm{Ant.}}=0.71$), while the intermittency is stronger for
the Vostok region ($\xi_{\mathrm{Vostok}}=0.82$). We parameterize the
temperature--isotope correlation in the DML region with $c_0=0.4$ and
$d_0=6000$\,km and in the Vostok region with $c_0=0.5$ and $d_0=2500$\,km
(Figure~\ref{fig:avg.cor.structure}). The only unconstrained parameter is the
decorrelation length of the effect of precipitation intermittency,
$\tau_{\mathrm{pw}}$, since it is unclear by which precipitation variable it is
mainly governed (total annual amount, seasonal amount, or its distribution). An
investigation with reanalysis data yielded scales between $\sim300$ to $500$\,km
for different precipitation variables \cite{Munch2018a}, while our model data
exhibits an average decorrelation length of $\sim600$\,km for the annual
precipitation amount. Here, for the conceptual model we choose a value of
$500$\,km.

We can test our assumption for the effect of intermittency based on using the
estimated values of $\tau$ and $\xi$ to predict the spatial decorrelation
between temperature and precipitation-weighted temperature
(equation~\ref{eq:t2m.pw.decorr}). Indeed, this yields a comparably good fit to
the data as an independent fit (root mean square deviation of $\sim0.03$ between
data and fit in both cases), supporting our assumption that intermittency can be
parameterized by a partial conversion of the time series into white noise.

%fB01
\begin{figure*}[t]%
\centering
\includegraphics[width=17cm]{../plots/main/fig_B01.png}
\caption{%
  Two-dimensional sampling correlation structures with temperature as predicted
  from our conceptual model using the model parameters from the DML
  (\textbf{a}--\textbf{c}) and Vostok (\textbf{d}--\textbf{f}) regions. Shown
  is the mean correlation of all possible single correlations for the average of
  two time series sampled from a pair of concentric rings around the target site
  for the fields of (\textbf{a, d}) $T_{\mathrm{2m}}$, (\textbf{b, e})
  $T_{\mathrm{2m}}^{\mathrm{(pw)}}$ and (\textbf{c, f})
  $\delta^{18}\mathrm{O}^{\mathrm{(pw)}}$. Note that the plots (\textbf{a}) and
  (\textbf{c}) are based on the same parameters and therefore identical.}
\label{fig:conceptual.model}%
\end{figure*}%

Similarly to analysing the climate model data, we now use our conceptual model
to predict the two-dimensional ($N=2$) sampling correlation structures for the
different model fields of $T_{\mathrm{2m}}$, $T_{\mathrm{2m}}^{\mathrm{(pw)}}$
and $\delta^{18}\mathrm{O}^{\mathrm{(pw)}}$ (Eqs.~\ref{eq:t2m.corr},
\ref{eq:t2m.pw.corr} and \ref{eq:oxy.pw.corr}). Since our model space is
continuous, we sample from locations placed \emph{on} concentric rings around
the target site. We either sample the two locations from the same ring or from
two different rings, using ring radii from $0$ to $2000$\,km in increments of
$10$\,km, and calculate the average correlation for a specific ring
combination. To obtain meaningful expectation values, we choose $36$ locations
distributed uniformly across each ring in steps of $10$\textdegree, combine
these locations one by one for each ring combination, and average across the
correlations for each location pair. With the model parameters from the DML and
Vostok regions we obtain the results displayed in
Figure~(\ref{fig:conceptual.model}), which are discussed and compared to the
estimated results from the climate model data in the main text.

\acknowledgments
Enter acknowledgments, including your data availability statement, here.

\bibliography{muench_etal_optimalcores}

\end{document}
