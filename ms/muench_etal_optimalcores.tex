\documentclass[cp, manuscript]{copernicus}

\begin{document}

\title{[working title] Optimal ice-core arrangement}

\Author[1]{Thomas}{M\"{u}nch}
\Author[2]{Martin}{Werner}
\Author[1,3]{Thomas}{Laepple}

\affil[1]{Alfred-Wegener-Institut Helmholtz-Zentrum f{\"u}r Polar- und
Meeresforschung, Research Unit Potsdam, Telegrafenberg A45, 14473 Potsdam,
Germany}
\affil[2]{Alfred-Wegener-Institut Helmholtz-Zentrum f{\"u}r Polar- und
Meeresforschung, Bussestra{\ss}e 24, 27570 Bremerhaven, Germany}
\affil[3]{University of Bremen, MARUM~--~Center for Marine Environmental
  Sciences and Faculty of Geosciences, 28334 Bremen, Germany}

\correspondence{Thomas M\"unch (thomas.muench@awi.de)}
\runningtitle{Optimal ice-core arrangement}
\runningauthor{T. M\"unch et al.}

\received{}
\pubdiscuss{}
\revised{}
\accepted{}
\published{}

\firstpage{1}

\maketitle

\begin{abstract} Many proxies used in climate research share one complicating
property: they are not only driven by the climatic target variable of interest,
e.g. temperature, but also influenced by secondary effects which cause
additional variability, frequently termed \emph{noise}. Noise in individual
proxy records can be reduced generally by averaging them together, but the
effectiveness of this approach depends on the spatial correlation scales of the
involved noise-generating processes. Here, we review this concept in the context
of Antarctic ice cores and apply it in order to optimise local-to-regional-scale
temperature reconstructions from ice-core isotope records. We use data from
climate model simulations to identify an optimal arrangement of ice-core arrays
which maximises the signal-to-noise ratio in the reconstruction with minimum
sampling effort. An intriguing result is that, in order to optimise
temperature reconstructions at a certain target site on the East Antarctic
Plateau using two ice-core records, it is mostly necessary to combine a local
with a more distant core, with distances between the cores of
$\sim500$--$1000$\,km. A similarly large spread between ice cores also remains
optimal for reconstructions that average more than two isotope records. We show
that these findings can be largely explained by the interplay of the two spatial
scales associated with the correlation structure of the temperature field and of
the noise generated by precipitation intermittency. In summary, our study helps
to maximise the usability of existing ice cores as well as to optimally plan
future drilling campaigns. It also deepens our knowledge concerning the typical
correlation scales of the different processes that shape the isotopic
record. Finally, the presented method can be direcly extended to other
palaeoclimate reconstruction problems.
\end{abstract}

% \copyrightstatement{TEXT}

\introduction

The oxygen and hydrogen isotopic composition of firn and ice recovered from
polar ice cores is a key proxy for past near-surface atmospheric temperature
changes \citep{Dansgaard1964,Lorius1969,Masson-Delmotte2008,Sjolte2011}.
Although the physical mechanisms that link local changes in temperature to the
isotopic composition of precipitated snow are generally well understood
\citep{Dansgaard1964,Craig1965,Jouzel1984} and can be modelled with general
circulation models
\citep{Joussaume1984,Werner2011,Werner2016,Sjolte2011,Goursaud2018}, the
quantitative interpretation of ice-core isotope variability in terms of
temperature variability is complicated by second-order processes that
additionally influence the isotopic record, creating noise \citep{Munch2018a}.
Here, an analysis of the typical spatial scales on which the different processes
act offers a way to minimise the variability which is not related to temperature
by identifying an optimal combination of locations to average ice-core records.

The isotopic record that is derived from an ice core is the result of a chain of
processes from (1) atmospheric temperature changes to (2) the isotopic
fractionation during the pathway of the atmospheric moisture, (3) the effect of
variable and intermittent precipitation and finally (4) local depositional and
post-depositional effects; each element of this chain can be associated with a
typical spatial scale.

Atmospheric temperature variations drive the fractionation of the isotopic
composition of the atmospheric moisture along its pathway to the final stage of
precipitation \citep{Dansgaard1964,Jouzel1984}. The spatial coherence
of the temperature-related isotopic signal in precipitation is hence given by
the spatial coherence of the atmospheric temperature field itself. Typical
spatial decorrelation scales for temperature are of the order of
$\gtrsim1000$\,km \citep{Jones1997}, which implies that ice cores distributed on
spatial scales below $\sim 1000$\,km should record a similar, i.e. correlated,
temperature signal. However, the temporal variability of the isotopic
composition in the local atmospheric moisture also depends on the variability of
the atmospheric circulation, since different air masses may exhibit different
source regions and distillation pathways \citep{Schlosser2004,Sodemann2008a,
Birks2009,Kuttel2012}. In addition, the isotopic composition profile across a
deposited layer of snow in an ice core will not one-to-one reflect the temporal
variability of the atmospheric isotope signal due to the intermittent nature of
precipitation \citep{Schleiss2015}, which weights the
initial isotope signal with the amount of precipitation, thereby introducing
bias \citep{Steig1994,Laepple2011a} and adding additional variability to
the isotopic record \citep{Persson2011,Casado2019}. Both processes are directly
linked to the atmospheric dynamics, and their typical scales may hence be
expected to range from mesoscales (i.e. tens of kilometres), driven by
topography and orographic effects, to synoptic scales of hundreds of kilometres
associated with cyclonic activity and the movement of high and low presssure
systems. Finally, in polar conditions the precipitated snow does not directly
settle but is constantly eroded, blown away and redeposited. This depositional
process has been shown to give rise to stratigraphic noise in the isotopic
record \citep{Fisher1985,Munch2016} which exhibits a small-scale decorrelation
of a few metres \citep{Munch2016}.%
\footnote{We note that the final isotopic record is influenced further by
  post-depositional processes within the snow and ice matrix, such as
  densification and diffusion, which are, however, not within the scope of this
  article.}

This hierarchy of process scales~--~from temperature to the atmospheric isotopic
composition, to the precipitation-weighted isotopic composition, and finally to
local depositional effects~--, determines the effectiveness of reducing overall
noise by averaging isotope records, since the reduction of the noise level will
depend on the spatial correlation scale of the different noise sources. For
example, if we only considered temperature and stratigraphic noise, it would be
sufficient to average records spaced just some tens of metres apart, as this
would ensure highly correlated temperature signals but uncorrelated
stratigraphic noise between the records. However, the comparison of
correlation-based signal-to-noise ratios of nearby isotope records
\citep{Munch2016,Munch2017} with those estimated from analysing the records'
temporal variability \citep{Laepple2018} shows that reproducibility on a local
scale does not necessarily imply a climatic, i.e. temperature-driven,
origin. Instead, circulation variability and precipitation intermittency act as
further noise sources which are expected to exhibit larger decorrelation lengths
than the stratigraphic noise \citep{Laepple2018,Munch2018a}. Taking this into
account, we expect some optimal length scale in between the local and the
temperature decorrelation scale which is a trade-off between averaging out
circulation and intermittency effects and ensuring a sufficient coherence of the
recorded temperature signal.

Here, we use data from a climate model equipped with stable isotope diagnostics
for a systematic study of the different process scales and link this to the
optimal spatial arrangement of ice-core locations for maximising the correlation
with temperature at a certain target site. Focussing on target sites on the East
Antarctic Plateau, our results show that the average of a combination of
ice-core isotope records yields a higher correlation with temperature when the
sampled locations are spread across distances of up to $1000$\,km from the
target site, or even more, than when they are located close to the target
site. While these results may seem counterintuitive at first, we can
qualitatively explain their general features with a simple analytical model that
uses the typical spatial correlation structures associated with the temperature
and isotope fields and with the noise generated by precipitation intermittency.
% This result also holds when assessing the range
% of correlations from sampling the large possible set of individual site
% combinations. In general, optimal sampling can increase the correlation with the
% target temperature by x-y\,\% when averaging x-y cores as compared to a local
% reconstruction using only a single record (\#EDIT).

\section{Data and methods}\label{methods}

\subsection{Climate model data}\label{methods:data}

We use data of the past-millennium simulation (800--2000\,CE;
\citealp{Sjolte2018}) of the fully coupled ECHAM5/MPI-OM-wiso atmosphere--ocean
general circulation model equipped with stable isotope diagnostics
\citep{Werner2016}. This simulation is forced by greenhouse gases, volcanic
aerosols, total solar irradiance and changes in land use and Earth's orbital
parameters. The model's atmospheric component ECHAM5-wiso has a T31 spectral
resolution ($3.75\degree\times3.75\degree$) with $19$ vertical levels
\citep{Sjolte2018}. Compared to observations, it generally reproduces the
climatological relation between temperature and precipitation isotopic
composition well, but has, in the used T31 setup, a warm bias and is not
depleted enough in isotopic composition over Antarctica \citep{Werner2011}. This
can be improved with a higher spatial resolution \citep{Werner2011}, which is,
however, not relevant for our study, since we are mainly interested in the
relative variability between sites. The full atmosphere--ocean model was
evaluated against observations for equilibrium simulations under pre-industrial
and Last Glacial Maximum conditions \citep{Werner2016}, and the past-millennium
simulation was used to reconstruct North Atlantic atmospheric circulation in
combination with ice-core isotope data \citep{Sjolte2018}.

Here, we use the ECHAM5/MPI-OM-wiso time series for all model grid cells
on the Antarctic continent ($N_{\mathrm{grid}}=442$) of two-metre air
temperature ($T_{2\mathrm{m}}$), precipitation ($p$) and oxygen isotopic
composition in precipitation (relative abundance of oxygen-18 to oxygen-16
istopes, denoted as $\delta^{18}\mathrm{O}$).

\subsection{Data processing}\label{methods:prc}

The model simulation output has monthly temporal resolution, while ice-core
isotope records are typically analysed on annual or longer resolution achieved
by averaging the isotopic data across the respective amount of snow and
ice. The annual data of isotopic composition will then include a weighting
effect due to the intra-annual variability in the amount of precipitation. To
account for this, we produce two versions of annual data from the monthly model
output: (1) the two-metre temperature and oxygen isotopic composition are simply
averaged to annual resolution without any weighting ($T_{2\mathrm{m}}$ and
$\delta^{18}\mathrm{O}$ in the following), (2) the respective monthly data are
averaged to annual resolution including a weighting by the monthly amount of
precipitation (denoted as precipitation-weighted data
$T_{2\mathrm{m}}^{\mathrm{(pw)}}$ and $\delta^{18}\mathrm{O}^{\mathrm{(pw)}}$ in
the following).

\subsection{Data analyses}\label{methods:main}

\subsubsection{General approach}\label{methods:general}

We investigate the relationship between the various model variables by assessing
the Pearson correlation coefficient (denoted as $r$). To link model data to ice
cores, we mimic ice-core isotope records by associating them with the
$\delta^{18}\mathrm{O}^{\mathrm{(pw)}}$ time series at the model grid cells. We
thus neglect stratigraphic noise and any further depositional as well as
post-depositional effects on the isotopic record, since here we are interested
in the upper limit of the extent to which ice cores can reconstruct the climatic
temperature signal in the atmosphere.

In order to learn about the typical spatial scales that govern the
temperature--isotope relationship, we set up the following general scheme: For a
given model grid cell $\mathbf{r}_0$ of interest (target site in the following),
we define consecutive rings around this site of $250$\,km radial width until a
maximum distance of $2000$\,km (Fig.~\ref{fig:concept}). Then, we determine all
grid cells that fall into each of these rings and either sample these grid cells
directly or sample from the possibilities of combining model grid cells and
rings.

%f01
\begin{figure}[t]%
\centering
\includegraphics[width=6.5cm]{../plots/main/fig_01.pdf}
\caption[Conceptual approach]{%
  Conceptual sketch of the general approach. Around a given Antarctic target
  site (black cross), we define consecutive rings (red lines) of $250$\,km
  radial width and analyse all model grid cells that fall into each of the
  rings.}
\label{fig:concept}%
\end{figure}%

% \subsubsection{Estimation of decorrelation lengths and spatial correlation
%   structures}\label{methods:decor.model}

% We estimate spatial temperature decorrelation lengths by fitting an isotropic
% exponential model to the correlation--distance dependence of the form
% \begin{equation}\label{eq:decor.model}
% c(r) = \exp{\left(-\frac{r}{\tau}\right)},
% \end{equation}
% where $c(r)$ is the fitted correlation at some distance $r$ from the target, and
% $\tau$ is the estimated decorrelation length at which the correlation has
% declined to $1/e$ ($\sim37\,\%$). The input relationship between correlation and
% distance is obtained in two different ways: (1) For a single target site, we
% correlate the $T_{2\mathrm{m}}$ time series at the target site with the
% $T_{2\mathrm{m}}$ time series of every other grid cell on the Antarctic
% continent and record the respective distances between the sites. (2) To reduce
% the uncertainty of these individual correlation estimates, we average across the
% correlations obtained between the $T_{2\mathrm{m}}$ time series at a target site
% and the $T_{2\mathrm{m}}$ time series of grid cells sampled from rings
% (Fig.~\ref{fig:concept}). In a second averaging step, we compute the mean of the
% so obtained correlation--distance relationships for several target sites from a
% given Antarctic region and use that as input to Eq.~\eqref{eq:decor.model}.

% The spatial correlation structure between the target site temperature and other
% model variables is assessed in the same manner as for the average temperature
% decorrelation.

\subsubsection{Picking optimal sites}\label{methods:picking}

To obtain an optimal set of ice cores to reconstruct $T_{2\mathrm{m}}$ at a
given target site, in a first step we randomly pick without replacement $N$ of
the grid cells that lie within a circle of $1000$\,km radius around the target
site and correlate the average $\delta^{18}\mathrm{O}^{\mathrm{(pw)}}$ time
series of these $N$ grid cells with the target site temperature. The optimal set
of cores for each $N$ is then determined from the maximum correlation across all
picking trials. To ensure stable results, we set the maximum number of trials to
$n=5000$ for $N=1$ and to $n=10^5$ for $N\geq3$.

\subsubsection{Optimal sampling structure}\label{methods:opt.sampling}

The optimal locations determined from the picking approach described above will
likely depend on the used model simulation and do not offer insights into the
governing spatial scales. To overcome these issues, we generalise the approach
in a next step by using all climate model variables and conducting a ring
sampling approach of $N$ locations around a given target site as introduced
previously (Fig.~\ref{fig:concept}). This is implemented as a two-step process:
(1) we determine all possible combinations of selecting $N$ rings with
replacement, and then, (2), identify for each ring combination the possibilities
of combining grid cells from each of the rings by selecting an individual grid
cell from each ring. For each of these grid-cell combinations, we average the
time series for the studied model variable and compute the correlation with the
target site temperature. Finally, we report the expected correlation for every
ring combination by averaging across all correlations of the analysed grid-cell
combinations. This gives insights into the average spatial structure of the
correlation with the target site temperature for sampling $N$ locations from the
model field depending on the distances between the locations; we denote this
quantity by \emph{sampling correlation structure} in the following. Note that in
the one-dimensional case ($N=1$), the sampling correlation structure is
identical with what is usually called the spatial correlation structure, i.e.\
the average correlation as a function of distance.

In the second step above, it is computationally feasible to identify all
possible grid-cell combinations until $N=2$; for $N\geq3$, we resort to Monte
Carlo sampling instead. For this, we estimate a sufficient number of Monte Carlo
iterations from comparing the Monte Carlo sampling solution for $N=2$ with its
exact solution. We find the correlation mismatch (root mean square deviation) to
be $\sim10^{-3}$ for $10^4$ iterations. We choose $10^5$ iterations for sampling
$N\geq3$ locations, since this larger number of locations involves a larger
number of possible ring combinations and thus many more possible grid-cell
combinations.

\subsubsection{Selected study regions}\label{methods:regions}

We focus our analyses mainly on two subregions of the East Antarctic Plateau
which both cover existing deep ice-core drilling sites as well as large arrays
of shallower ice and firn cores.

For the first region, the Dronning Maud Land (DML) region in the following, we
choose all model grid cells ($N_{\mathrm{grid}}=26$) within a range of
$\pm17.5\degree$ longitude and $\pm5\degree$ latitude around the EPICA Dronning
Maud Land site (EDML; $-75\degree$\,S, $0\degree$\,E). This region covers the
locations of the deep EDML ice core \citep{EPICAcommunitymembers2006} and of
$>50$ firn and shallow ice cores \citep{Altnau2015}. For the second region,
referred to as the Vostok region, we choose an identical latitudinal and
longitudinal coverage ($N_{\mathrm{grid}}=30$) with respect to the Vostok
station ($-78.47\degree$\,S, $106.83\degree$\,E), covering the locations of the
deep Vostok and Dome C ice cores as well as of several shallower cores
\citep{Stenni2017}, and including the new deep drilling site (``Little Dome C'')
for retrieving an ice core reaching back more than one million years.

\section{Results}\label{results}

\subsection{Temperature decorrelation and temperature--isotope relationship}
\label{results:t2m-iso}

We first assess the extent to which a local ice-core record, i.e. the annual
isotope time series of a single grid cell in the model simulation, is
representative for the local and regional variability of the near-surface
atmospheric temperature.

The temperature field over Antarctica in the climate model exhibits generally
large-scale coherent variations (Fig.~\ref{fig:correlation.maps}a), but with a
clear two-part structure roughly divided by the range of the Transantarctic
Mountains: For most regions of the East Antarctic Plateau, the temperature field
shows typical decorrelation lengths of $\sim1500$ to $2500$\,km, while the
decorrelation lengths are significantly lower with values $\lesssim1000$\,km for
larger parts of the West Antarctic Ice Sheet and for the Antarctic Peninsula.
Still, for perfect ice cores, i.e.\ assuming an ideal temperature proxy which is
only governed by local temperature variations, a single ice core would capture
the temperature variability in both regions across hundreds of kilometres.
% In order to reduce the uncertainty associated with the
% individual correlation estimates from single grid cells we analyse the expected
% correlation as a function of distance averaged across grid cells and across a
% region of target sites (Fig.~\ref{fig:avg.cor.structure}). Here, as an example
% for an existent ice-core drilling region, we choose as target sites all grid
% cells within a range of $\pm17.5\degree$ longitude and $\pm5\degree$ from the
% EPICA Dronning Maud Land (EDML) site ($-75\degree$\,S, $0\degree$\,E), referred
% to as the Dronning Maud Land (DML) region in the following. We find the
% temperature decorrelation in this region to clearly follow an exponential decay
% with a length scale of $\sim1900$\,km. 

%f02
\begin{figure*}[t]%
\centering
\includegraphics[width=16cm]{../plots/main/fig_02.png}
\caption{%
  Temperature decorrelation lengths and temperature--isotope
  relationship. (\textbf{a}) The temperature decorrelation lengths ($\tau$, in
  km) for each Antarctic model grid cell estimated by fitting an exponential
  model to the correlation--distance relationship (cf. Eq.~\ref{eq:t2m.decorr})
  obtained from correlating the local annual near-surface $T_{2\mathrm{m}}$
  time series with the respective temperature time series from all other grid
  cells. Note that only continental grid cells are used for the fit.
  (\textbf{b}) The local correlation between the annual near-surface temperature
  ($T_{2\mathrm{m}}$) and precipitation-weighted oxygen isotope composition
  ($\delta^{18}\mathrm{O}^{\mathrm{(pw)}}$) time series for each Antarctic model
  grid cell.}
\label{fig:correlation.maps}%
\end{figure*}%

However, as simulated by the climate model, actual single Antarctic ice-core
isotope records only explain a low portion of the local temperature variations
and thus of the regional temperature field. This is seen from analysing the
annual precipitation-weighted field of modelled
$\delta^{18}\mathrm{O}^{\mathrm{(pw)}}$, which most closely mimicks a real
ice-core record (Fig.~\ref{fig:correlation.maps}b). We find the local
correlation between the $T_{2\mathrm{m}}$ and
$\delta^{18}\mathrm{O}^{\mathrm{(pw)}}$ time series for each model grid cell to
be generally low, ranging across all analysed grid cells from $<0.1$ up to
$\sim0.53$ with $\sim75\,\%$ of the correlations $\leq0.4$ and a mean
correlation of $0.36$.

In the following, we will assess the extent to which the correlation with
temperature can be increased and how this is related to the spatial scales
studied.
% In the second step, we determine the average spatial
% strucure of the temperature--isotope correlation as a function of distance
% (Fig.~\ref{fig:avg.cor.structure}), assessed for the DML region in the same
% manner as the average temperature decorrelation structure presented above. In
% contrast to the exponential temperature decorrelation, the
% $\delta^{18}\mathrm{O}$ field in the DML region exhibits a much lower but also
% more stable structure of correlation with the target site temperature. Starting
% from a local value of $\sim0.4$, the correlations decrease only slightly with
% increasing distance up to $\sim1300$\,km, followed by a little steeper decrease
% for larger distances and constant levels of $\lesssim0.2$ for distances above
% $\sim1700$\,km. Precipitation weighting, as seen from analysing the
% $\delta^{18}\mathrm{O}^{\mathrm{(pw)}}$ field, results in overall even lower
% correlation values, but it does not affect much the correlation structure
% itself.

\subsection{Picking optimal ice-core sites for temperature reconstructions}
\label{results:picking}

The above analysis shows that isotope records of single ice cores overall only
capture a low portion of the local interannual temperature variability,
suggesting that additional processes influence the isotopic signal which lower
the correlation with the local temperature record. Interpreting these processes
as noise raises the question of whether the correlation with temperature can be
improved by averaging isotope records across space. Here, we first assume an
ideal world, in which the climate model data are a perfect surrogate for the
true climate and proxy variations at each site. By doing so, we can set up the
simple experiment of randomly picking and averaging
$\delta^{18}\mathrm{O}^{\mathrm{(pw)}}$ grid cells to determine that spatial
array of $N$ ice cores which optimises the temperature correlation with a target
site.

For our specific model simulation and the EDML drilling site as a target, we
obtain the interesting result that the optimal location for a single core is not
the local grid cell -- as one might expect -- but a site $\sim960$\,km away from
the target towards the southeast (Fig.~\ref{fig:picking}a). Choosing this site
increases the correlation with the target temperature from the local value of
$0.26$ to a value of $0.43$. The maximum correlation with the target temperature
for averaging a set of three and five cores (Fig.~\ref{fig:picking}b--c) is
obtained from locations that in both cases scatter at significant distances
around the target, but with only slightly higher correlations than for $N=1$
($r=0.46$ for $N=3$, $r=0.47$ for $N=5$). We obtain similar results for
choosing the Vostok drilling site as a target (Fig.~\ref{fig:picking}d--f). The
optimal single core for this model simulation would be at a location
$\sim300$\,km north of Vostok ($r=0.45$, compared to the local correlation of
$0.34$). As for EDML, the optimal locations for averaging three and five cores
all lie around the target without including it, but here, the optimal set of
three cores ($r=0.56$) yields a significant increase in correlation compared to
the optimal single core, while the optimal set of five cores likewise shows no
further increase ($r=0.57$).

%f03
\begin{figure*}[t]%
\centering
\includegraphics[width=16cm]{../plots/main/fig_03.pdf}
\caption[Picking optimal sites]{%
  Picking ice core locations that optimally reconstruct interannual temperatures
  at the EDML and Vostok drilling sites. The maps show the correlation in the
  model data between the annual temperature time series at the target sites
  (black crosses) EDML (\textbf{a}--\textbf{c}) and Vostok
  (\textbf{d}--\textbf{f}) with the fields of precipitation-weighted oxygen
  isotope composition. Filled black circles denote those grid cells that
  maximise the correlation with the target site temperature for choosing either
  a single grid cell ($N=1$; \textbf{a}, \textbf{d}) or for averaging across
  $N=3$ (\textbf{b}, \textbf{e}) or $N=5$ (\textbf{c}, \textbf{f}) grid cells.}
\label{fig:picking}%
\end{figure*}%

We generalise these findings by analysing the core location of the optimal
correlation for all Antarctic model grid cells as target sites. Indeed, the
majority ($\sim65$\,\%) of optimal locations lie at distances from the
respective target site of between $400$--$600$ and $800$--$1000$\,km.

\subsection{Optimal ice-core sampling structures}
\label{results:optim-spacing}

The approach of picking optimal ice-core locations yields straightforward and
instructive results, but it also has several shortcomings. Firstly, the results
will likely depend on the used model simulation and are therefore difficult to
generalise. Secondly, the dependency of the results on the chosen target site
(especially for averaging $N\ge2$ cores) complicates the inference of typical
spatial scales that govern the correlation with temperature. Thus, as a next
step, we adapt our approach in order to overcome these issues and to learn about
the general optimal spatial arrangement of ice cores maximising the correlation
with temperature. For this, we compute the mean of correlation results obtained
between a target site temperature and individual grid cells in order to reduce
local variability in the model data. We perform this averaging step across
combinations of $250$\,km wide concentric rings around a target site
(Fig.~\ref{fig:concept} and Sect.~\ref{methods:opt.sampling}) to prevent the
results from depending on any specified direction. Additionally, if applicable,
we average so obtained results across the target sites within our defined DML
and Vostok regions (Sect.~\ref{methods:regions}) to obtain regional
estimates. Finally, we analyse each of the model variables to highlight the
differences between the individual fields.

%f04
\begin{figure*}[t]%
\centering
\includegraphics[width=12cm]{../plots/main/fig_04.pdf}
\caption{%
  Sampling correlation structures with temperature for the DML and Vostok
  regions in the one-dimensional case of sampling single locations. Shown is the
  average correlation as a function of distance between the interannual
  near-surface temperature ($T_{2\mathrm{m}}$) at a target site and the spatial
  fields of $T_{2\mathrm{m}}$ (black), oxygen isotope composition
  ($\delta^{18}\mathrm{O}$, green) and precipitation-weighted oxygen isotope
  composition ($\delta^{18}\mathrm{O}^{\mathrm{(pw)}}$, blue). Averaging is
  performed in two steps: first, correlations are averaged across grid cells
  falling in $250$\,km wide consecutive rings around a given target site, and
  secondly, these results are averaged across all respective target sites in the
  DML (\textbf{a}) and Vostok (\textbf{b}) region (see Methods). Dashed lines
  indicate an exponential fit to the $T_{2\mathrm{m}}$ data.}
\label{fig:avg.cor.structure}%
\end{figure*}%

For sampling a single location ($N=1$) from the $T_{\mathrm{2m}}$ field in the
DML and Vostok region, we find that the sampling correlation structure
(Fig.~\ref{fig:avg.cor.structure}) can be described by an exponential decay with
a length scale of $\sim1900$\,km in both cases, consistent with the results on
the local scale (Fig.~\ref{fig:correlation.maps}a). These results also imply
that the maximum expected correlation with the target site temperature
is obtained for sampling from the innermost ring only.

The $\delta^{18}\mathrm{O}$ field in the DML region exhibits a lower average
correlation with the target site temperature as a function of distance
(Fig.~\ref{fig:avg.cor.structure}a). From a local value of $r\sim0.4$, the
correlations decrease only slightly with increasing distance up to
$\sim1300$\,km, resulting in the highest correlation to be found also for
sampling from the innermost ring, but with nearly equally high correlations
obtained for rings till a distance of $\lesssim1000$\,km. This slight decrease
is then followed by a little steeper decrease for larger distances and constant
levels of $r\lesssim0.2$ for distances above $\sim1700$\,km. For the Vostok
region (Fig.~\ref{fig:avg.cor.structure}b), the sampling correlation structure
for $\delta^{18}\mathrm{O}$ exhibits a nearly linear decrease from an initial
value of $r\gtrsim0.5$ to $r\sim0.1$ in the final ring ($>2000$\,km).
Precipitation weighting induces overall even lower correlation values in both
regions, as seen from analysing the $\delta^{18}\mathrm{O}^{\mathrm{(pw)}}$
fields, but it does not affect much the sampling correlation structure itself.

%f05
\begin{figure*}[t]%
\centering
\includegraphics[width=17cm]{../plots/main/fig_05.png}
\caption{%
  Sampling correlation structures with temperature in the two-dimensional case
  of sampling two locations in the DML region. Shown is the mean correlation of
  all possible single correlations from averaging two grid cells of (\textbf{a})
  $T_{\mathrm{2m}}$, (\textbf{b}) $T_{\mathrm{2m}}^{\mathrm{(pw)}}$ and
  (\textbf{c}) $\delta^{18}\mathrm{O}^{\mathrm{(pw)}}$ time series sampled from
  the same or from two different rings, averaged over all target sites in the
  given region. The axes display the distance from the target, where the $x$
  ($y$) axis stands for the first (second) sampled ring and tickmarks indicate
  the radius of the midpoints of the rings.}
\label{fig:two-core-correlation}%
\end{figure*}%

The difference in sampling correlation structure between the fields of
$T_{\mathrm{2m}}$ and $\delta^{18}\mathrm{O}^{\mathrm{(pw)}}$ is even more
pronounced in the two-dimensional case of averaging $N=2$ locations. Here, we
can assess either the possibility of sampling locations from the same ring or
from two different ones. As one would expect, the maximum expected correlation
for $T_{\mathrm{2m}}$ is still found when sampling both locations from the
innermost ring, as shown for the DML region
(Fig.~\ref{fig:two-core-correlation}a). However, for
$\delta^{18}\mathrm{O}^{\mathrm{(pw)}}$ the optimal arrangement of two locations
is obtained for sampling one location from the innermost ring but the second
one from the fifth ring, i.e.\ between $\sim1000$ and $1250$\,km from the target
site (Fig.~\ref{fig:two-core-correlation}c). Part of this structure is related
to the effect of precipitation intermittency, which can be seen from the
sampling correlation structure for the $T_{\mathrm{2m}}^{\mathrm{(pw)}}$ field
(Fig.~\ref{fig:two-core-correlation}b). Here, in contrast to $T_{\mathrm{2m}}$,
the correlation is about as high when we combine the innermost ring and one ring
further away, as when we sample both locations from the innermost ring.

The same effect of precipitation intermittency is also evident in the equivalent
results for the Vostok study region (Appendix~\ref{app:vostok.n2}:
Fig.~\ref{fig:two-core-correlation-vostok}), with a similar difference in
sampling correlation structure between $T_{\mathrm{2m}}$ and
$T_{\mathrm{2m}}^{\mathrm{(pw)}}$ as for the DML region, and a similar structure
of $T_{\mathrm{2m}}^{\mathrm{(pw)}}$ and
$\delta^{18}\mathrm{O}^{\mathrm{(pw)}}$ for distances $\lesssim1000$\,km.
However, the results for the $\delta^{18}\mathrm{O}^{\mathrm{(pw)}}$ field
(Fig.~\ref{fig:two-core-correlation-vostok}c) do not display such a pronounced
maximum when sampling one location from the innermost ring and the second one
from a ring further away as it is observed for the DML region, which points at
an influence of the regional differences in the spatial correlation structure of
the $\delta^{18}\mathrm{O}$ field (Fig.~\ref{fig:avg.cor.structure}).

Despite these regional differences, the general feature for sampling
$\delta^{18}\mathrm{O}^{\mathrm{(pw)}}$ is robust throughout Antarctica. When we
fix the first location to the innermost ring and assess for all available
Antarctic target sites the optimal ring of the second location, for which the
expected correlation with the target site temperature is maximal, we find that
in $\sim77\,\%$ of all cases the optimal configuration for the second location
is at least the second ring ($250$--$500$\,km), and in $\sim61\,\%$ of the cases
it is within the second to fourth ring ($250$--$1000$\,km).

%f06
\begin{figure*}[t]%
\centering
\includegraphics[width=16cm]{../plots/main/fig_06.pdf}
\caption{%
  The optimal arrangement for averaging three or five
  $\delta^{18}\mathrm{O}^{\mathrm{(pw)}}$ ice cores to reconstruct the target
  site temperature at the EDML (\textbf{a}, \textbf{c}) and Vostok (\textbf{b},
  \textbf{d}) drilling sites. Displayed are subsets of the sampling correlation
  structures for $N=3$ and $5$, showing the optimal five of all possible
  combinations of rings, i.e. those which exhibit the highest mean correlation
  across $10^5$ random trials of averaging $N=3$ (\textbf{c}, \textbf{d}) or
  $N=5$ (\textbf{a}, \textbf{b}) grid cells from these rings.}
\label{fig:binning}%
\end{figure*}%

Furthermore, we obtain similar results also for averaging $N=3$ and $5$
locations of the $\delta^{18}\mathrm{O}^{\mathrm{(pw)}}$ field to reconstruct
the target site temperature (Fig.~\ref{fig:binning}). For EDML as a target site,
the optimal sampling configuration is such that one location lies in the
innermost ring while the others are distributed at distances between $\sim500$
and $1500$\,km from the target. For reconstructing the Vostok target site
temperature, the optimal locations are mostly distributed across the second to
third ($250$--$750$\,km) ring. In summary, the averaging of locations clearly
increases the expected correlation with the target site temperature, if it
follows an optimal combination of rings, as compared to sampling only locally
(Fig.~\ref{fig:cor.increase.risk}a). The increase becomes larger by averaging
more locations, with a gain in explained variance of $27$--$38\,\%$
(Vostok--EDML) for $N=2$ and $82$--$98\,\%$ for $N=10$.

Finally, we note that these results are the expectation value from averaging
across many possibilities of combining individual locations. In reality, only a
single combination of locations can usually be realised when drilling new or
analysing existing ice cores. Therefore, we assess the risk of an ``adverse
optimal sampling'', which yields a correlation that is worse in this specific
sampling realisation than the local value. For this, we compare the distribution
of all individual correlations from sampling the optimal ring combination with
the value obtained from sampling only the local sites which lie in the innermost
ring (Fig.~\ref{fig:cor.increase.risk}b). We find the risk of adverse optimal
sampling to be overall low, since more than $92\,\%$ of all individual
correlation values in the example of $N=3$ are actually larger than the
respective local correlation.

%f07
\begin{figure*}[t]%
\centering
\includegraphics[width=14cm]{../plots/main/fig_07.pdf}
\caption{%
  Correlation increase and risk of adverse sampling. (\textbf{a}) The increase
  in expected correlation with the target temperature at the EDML (red) and
  Vostok (blue) sites depending on the number of locations for averaging
  $\delta^{18}\mathrm{O}^{\mathrm{(pw)}}$ time series. Sampling is performed
  either from the innermost ring only (dashed lines), or from all possible
  individual combinations of locations for the respective optimal ring
  combination determined for each $N$ (solid lines). (\textbf{b}) Histogram of
  all possible individual correlations for sampling from the optimal ring
  combination for averaging $N=3$ locations compared to the correlation
  (vertical lines) for sampling from the innermost ring only, displayed for the
  EDML (red) and Vostok (blue) target sites.}
\label{fig:cor.increase.risk}%
\end{figure*}%

\section{Discussion}\label{discussion}

Oxygen isotope records derived from ice cores are commonly interpreted to
reflect local temperature changes at the ice-core drilling site. Here, in a
systematic study of the interannual correlation in a climate model between
precipitation-weighted oxygen isotope composition and near-surface atmospheric
temperature, we show that while there is local isotope--temperature correlation
(Fig.~\ref{fig:correlation.maps}b), this correlation is considerably increased
by averaging isotope records across space following a distinct spatial pattern
which mostly combines the local target site with locations located between a few
hundred to up to $\sim1000$\,km from this site
(Figs.~\ref{fig:two-core-correlation}c, \ref{fig:binning} and
\ref{fig:two-core-correlation-vostok}c). In the following, we develop a
qualitative understanding of these results from a conceptual model that predicts
the sampling correlation structure from the processes that shape the isotopic
composition time series, before we discuss the relevance of our results for
actual ice-cores studies.

\subsection{Conceptual model of the optimal sampling structure}
\label{discussion:concept.model}

For a conceptual model of the sampling correlation structure, we focus on three
processes that influence the oxygen isotope records in ice cores: (i)
temperature variations, (ii) precipitation intermittency, and (iii) the
temperature--isotope relationship, and we statistically model the associated
fields of $T_{\mathrm{2m}}$, $T_{2\mathrm{m}}^{\mathrm{(pw)}}$ and
$\delta^{18}\mathrm{O}^{\mathrm{(pw)}}$ separately in order to understand the
influence of each process (see Appendix~\ref{app:concept.model} for details).
For comparable results, we assess the expected sampling correlation structure
with the target site temperature in the two-dimensional case of averaging two
locations in the same manner as for analysing the climate model data.

To model the atmospheric temperature field, we assume an isotropic exponential
decay of the correlation in space with a constant decorrelation length
(Appendix~\ref{app:concept.model.t2m}). Such an exponential temperature
decorrelation is a commonly observed feature \citep{Jones1997} (\#MORE refs?)
and also confirmed by our climate model data (Fig.~\ref{fig:correlation.maps}a
and Fig.~\ref{fig:avg.cor.structure}). Given this relationship, we obtain a good
agreement for the sampling correlation structure for $N=2$ between the
conceptual model and the climate model data, both regarding absolute correlation
values as well as spatial pattern (Fig.~\ref{fig:conceptual.model}a). We
emphasise that the maximum correlation with the target site temperature
naturally occurs, in case of an isotropic correlation decay, when the averaged
two (or $N$) locations are close to the target site, as any location which is
further away will result in a smaller common temperature signal between the
locations.

To elucidate the role of precipitation intermittency, we follow the simplest
assumption that this process can be described by partly aliasing the original
temperature signal into temporal white noise \citep{Laepple2018,Casado2019}. We
further assume that this noise is not independent between sites but follows the
spatial scale of the occurrence of precipitation events, which we describe by an
exponential decorrelation in space with a second length scale
(Appendix~\ref{app:concept.model.t2m.pw}). This intermittency length scale is
related to the atmospheric processes that deliver the precipitation,
e.g. synoptic systems, and is hence assumed to be smaller than the length scale
of the temperature anomalies. The introduction of such a second length scale
into our conceptual model can generally explain the optimal sampling structure
we obtain from the climate model data. Qualitatively, close-by locations exhibit
a strong correlation of temperature but also of the noise from precipitation
intermittency, which thus cannot be reduced by averaging those locations,
yielding an overall low signal-to-noise ratio. However, with increasing distance
between the locations, the intermittency noise decorrelates faster than the
temperature field due to the different decorrelation scales, which results in an
optimal distance of maximum signal-to-noise ratio. This is also reflected in our
conceptual model (Fig.~\ref{fig:conceptual.model}b). For fixing one location to
the target site, the correlation with the target site temperature first
increases with increasing distance of the second location and maximises at an
optimal distance, before it decays with a further increase in distance. In the
climate model data, we observe a similar feature for the precipitation-weighted
temperature (Figs.~\ref{fig:two-core-correlation} and
\ref{fig:two-core-correlation-vostok}), though not as clear as in the conceptual
model. This mismatch could be related to the assumed isotropy in the conceptual
model and the according azimuthal averaging done in the climate model data
analysis, which potentially smears out the intermittency effect in the climate
model data due to slight differences in the decorrelation lengths between the
different horizontal directions.
% Also the range of correlation values matches reasonable well between the
% climate model data and the conceptual model; the latter, however, does not
% reproduce the detailed spatial structure and also fails to explain the clear
% differences in correlation structure between the DML and Vostok regions. This
% suggests a too simplistic implementation of precipitation intermittency in our
% conceptual model, which does not capture its detailed effects and regional
% differences.

In order to incorporate the field of $\delta^{18}\mathrm{O}^{\mathrm{(pw)}}$
into the conceptual model, we need to account for the spatial
temperature--isotope relationship. For this, we parameterise the spatial
dependence of the correlation between temperature and the oxygen isotope
composition with a simple isotropic linear model based on the climate model data
results (Fig.~\ref{fig:avg.cor.structure} and
Appendix~\ref{app:concept.model.oxy.pw}). In addition, it is straightforward to
assume the same effect of precipitation intermittency as for the temperature
field. With these simple assumptions, we obtain a good qualitative agreement for
the DML region between the conceptual model and the climate model data results
(cf. Figs.~\ref{fig:conceptual.model}c and \ref{fig:two-core-correlation}c). In
addition, when we change the parameterised isotope--temperature relationship
such that it more resembles the Vostok region data
(Fig.~\ref{fig:avg.cor.structure}b), also the sampling correlation structure in
the conceptual model (not shown) is more similar to the observed one
(Fig.~\ref{fig:two-core-correlation-vostok}c). However, in general the
conceptual model here fails to reproduce the actual range of correlations by
showing much lower values than expected.

In summary, our conceptual model provides a quantitative understanding of the
correlation between temperature in the climate model data, and~--~at least~--~a
qualitative understanding of the processes that affect the correlation with the
$\delta^{18}\mathrm{O}^{\mathrm{(pw)}}$ field, i.e.\ precipitation intermittency
and the spatial temperature--isotope relationship. The deficiencies of the
conceptual model can probably be attributed to its simplicity. We only assume
spatially constant and isotropic length scales for the governing processes,
neglecting local and direction-related differences in, e.g., temperature
decorrelation lengths (cf.~Fig.~\ref{fig:correlation.maps}a) or the spatial
extent of the coherence of precipitation intermittency, which may be not
constant but different for different types of precipitation, e.g.\ synoptic
versus clear-sky precipitation, and may exhibit directional dependencies
related to topography. Furthermore, we assume constant variance of all
time series, which ignores potential weighting effects on the correlation for
the spatial average of locations due to different variability between them.

\subsection{Relevance for ice-core studies}
\label{discussion:relevance}

The optimal sampling correlation structures, which we obtained from analysing
the climate model data and substantiated with our conceptual model, directly
give a guide where to drill or analyse $N=1, 2, 3$ or more ice cores in order to
optimally reconstruct the atmospheric temperature signal for a certain target
site or region.

If it is only possible to drill or analyse a single ice core, our results
show that it is always best to sample locally, i.e.\ to place this core
close to the target site of interest. This is also common practice, given the
usual interpretation of ice-core isotope records as a proxy for local
temperatures. However, due to the effect of precipitation intermittency, it is
already for the option of drilling two ice cores no longer optimal to place both
these cores locally, but instead to drill one core at the target site and one
further away by at least $500$\,km. In the case of drilling or analysing three
or more ice cores, we expect the optimal spatial configuration to be more
dependent on the study region. But our results indicate that in general it is
still likely better to place one core locally and distribute the others across
several hundreds of kilometres.

We note that these inferences are based on data from a single climate model
simulation together with a simple statistical conceptual model, which should be
tested against observations. In order to do so we need to create an isotope
record that is in first order only governed by temperature variations and
precipitation intermittency by removing the impact of local stratigraphic noise
from the actual measured records (assuming that any further processes in the
pre-depositional to depositional phase only contribute negligibly to the local
isotopic variations). For this it has proven successful to conduct trench
sampling campaigns (see \citealp{Munch2016,Munch2017} for the EDML site). One
test of our optimal sampling structures could then be to combine one trench
record, e.g.\ one from EDML, with another trench sampled at the optimal distance
based on our results for $N=2$, and correlate the average of these two trench
records with the instrumental temperature data set available for EDML. Based on
our results here we would expect a higher correlation in this case than compared
to using only one local trench record from EDML. We note that such an approach
would be limited by the small amount of available instrumental data
($\sim20$\,years for EDML) and by the inevitable dating uncertainties between
the the two trench records.

% \noindent
% TBD.

% \noindent Some first thoughts, feel free to add more:
% \begin{itemize}
% \item discuss usefulness of large clusters of relatively nearby cores, e.g. as
%   existent in DML, but also consider stratigraphic noise as additional important
%   factor. In light of our results, then largely spaced clusters of local cores
%   would be more useful.
% \item Weighting of ice-core time series to account for an optimal spacing.
% \end{itemize}

% \codedataavailability{TEXT} %% use this section when having data sets and software code available

\conclusions

In this study we assessed the question of which spatial sampling configuration
for ice cores optimally reconstructs the annual near-surface temperatures at a
certain site. This problem was motivated by the expectation that the major
processes influencing the isotopic record of ice cores operate on different
spatial scales.

Indeed, by analysing the temperature and isotope data of an isotope-enabled
atmosphere--ocean climate model simulating the climatic history over the last
millennium we showed that while in the optimal setup a single ice core should be
placed on average close to the target site of interest, already the second core
should be put far ($>500$\,km) from the first core. At first sight, this might
be an unexpected result, but we could show by comparing our climate model
results with a simple conceptual model that it can be straightforwardly
explained by the interplay of two different correlation lengths in space: one
for the temperature anomalies and one parameterising the spatial coherence of
the effect of precipitation intermittency.

Our study hence explicitly helps improving the setting up of drilling or
analysis campaigns for spatial networks of ice-core isotope records. In
addition, it provides in general the means to analyse an optimal configuration
of measurement locations for proxies, which are influenced by processes that
exhibit different spatial correlation scales.

\appendix

\section{Two-dimensional sampling correlation structure for the Vostok region}
\label{app:vostok.n2}

In order to streamline the main text, we provide here the results of the
average two-dimensional sampling correlation structures ($N=2$) for the Vostok
study region (Fig.~\ref{fig:two-core-correlation-vostok}).

%fA01
\begin{figure*}[t]%
\centering
\includegraphics[width=17cm]{../plots/main/fig_A01.png}
\caption{%
  Sampling correlation structures with temperature in the two-dimensional case
  of sampling two locations in the Vostok region. Shown is the mean correlation
  of all possible single correlations from averaging two grid cells of
  (\textbf{a}) $T_{\mathrm{2m}}$, (\textbf{b}) $T_{\mathrm{2m}}^{\mathrm{(pw)}}$
  and (\textbf{c}) $\delta^{18}\mathrm{O}^{\mathrm{(pw)}}$ time series sampled
  from the same or from two different rings, averaged over all target sites in
  the given region. The axes display the distance from the target, where the $x$
  ($y$) axis stands for the first (second) sampled ring and tickmarks
  indicate the radius of the midpoints of the rings. Note that for
  $\delta^{18}\mathrm{O}^{\mathrm{(pw)}}$ the -- albeit marginal -- correlation
  maximum is located for combining the innermost ring with the ring between
  $500$--$750$\,km.}
\label{fig:two-core-correlation-vostok}%
\end{figure*}%

\section{Conceptual model of sampling correlation structures}
\label{app:concept.model}

\subsection{General model}
\label{app:concept.model.general}

We set up a conceptual model for the correlation between a target temperature
time series and the spatial average of a set of locations sampled from a
climatic field (sampling correlation structure). Our model builds upon assuming
simple isotropic and exponential decorrelation structures for the involved
climatic fields and on previous work which suggests that precipitation
intermittency can be described by partly aliasing the original temperature
signal into white noise.

For the model, we consider a temperature time series $T_0$ at some target site
$\mathbf{r}_0$ and a field $x$ of a given climate variable. From this field, we
select $N$ time series $x_i$ at the locations $\mathbf{r}_i$, $i=1,\dotsc,N$,
and denote the spatial average of these time series by
$\overline{x}=\frac{1}{N}\sum_{i=1}^{N}{x_i}$. The distances of the $N$
locations from the target site and the distances between the locations are given
by $r_i=|\mathbf{r}_i-\mathbf{r}_0|$ and by
$d_{ij}=|{\mathbf{r}_i-\mathbf{r}_j}|$, respectively. The correlation between
$T_0$ and $\overline{x}$ follows from
%
\begin{equation}
\label{eq:corr.general}
\mathrm{cor}(T_0,\overline{x})=\frac
{\mathrm{cov}(T_0,\overline{x})}
{\sqrt{\mathrm{var}(T_0)\mathrm{var}(\overline{x})}},
\end{equation}
and it is governed by the covariance between the temperature at the target site
and the climate field at the sampling locations $\mathbf{r}_i$,
%
\begin{equation}
\label{eq:cov.general}
\mathrm{cov}(T_0,\overline{x})=
\frac{1}{N}\sum_{i}^{N}{\mathrm{cov}(T_0,x_i)},
\end{equation}
%
and by the covariance between the sampling locations through the variance of
their spatial average,
\begin{equation}
\label{eq:var.general}
\mathrm{var}(\overline{x})=
\frac{1}{N^2}\left(
\sum_{i}^{N}{\mathrm{var}(x_i)} +
2\sum_{i}^{N-1}\sum_{j}^{N}{\mathrm{cov}(x_i,x_j)}
\right).
\end{equation}
%
In our model, as we will show in the following, these quantities depend on the
distances between sites and on the correlation structure of the respective field
$x$.

\subsection{Temperature}
\label{app:concept.model.t2m}

For the near-surface temperature field, $x \equiv T$, we assume a spatially
constant variance, $\mathrm{var}(T_0)=\mathrm{var}(T_i)\equiv\sigma_T^2$, and an
isotropic decorrelation following an exponential decay with a decorrelation
length $\tau$; i.e. the covariance between sites is given by
%
\begin{align}
\label{eq:t2m.decorr}
\mathrm{cov}(T_0,T_i)&=\sigma_T^2\exp{\left(-\frac{r_i}{\tau}\right)},\\
\mathrm{cov}(T_i,T_j)&=\sigma_T^2\exp{\left(-\frac{d_{ij}}{\tau}\right)}.
\end{align}
%
For the correlation between the target site temperature and the spatial average
of $N$ temperature time series then follows
%
\begin{equation}
\label{eq:t2m.corr}
\mathrm{cor}(T_0,\overline{T})=
\frac{\sum_{i=1}^{N}\exp{\left(-\frac{r_i}{\tau}\right)}}
{\sqrt{N+2\sum_{i=1}^{N-1}
\sum_{j=i+1}^{N}{\exp{\left(-\frac{d_{ij}}{\tau}\right)}}}}.
\end{equation}

\subsection{Precipitation-weighted temperature}
\label{app:concept.model.t2m.pw}

To model the effect of precipitation intermittency, we follow
\citet{Laepple2018} and assume that precipitation intermittency redistributes
the energy of the temperature time series constantly across frequencies,
i.e. creating temporal white noise without changing the total variance. Then,
the precipitation-weighted temperature time series at the location
$\mathbf{r}_i$ arises from $T_i$ as
%
\begin{equation}
\label{eq:precip.weighting}
T_i^{\mathrm{(pw)}}=
\left(1-\xi\right)^{1/2}T_i + \xi^{1/2} \sigma_T \varepsilon_i(0,1),
\end{equation}
%
where $\varepsilon_i(0,1)$ are independent normally distributed random variables
with zero mean and standard deviation $1$, and the parameter $0\leq\xi\leq1$
determines the fraction of the input temperature time series which is aliased
into white noise.

The covariance between the target site temperature and a precipitation-weighted
temperature time series is then
\begin{equation}
\label{eq:t2m.pw.decorr}
\mathrm{cov}(T_0,T_i^{\mathrm{(pw)}})=
(1-\xi)^{1/2}\sigma_T^2\exp{\left(-\frac{r_i}{\tau}\right)},
\end{equation}
%
which implies that the spatial correlation structure between $T_0$ and the
precipitation-weighted temperature follows the same exponential decay as
Eq.~\eqref{eq:t2m.decorr}, only scaled by the factor $(1-\xi)^{1/2}$. The factor
$\xi$ can be estimated from the climate model data by analysing the local
correlation, i.e. at the same grid cell, between temperature and
precipitation-weighted temperature.

We further assume that the effect of precipitation intermittency is not
independent between sites but is related to the spatial coherence of the
precipitation fields, for which we assume an exponential decorrelation structure
with a decay length $\tau_{\mathrm{pw}}$. Following these assumptions, the
spatial covariance between sites of the white noise terms induced by the effect
of precipitation intermitteny has the form
%
\begin{equation}
\label{eq:noise.cov}
\mathrm{cov}(\varepsilon_i,\varepsilon_j)=
\exp{\left(-\frac{d_{ij}}{\tau_{\mathrm{pw}}}\right)}.
\end{equation}
%
Then, the correlation between the target site temperature and the spatial
average of $N$ precipitation-weighted temperature time series is governed by the
intermittency factor $\xi$ and by the two spatial length scales $\tau$ and
$\tau_{\mathrm{pw}}$,
%
\begin{equation}
\label{eq:t2m.pw.corr}
\mathrm{cor}\left(T_0,\overline{T}^{\mathrm{(pw)}}\right)=
\frac
{\sqrt{1-\xi}\sum_{i=1}^{N}\exp{\left(-\frac{r_i}{\tau}\right)}}
{\sqrt{N + 2\sum_{i=1}^{N-1}\sum_{j=i+1}^{N}
  g(d_{ij}; \tau, \tau_{\mathrm{pw}}, \xi)}},
\end{equation}
%
with
\begin{equation}
\label{eq:exp.fun}
g(z; \tau, \tau_{\mathrm{pw}}, \xi):=
(1-\xi)\exp{\left(-\frac{z}{\tau}\right)} +
\xi\exp{\left(-\frac{z}{\tau_{\mathrm{pw}}}\right)}.
\end{equation}

\subsection{Precipitation-weighted oxygen isotope composition}
\label{app:concept.model.oxy.pw}

For the field of precipitation-weighted oxygen isotope composition, $x \equiv
\delta^{\mathrm{(pw)}}$, we assume the same effect of precipitation
intermittency as for the temperature field. Furthermore, an analysis of the
climate model data suggests that also the oxygen isotope field largely exhibits
an exponential decorrelation structure in space (not shown). Hence, the
correlation between the target site temperature and the spatial average of $N$
$\delta^{\mathrm{(pw)}}$ time series is obtained in a similar manner as for
$T^{\mathrm{(pw)}}$, i.e.
%
\begin{equation}
\label{eq:oxy.pw.corr}
\mathrm{cor}\left(T_0,
  \overline{\delta}^{\mathrm{(pw)}}\right)=
\frac
{\sqrt{1-\xi}\sum_{i=1}^{N}\mathrm{cor}\left(T_0,\delta_i\right)}
{\sqrt{N + 2\sum_{i=1}^{N-1}\sum_{j=i+1}^{N}
  g(d_{ij}; \tau_{\delta}, \tau_{\mathrm{pw}}, \xi)}},
\end{equation}
%
where $\tau_{\delta}$ is the decorrelation length of the $\delta$ field and the
only difference to Eq.~\eqref{eq:t2m.pw.corr} is the unknown spatial correlation
structure between the temperature at the target site and the oxygen isotope
field, $\mathrm{cor}\left(T_0,\delta_i\right)$.  Based on our climate model
results (Fig.~\ref{fig:avg.cor.structure}), we parameterise this function with a
simple linear decay of the form
%
\begin{equation}
\label{eq:t2m.oxy.corr}
\mathrm{cor}\left(T_0,\delta_i\right)=
\begin{cases}
  c_0 - \gamma d, & d \le d_0,\\
  c_1, & d > d_0,
\end{cases}
\end{equation}
%
where $\gamma=(c_0-c_1)/d_0$ and $d_0$ is some threshold distance above which
the correlation stays constant.

\subsection{Model parameter estimation and model results}
\label{app:concept.model.estimation}

In total, our model is governed by three decorrelation lengths ($\tau$,
$\tau_{\delta}$, $\tau_{\mathrm{pw}}$), the intermittency factor $\xi$, and
three parameters describing the temperature--isotope correlation ($c_0$, $c_1$,
$d_0$).

We estimate $\tau$ from the climate model data for the DML and Vostok regions
(Fig.~\ref{fig:avg.cor.structure}), as well as in the same way
$\tau_{\delta}$, and find for both regions values of $\tau=1900$\,km and
$\tau_{\delta}=1100$\,km. The intermittency factor $\xi$ follows from the
local correlation between temperature and precipitation-weighted temperature
(Eq.~\ref{eq:t2m.pw.decorr}). We find an average value for the DML region of
$\xi_{\mathrm{DML}}=0.73$ which is close to the average value for entire
Antarctica ($\xi_{\mathrm{Ant.}}=0.71$), while the intermittency is stronger for
the Vostok region ($\xi_{\mathrm{Vostok}}=0.82$). We parameterise the
temperature--isotope correlation in the DML region with $c_0=0.4$, $c_1=0.2$,
$d_0=1500$\,km, while the values are rather $c_0=0.5$ and $c_1=0$ for
$d_0>2000$\,km in the Vostok region (Fig.~\ref{fig:avg.cor.structure}). The only
unconstrained parameter is the decorrelation length of the effect of
precipitation intermittency, $\tau_{\mathrm{pw}}$, for which we choose a value
of $500$\,km based on investigations with reanalysis data \citep{Munch2018a}.

We can test our assumption for the effect of intermittency based on using the
estimated values of $\tau$ and $\xi$ to predict the spatial decorrelation
between temperature and precipitation-weighted temperature
(Eq.~\ref{eq:t2m.pw.decorr}). Indeed, this yields a comparably good fit to the
data as an independent fit (root mean square deviation of $\sim0.03$ between
data and fit in both cases), supporting our assumption that intermittency can be
parameterised by a partial conversion of the time series into white noise.

%fB01
\begin{figure*}[t]%
\centering
\includegraphics[width=17cm]{../plots/main/fig_B01.png}
\caption{%
  Two-dimensional sampling correlation structures with temperature as predicted
  from our conceptual model using the model parameters for the DML region. Shown
  is the mean correlation of all possible single correlations from averaging two
  time series sampled from a pair of concentric rings around the target site for
  the fields of (\textbf{a}) $T_{\mathrm{2m}}$, (\textbf{b})
  $T_{\mathrm{2m}}^{\mathrm{(pw)}}$ and (\textbf{c})
  $\delta^{18}\mathrm{O}^{\mathrm{(pw)}}$.}
\label{fig:conceptual.model}%
\end{figure*}%

In a similar manner as for analysing the climate model data, we now use our
conceptual model to predict the two-dimensional ($N=2$) sampling correlation
structures for the different model fields of $T_{\mathrm{2m}}$,
$T_{\mathrm{2m}}^{\mathrm{(pw)}}$ and $\delta^{18}\mathrm{O}^{\mathrm{(pw)}}$
(Eqs.~\ref{eq:t2m.corr}, \ref{eq:t2m.pw.corr} and \ref{eq:oxy.pw.corr}). Since
our model space is continuous, we sample from locations placed \emph{on}
concentric rings around the target site. We either sample the two locations from
the same ring or from two different rings, using ring radii from $0$ to
$2000$\,km in increments of $10$\,km, and calculate the expected correlation for
a specific ring combination. To obtain meaningful expectation values, we choose
$36$ locations distributed uniformly across each ring in steps of $10\degree$,
combine these locations one by one for each ring combination, and average across
the correlations from each location pair. With the model parameters for the DML
region we obtain the results displayed in Fig.~(\ref{fig:conceptual.model}),
which we discuss and compare to the estimated results from the climate model
data in the main text.

\noappendix

% \authorcontribution{TEXT} %% this section is mandatory
% \competinginterests{TEXT} %% this section is mandatory even if you declare that no competing interests are present
% \disclaimer{TEXT} %% optional section

% \begin{acknowledgements}
% \end{acknowledgements}

\bibliographystyle{copernicus}
\bibliography{muench_etal_optimalcores}

\end{document}
