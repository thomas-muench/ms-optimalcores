\documentclass[cp, manuscript]{copernicus}

\begin{document}

\title{[working title] Optimal ice-core arrangement}

\Author[1]{Thomas}{M\"{u}nch}
\Author[2]{Martin}{Werner}
\Author[1,3]{Thomas}{Laepple}

\affil[1]{Alfred-Wegener-Institut Helmholtz-Zentrum f{\"u}r Polar- und
Meeresforschung, Research Unit Potsdam, Telegrafenberg A45, 14473 Potsdam,
Germany}
\affil[2]{Alfred-Wegener-Institut Helmholtz-Zentrum f{\"u}r Polar- und
Meeresforschung, Bussestra{\ss}e 24, 27570 Bremerhaven, Germany}
\affil[3]{University of Bremen, MARUM~--~Center for Marine Environmental
  Sciences and Faculty of Geosciences, 28334 Bremen, Germany}

\correspondence{Thomas M\"unch (thomas.muench@awi.de)}
\runningtitle{Optimal ice-core arrangement}
\runningauthor{T. M\"unch et al.}

\received{}
\pubdiscuss{}
\revised{}
\accepted{}
\published{}

\firstpage{1}

\maketitle

\begin{abstract} Many proxies used in climate research share one complicating
property: they are not only driven by the climatic target variable of interest,
e.g. temperature, but also influenced by secondary effects which cause
additional variability, frequently termed \emph{noise}. Noise in individual
proxy records can be reduced generally by averaging them together, but the
effectiveness of this approach depends on the spatial correlation scales of the
involved noise-generating processes. Here, we review this concept in the context
of Antarctic ice cores and apply it in order to optimise local to regional scale
temperature reconstructions from ice-core isotope records. We use data from
climate model simulations to identify an optimal arrangement of ice-core arrays
which maximises the signal-to-noise ratio in the reconstruction with minimum
sampling effort. A first intriguing result is that, in order to optimise
temperature reconstructions at a certain target site, it is necessary to combine
local with more distant ice-core records. More specifically, for a target site
in Dronning Maud Land on the East Antarctic Plateau (\#EDIT generalise to EAP?)
our results suggest to use both local records and ice cores separated
$\sim500-1000$\,km away from the target. We show that these findings can be
largely explained by the interplay of the two spatial scales associated with the
correlation structure of the temperature field and of the noise generated by
precipitation intermittency. In summary, our study helps to maximise the
usability of existing ice cores as well as to optimally plan future drilling
campaigns. It also deepens our knowledge concerning the typical correlation
scales of the different processes that shape the isotopic record. Finally, the
presented method can be direcly extended to other palaeoclimate reconstruction
problems.
\end{abstract}

\introduction

The oxygen and hydrogen isotopic composition of firn and ice recovered from
polar ice cores is a key proxy for past near-surface air temperature changes
(\#REF). Although the physical mechanisms that link local changes in temperature
to the isotopic composition of precipitated snow are generally well understood
(\#REF) and can be modelled with general circulation models (\#REF), the
quantitative interpretation of ice-core isotope variabily in terms of
temperature variability is complicated by second order processes that
additionally influence the isotopic record, creating noise (\#REF). Analysing
the typical spatial scales on which the different processes act helps to
minimise variability, which is not related to temperature, by averaging ice-core
records from an array of sites.

The isotopic record that is derived from an ice core is the result of a chain of
processes from (1) atmospheric temperature changes to (2) the isotopic
fractionation during the pathway of the atmospheric moisture, (3) the effect of
variable and intermittent precipitation and finally (4) post-depositional
effects; each element of this chain can be associated with a typical spatial
scale.

Atmospheric temperature variations drive the fractionation of the isotopic
composition of the atmospheric moisture along its pathway to the final stage of
precipitation (\#REF). The spatial coherence of the temperature-related isotopic
signal in precipitation is hence given by the spatial coherence of the
atmospheric temperature field itself. Typical spatial decorrelation scales for
temperature are of the order of $\gtrsim1000$\,km (\#REF), which implies that
ice cores distributed on spatial scales below $\sim 1000$\,km should record a
similar, i.e. correlated, temperature signal. However, the temporal variability
of the isotopic composition in the local atmospheric moisture also depends on
the variability of the atmospheric circulation, since different air masses may
exhibit different source regions and distillation pathways (\#REF). In addition,
in an ice core, the isotopic composition in a deposited layer of snow will not
one-to-one reflect the temporal variability of the atmospheric isotope signal
due to the intermittent nature of precipitation (\#REF), which weights the
initial isotope signal with the amount of precipitation, thereby introducing
bias and adding additional variability to the isotopic record (\#REF). Both
processes are directly linked to the atmospheric dynamics, and their typical
scales are hence expected to range from mesoscales (i.e. tens of kilometres),
driven by topography and orographic effects, to synoptic scales of hundreds of
kilometres associated with cyclonic activity and the movement of high and low
presssure systems (\#REF). Finally, in polar conditions the precipitated snow
does not directly settle but is constantly eroded, blown away and
redeposited. It has been shown that this gives rise to stratigraphic noise in
the isotopic record (\#REF:fisher) which exhibits a small-scale decorrelation of
a few metres (\#REF:trench).%
\footnote{We note that the final isotopic record is influenced by further
  post-depositional processes within the snow and ice matrix, such as
  densification and diffusion, which are, however, not within the scope of this
  article.}

This hierarchy of process scales~--~from temperature to the atmospheric isotopic
composition, to the precipitation-weighted isotopic composition, and finally to
post-depositional effects~--, determines the effectiveness of reducing overall
noise by averaging isotope records, since the reduction of the noise level will
depend on the spatial correlation scale of the different noise sources. For
example, if we only considered temperature and stratigraphic noise, it would be
sufficient to average records spaced just some tens of metres apart, as this
would ensure highly correlated temperature signals but uncorrelated
stratigraphic noise between the records. However, the comparison of
correlation-based signal-to-noise ratios of nearby isotope records (\#REF) with
those estimated from analysing the record's temporal variability (\#REF) has
shown that reproducibility on a local scale does not necessarily imply a
climatic, i.e. temperature-driven, origin. Instead, circulation variability and
precipitation intermittency act as further noise sources which are expected to
exhibit larger decorrelation lengths (\#REF:cp18,cycles) than the stratigraphic
noise. Taking this into account, we expect some optimal length scale in between
the local and the temperature decorrelation scale which is a trade-off between
averaging out circulation and intermittency effects and ensuring a sufficient
coherence of the recorded temperature signal.

Here, we use data from a climate model equipped with stable isotope diagnostics
to learn about the different process scales and to link this to the optimal
arrangement of ice-core locations, with a focus on the East Antarctic Plateau
(EPA). For a target location in Dronning Maud Land (\#EDIT), our results suggest
a combination of local records with ice cores separated $\sim500-1000$\,km away
from the target in order to maximise the correlation with the target temperature
signal. This result also holds when assessing the range of correlations for an
arbitrary sampling of such combinations (\#riskanalysis). In general, optimal
sampling can increase the correlation with the target temperature by x-y\,\%
when averaging x-y cores as compared to a local reconstruction using only a
single record (\#EDIT).

\section{Data and methods}\label{methods}

\subsection{Climate model data}\label{methods:data}

We use data of the past-millennium simulation (800--2000\,CE; \#REF:Sjolte2018)
of the fully coupled ECHAM5/MPI-OM-wiso atmosphere--ocean general circulation
model equipped with stable isotope diagnostics (\#REF:Werner2016). This
simulation is forced by greenhouse gases, volcanic aerosols, total solar
irradiance and changes in land use and Earth's orbital parameters. The
atmospheric component of the model has a T31 spectral resolution
($3.75\degree\times3.75\degree$) with $19$ vertical levels
(\#REF:Sjolte2018). Compared to observations, the atmospheric model ECHAM5-wiso
generally reproduces the climatological relation between temperature and
precipitation isotopic composition well, but has, in the used T31 setup, a warm
bias and is not depleted enough in isotopic composition over Antarctica
(\#REF:Werner2011). This can be improved by a higher spatial resolution
(\#REF:Werner2011) but is not relevant for our study since we are mainly
interested in the relative variability between sites. The full atmosphere--ocean
model was evaluated against observations for equilibrium simulations under
pre-industrial and Last Glacial Maximum conditions (\#REF:Werner2016), and the
past-millennium simulation was used to reconstruct North Atlantic atmospheric
circulation in combination with ice-core isotope data (\#REF:Sjolte2018).

Here, we use the ECHAM5/MPI-OM-wiso time series for all model grid cells
on the Antarctic continent (in total $442$) of two-metre air temperature
($T_{2\mathrm{m}}$), precipitation ($p$) and oxygen isotopic composition in
precipitation (relative abundance of oxygen-18 to oxygen-16 istopes,
$\delta^{18}\mathrm{O}$).

\subsection{Data processing}\label{methods:prc}

The model simulation output has monthly temporal resolution, while ice-core
isotope records are typically analysed on annual or longer resolution achieved
by averaging the isotopic data across the respective amount of snow and
ice. The annual data of isotopic composition will then include a weighting
effect due to the intra-annual variability in the amount of precipitation. To
account for this, we produce two versions of annual data from the monthly model
output: (1) the two-metre temperature and oxygen isotopic composition are simply
averaged to annual resolution without any weighting ($T_{2\mathrm{m}}$ and
$\delta^{18}\mathrm{O}$ in the following), (2) the respective monthly data is
averaged to annual resolution including a weighting by the monthly amount of
precipitation (denoted as precipitation-weighted data
$T_{2\mathrm{m}}^{\mathrm{(pw)}}$ and $\delta^{18}\mathrm{O}^{\mathrm{(pw)}}$ in
the following).

\subsection{Methods to analyse the model data}\label{methods:main}

All used methods are essentially based on analysing the correlation between
model variables, for which we use the Pearson correlation coefficient. To link
model data to ice cores, we mimick ice-core isotope records by associating them
with the $\delta^{18}\mathrm{O}^{\mathrm{(pw)}}$ time series at the model grid
cells. We thus neglect stratigraphic noise and any post-depositional effects on
the isotopic record since here we are interested in the upper limit of the
extent to which ice cores can reconstruct the climatic temperature signal.

\subsubsection{Picking optimal sites}\label{methods:picking}

To obtain an optimal set of cores to reconstruct $T_{2\mathrm{m}}$ at a given
grid cell $r_0$ (target site), we randomly pick without replacement $N$ of the
grid cells that lie within a circle of $1000$\,km radius around the target site
and correlate the average $\delta^{18}\mathrm{O}^{\mathrm{(pw)}}$ time series of
these $N$ grid cells with the target site temperature. The optimal set of cores
for each $N$ is then determined from the maximum correlation across all picking
trials. To ensure stable results, we set the maximum number of trials to
$n=1000$ (\#EDIT:check/update this number).

\subsubsection{Generalisation to optimal core spacings}\label{methods:general}

The optimal sites determined from the picking approach described above will
strongly depend on the used model simulation and the analysed time span. To
overcome this issue, we generalise the approach by defining consecutive rings
around a target site of $250$\,km radial width until a maximum distance from the
target of $2000$\,km (Fig.~\ref{fig:concept}). We determine all grid cells that fall into
each of these rings and then assess the possibilities of combining $N$ grid
cells. This is implemented as a two-step process: (1) we determine all possible
combinations of selecting $N$ rings with replacement, and then, (2), identify
for each ring combination the possibilities of combining single grid cells from
these rings. For this second part, it is analytically feasible to identify all
possible grid-cell combinations until $N=2$; for $N\geq3$ we resort to Monte
Carlo sampling, for which we use a maximum number of Monte Carlo trials of
$n=1000$ (\#EDIT:check/update this number). Finally, for a specific ring
combination, we average the $\delta^{18}\mathrm{O}^{\mathrm{(pw)}}$ time series
for each grid-cell combination and compute the correlation with the target
site temperature. We report the expected correlation for this ring combination
by averaging across all correlations of the analysed grid-cell combinations.

%f01
\begin{figure}[t]%
\centering
\includegraphics[width=8.5cm]{../plots/echam5_mpiom_wiso_fig_01.pdf}
\caption[Conceptual approach]{%
  Conceptual sketch of the applied general approach. Around a given Antarctic
  target site (black cross), we define consecutive rings (red lines) of
  $250$\,km radial width and analyse all model grid cells that fall into each of
  the rings.}
\label{fig:concept}%
\end{figure}%

\subsubsection{Estimation of decorrelation lengths}\label{methods:decor.model}

Spatial decorrelation lengths are estimated throughout the paper by fitting an
isotropic exponential model to the correlation--distance dependence at a target
site of the form
\begin{equation}\label{eq:decor.model}
c(r) = \exp{\left(-\frac{r}{\tau}\right)},
\end{equation}
where $c(r)$ is the fitted correlation at some distance $r$ from the target
site, and $\tau$ is the estimated decorrelation length at which the correlation
has declined to $1/e$ ($\sim37\,\%$). The input relationship between correlation
and distance for the variable in question is obtained from correlating the time
series at the target site with every other grid cell on the Antarctic continent
and recording the respective distances between the grid cell
centres.














































%\copyrightstatement{}

% \codedataavailability{TEXT} %% use this section when having data sets and software code available

% \appendix
% \section{}    %% Appendix A
% \subsection{}     %% Appendix A1, A2, etc.

% \noappendix       %% use this to mark the end of the appendix section

% \authorcontribution{TEXT} %% this section is mandatory
% \competinginterests{TEXT} %% this section is mandatory even if you declare that no competing interests are present
% \disclaimer{TEXT} %% optional section

% \begin{acknowledgements}
% \end{acknowledgements}

% \bibliographystyle{copernicus}
% \bibliography{muench-etal_optimalcores}

\end{document}
