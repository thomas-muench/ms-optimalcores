\documentclass[cp, manuscript]{copernicus}

\begin{document}

\title{[working title] Optimal ice-core arrangement}

\Author[1]{Thomas}{M\"{u}nch}
\Author[2]{Martin}{Werner}
\Author[1,3]{Thomas}{Laepple}

\affil[1]{Alfred-Wegener-Institut Helmholtz-Zentrum f{\"u}r Polar- und
Meeresforschung, Research Unit Potsdam, Telegrafenberg A45, 14473 Potsdam,
Germany}
\affil[2]{Alfred-Wegener-Institut Helmholtz-Zentrum f{\"u}r Polar- und
Meeresforschung, Bussestra{\ss}e 24, 27570 Bremerhaven, Germany}
\affil[3]{University of Bremen, MARUM~--~Center for Marine Environmental
  Sciences and Faculty of Geosciences, 28334 Bremen, Germany}

\correspondence{Thomas M\"unch (thomas.muench@awi.de)}
\runningtitle{Optimal ice-core arrangement}
\runningauthor{T. M\"unch et al.}

\received{}
\pubdiscuss{}
\revised{}
\accepted{}
\published{}

\firstpage{1}

\maketitle

\begin{abstract} Many proxies used in climate research share one complicating
property: they are not only driven by the climatic target variable of interest,
e.g. temperature, but also influenced by secondary effects which cause
additional variability, frequently termed \emph{noise}. Noise in individual
proxy records can be reduced generally by averaging them together, but the
effectiveness of this approach depends on the spatial correlation scales of the
involved noise-generating processes. Here, we review this concept in the context
of Antarctic ice cores and apply it in order to optimise local to regional scale
temperature reconstructions from ice-core isotope records. We use data from
climate model simulations to identify an optimal arrangement of ice-core arrays
which maximises the signal-to-noise ratio in the reconstruction with minimum
sampling effort. A first intriguing result is that, in order to optimise
temperature reconstructions at a certain target site, it is necessary to combine
local with more distant ice-core records. More specifically, for a target site
in Dronning Maud Land on the East Antarctic Plateau (\#EDIT generalise to EAP?)
our results suggest to use both local records and ice cores separated
$\sim500-1000$\,km away from the target. We show that these findings can be
largely explained by the interplay of the two spatial scales associated with the
correlation structure of the temperature field and of the noise generated by
precipitation intermittency. In summary, our study helps to maximise the
usability of existing ice cores as well as to optimally plan future drilling
campaigns. It also deepens our knowledge concerning the typical correlation
scales of the different processes that shape the isotopic record. Finally, the
presented method can be direcly extended to other palaeoclimate reconstruction
problems.
\end{abstract}

\introduction

The oxygen and hydrogen isotopic composition of firn and ice recovered from
polar ice cores is a key proxy for past near-surface air temperature changes
(\#REF). Although the physical mechanisms that link local changes in temperature
to the isotopic composition of precipitated snow are generally well understood
(\#REF) and can be modelled with general circulation models (\#REF), the
quantitative interpretation of ice-core isotope variabily in terms of
temperature variability is complicated by second order processes that
additionally influence the isotopic record, creating noise (\#REF). Analysing
the typical spatial scales on which the different processes act helps to
minimise variability, which is not related to temperature, by averaging ice-core
records from an array of sites.

The isotopic record that is derived from an ice core is the result of a chain of
processes from (1) atmospheric temperature changes to (2) the isotopic
fractionation during the pathway of the atmospheric moisture, (3) the effect of
variable and intermittent precipitation and finally (4) post-depositional
effects; each element of this chain can be associated with a typical spatial
scale.

Atmospheric temperature variations drive the fractionation of the isotopic
composition of the atmospheric moisture along its pathway to the final stage of
precipitation (\#REF). The spatial coherence of the temperature-related isotopic
signal in precipitation is hence given by the spatial coherence of the
atmospheric temperature field itself. Typical spatial decorrelation scales for
temperature are of the order of $\gtrsim1000$\,km (\#REF), which implies that
ice cores distributed on spatial scales below $\sim 1000$\,km should record a
similar, i.e. correlated, temperature signal. However, the temporal variability
of the isotopic composition in the local atmospheric moisture also depends on
the variability of the atmospheric circulation, since different air masses may
exhibit different source regions and distillation pathways (\#REF). In addition,
in an ice core, the isotopic composition in a deposited layer of snow will not
one-to-one reflect the temporal variability of the atmospheric isotope signal
due to the intermittent nature of precipitation (\#REF), which weights the
initial isotope signal with the amount of precipitation, thereby introducing
bias and adding additional variability to the isotopic record (\#REF). Both
processes are directly linked to the atmospheric dynamics, and their typical
scales are hence expected to range from mesoscales (i.e. tens of kilometres),
driven by topography and orographic effects, to synoptic scales of hundreds of
kilometres associated with cyclonic activity and the movement of high and low
presssure systems (\#REF). Finally, in polar conditions the precipitated snow
does not directly settle but is constantly eroded, blown away and
redeposited. It has been shown that this gives rise to stratigraphic noise in
the isotopic record (\#REF:fisher) which exhibits a small-scale decorrelation of
a few metres (\#REF:trench).%
\footnote{We note that the final isotopic record is influenced by further
  post-depositional processes within the snow and ice matrix, such as
  densification and diffusion, which are, however, not within the scope of this
  article.}

This hierarchy of process scales~--~from temperature to the atmospheric isotopic
composition, to the precipitation-weighted isotopic composition, and finally to
post-depositional effects~--, determines the effectiveness of reducing overall
noise by averaging isotope records, since the reduction of the noise level will
depend on the spatial correlation scale of the different noise sources. For
example, if we only considered temperature and stratigraphic noise, it would be
sufficient to average records spaced just some tens of metres apart, as this
would ensure highly correlated temperature signals but uncorrelated
stratigraphic noise between the records. However, the comparison of
correlation-based signal-to-noise ratios of nearby isotope records (\#REF) with
those estimated from analysing the record's temporal variability (\#REF) has
shown that reproducibility on a local scale does not necessarily imply a
climatic, i.e. temperature-driven, origin. Instead, circulation variability and
precipitation intermittency act as further noise sources which are expected to
exhibit larger decorrelation lengths (\#REF:cp18,cycles) than the stratigraphic
noise. Taking this into account, we expect some optimal length scale in between
the local and the temperature decorrelation scale which is a trade-off between
averaging out circulation and intermittency effects and ensuring a sufficient
coherence of the recorded temperature signal.

Here, we use data from a climate model equipped with stable isotope diagnostics
to learn about the different process scales and to link this to the optimal
arrangement of ice-core locations, with a focus on the East Antarctic Plateau
(EPA). For a target location in Dronning Maud Land (\#EDIT), our results suggest
a combination of local records with ice cores separated $\sim500-1000$\,km away
from the target in order to maximise the correlation with the target temperature
signal. This result also holds when assessing the range of correlations for an
arbitrary sampling of such combinations (\#riskanalysis). In general, optimal
sampling can increase the correlation with the target temperature by x-y\,\%
when averaging x-y cores as compared to a local reconstruction using only a
single record (\#EDIT).

\section{Data and methods}\label{methods}

\subsection{Climate model data}\label{methods:data}

We use data of the past-millennium simulation (800--2000\,CE; \#REF:Sjolte2018)
of the fully coupled ECHAM5/MPI-OM-wiso atmosphere--ocean general circulation
model equipped with stable isotope diagnostics (\#REF:Werner2016). This
simulation is forced by greenhouse gases, volcanic aerosols, total solar
irradiance and changes in land use and Earth's orbital parameters. The
atmospheric component of the model has a T31 spectral resolution
($3.75\degree\times3.75\degree$) with $19$ vertical levels
(\#REF:Sjolte2018). Compared to observations, the atmospheric model ECHAM5-wiso
generally reproduces the climatological relation between temperature and
precipitation isotopic composition well, but has, in the used T31 setup, a warm
bias and is not depleted enough in isotopic composition over Antarctica
(\#REF:Werner2011). This can be improved by a higher spatial resolution
(\#REF:Werner2011) but is not relevant for our study since we are mainly
interested in the relative variability between sites. The full atmosphere--ocean
model was evaluated against observations for equilibrium simulations under
pre-industrial and Last Glacial Maximum conditions (\#REF:Werner2016), and the
past-millennium simulation was used to reconstruct North Atlantic atmospheric
circulation in combination with ice-core isotope data (\#REF:Sjolte2018).

Here, we use the ECHAM5/MPI-OM-wiso time series for all model grid cells
on the Antarctic continent ($N_{\mathrm{grid}}=442$) of two-metre air
temperature ($T_{2\mathrm{m}}$), precipitation ($p$) and oxygen isotopic
composition in precipitation (relative abundance of oxygen-18 to oxygen-16
istopes, denoted as $\delta^{18}\mathrm{O}$).

\subsection{Data processing}\label{methods:prc}

The model simulation output has monthly temporal resolution, while ice-core
isotope records are typically analysed on annual or longer resolution achieved
by averaging the isotopic data across the respective amount of snow and
ice. The annual data of isotopic composition will then include a weighting
effect due to the intra-annual variability in the amount of precipitation. To
account for this, we produce two versions of annual data from the monthly model
output: (1) the two-metre temperature and oxygen isotopic composition are simply
averaged to annual resolution without any weighting ($T_{2\mathrm{m}}$ and
$\delta^{18}\mathrm{O}$ in the following), (2) the respective monthly data are
averaged to annual resolution including a weighting by the monthly amount of
precipitation (denoted as precipitation-weighted data
$T_{2\mathrm{m}}^{\mathrm{(pw)}}$ and $\delta^{18}\mathrm{O}^{\mathrm{(pw)}}$ in
the following).

\subsection{Data analyses}\label{methods:main}

\subsubsection{General approach}

We investigate the relationship between the various model variables by assessing
the Pearson correlation coefficient. To link model data to ice cores, we mimic
ice-core isotope records by associating them with the
$\delta^{18}\mathrm{O}^{\mathrm{(pw)}}$ time series at the model grid cells. We
thus neglect stratigraphic noise and any post-depositional effects on the
isotopic record since here we are interested in the upper limit of the extent to
which ice cores can reconstruct the climatic temperature signal. In order to
learn about the typical spatial scales that govern the temperature--isotope
relationship, we set up the following general scheme: For a given model grid
cell $r_0$ of interest (target site in the following), we define consecutive
rings around this site of $250$\,km radial width until a maximum distance of
$2000$\,km (Fig.~\ref{fig:concept}). Then, we determine all grid cells that fall
into each of these rings and either sample these grid cells directly or sample
from the possibilities of combining model grid cells and rings.

%f01
\begin{figure}[t]%
\centering
\includegraphics[width=6.5cm]{../plots/main/fig_01.pdf}
\caption[Conceptual approach]{%
  Conceptual sketch of the general approach. Around a given Antarctic target
  site (black cross), we define consecutive rings (red lines) of $250$\,km
  radial width and analyse all model grid cells that fall into each of the
  rings.}
\label{fig:concept}%
\end{figure}%

\subsubsection{Estimation of decorrelation lengths and spatial correlation
  structures}\label{methods:decor.model}

We estimate spatial temperature decorrelation lengths by fitting an isotropic
exponential model to the correlation--distance dependence of the form
\begin{equation}\label{eq:decor.model}
c(r) = \exp{\left(-\frac{r}{\tau}\right)},
\end{equation}
where $c(r)$ is the fitted correlation at some distance $r$ from the target, and
$\tau$ is the estimated decorrelation length at which the correlation has
declined to $1/e$ ($\sim37\,\%$). The input relationship between correlation and
distance is obtained in two different ways: (1) For a single target site, we
correlate the $T_{2\mathrm{m}}$ time series at the target site with the
$T_{2\mathrm{m}}$ time series of every other grid cell on the Antarctic
continent and record the respective distances between the sites. (2) To reduce
the uncertainty of these individual correlation estimates, we average across the
correlations obtained between the $T_{2\mathrm{m}}$ time series at a target site
and the $T_{2\mathrm{m}}$ time series of grid cells sampled from rings
(Fig.~\ref{fig:concept}). In a second averaging step, we compute the mean of the
so obtained correlation--distance relationships for several target sites from a
given Antarctic region and use that as input to Eq.~\eqref{eq:decor.model}.

The spatial correlation structure between the target site temperature and other
model variables is assessed in the same manner as for the average temperature
decorrelation.

\subsubsection{Picking optimal sites}\label{methods:picking}

To obtain an optimal set of ice cores to reconstruct $T_{2\mathrm{m}}$ at a
given target site, we randomly pick without replacement $N$ of the
grid cells that lie within a circle of $1000$\,km radius around the target site
and correlate the average $\delta^{18}\mathrm{O}^{\mathrm{(pw)}}$ time series of
these $N$ grid cells with the target site temperature. The optimal set of cores
for each $N$ is then determined from the maximum correlation across all picking
trials. To ensure stable results, we set the maximum number of trials to
$n=5000$ for $N=1$ and to $n=10^5$ for $N\geq3$.

\subsubsection{Optimal ice-core spacing}\label{methods:general}

The optimal sites determined from the picking approach described above will
strongly depend on the used model simulation and the analysed time span. To
overcome this issue, we generalise the approach by sampling $N$ sites from the
ring regions around the target site (Fig.~\ref{fig:concept}).  This is
implemented as a two-step process: (1) we determine all possible combinations of
selecting $N$ rings with replacement, and then, (2), identify for each ring
combination the possibilities of combining single grid cells from each of the
rings. For this second part, it is analytically feasible to identify all
possible grid-cell combinations until $N=2$; for $N\geq3$, we resort to Monte
Carlo sampling, for which we use a maximum number of Monte Carlo trials of
$n=1000$ (\#EDIT:check/update this number). Finally, for a specific ring
combination, we average the $\delta^{18}\mathrm{O}^{\mathrm{(pw)}}$ time series
for each grid-cell combination and compute the correlation with the target site
temperature. We then report the expected correlation for this ring combination
by averaging across all correlations of the analysed grid-cell combinations.

\section{Results}\label{results}

\subsection{Spatial correlation structures}\label{results:cor.struct}

We first examine the spatial correlation structure of the analysed model
variables.

The near-surface atmospheric temperature field over Antarctica generally
exhibits large-scale coherent variations (Fig.~\ref{fig:correlation.maps}a),
however, with a clear two-part structure roughly divided by the range of the
Transantarctic Mountains: For most regions of the East Antarctic Plateau, the
temperature field has typical decorrelation lengths of $\sim1500$ to $2500$\,km,
while the decorrelation lengths are significantly lower with values below
$1000$\,km for larger parts of the West Antarctic Ice Sheet and for the
Antarctic Peninsula. Our analysis implicitly assumes an exponential dependence
on distance of the temperature correlation between sites
(Eq.~\ref{eq:decor.model}). We verify this assumption by analysing the expected
correlation as a function of distance averaged across grid cells and across a
region of target sites (Fig.~\ref{fig:avg.decor.lengths}) in order to reduce the
uncertainty associated with the individual correlation estimates. Here, as an
example, we choose the EPICA Dronning Maud Land (EDML) drilling site
($-75\degree$\,S, $0\degree$\,E) as a starting point and use all grid cells
within a range of $\pm17.5\degree$ longitude and $\pm5\degree$ latitude as
target sites (Dronning Maud Land (DML) region in the following). We find the
temperature decorrelation in this region to clearly follow an exponential decay
with a length scale of $\sim1900$\,km. Assuming ice cores with a perfect
temperature proxy, a single core would hence be enough to capture the
temperature variability across hundreds of kilometres.

%f02
\begin{figure*}[t]%
\centering
\includegraphics[width=16cm]{../plots/main/fig_02.png}
\caption{%
  Temperature decorrelation lengths and temperature--isotope
  relationship. (\textbf{a}) The temperature decorrelation lengths ($\tau$, in
  km) for each Antarctic model grid cell estimated by fitting an exponential
  model to the correlation--distance relationship obtained from correlating the
  local near-surface $T_{2\mathrm{m}}$ time series with the temperature time
  series from all other grid cells (Eq.~\ref{eq:decor.model}). Note that only
  continental grid cells are used for the fit. (\textbf{b}) The local
  correlation between the interannual near-surface temperature
  ($T_{2\mathrm{m}}$) and precipitation-weighted oxygen isotope composition
  ($\delta^{18}\mathrm{O}^{\mathrm{(pw)}}$) time series for each Antarctic model
  grid cell.}
\label{fig:correlation.maps}%
\end{figure*}%

We now assess the extent to which actual single Antarctic ice-core isotope
records explain local temperature variations. Here, we can distinguish between
the unweighted interannual field of $\delta^{18}\mathrm{O}$ and the
precipitation-weighted field of $\delta^{18}\mathrm{O}^{\mathrm{(pw)}}$, which
most closely mimicks a real ice-core record. We first analyse the local
correlation between the $T_{2\mathrm{m}}$ and
$\delta^{18}\mathrm{O}^{\mathrm{(pw)}}$ time series for each model grid cell
(Fig.~\ref{fig:correlation.maps}b). We find the temperature--isotope correlations
to be generally low, ranging across all analysed grid cells from virtually $0$
up to $\sim0.53$, however, with $\sim75\,\%$ of the correlations $\leq0.4$ and a
mean correlation of $0.36$. In the second step, we determine the average spatial
strucure of the temperature--isotope correlation as a function of distance
(Fig.~\ref{fig:avg.decor.lengths}), assessed for our DML region in the same
manner as the average temperature decorrelation structure presented above. In
contrast to the exponential temperature decorrelation, the
$\delta^{18}\mathrm{O}$ field in the DML region exhibits a much lower but also
more stable structure of correlation with the target site temperature. Starting
from a local value of $\sim0.4$, the correlations decrease only slightly with
increasing distance up to $\sim1000$\,km, followed by a little steeper decrease
for larger distances and constant levels of $\lesssim0.2$ for distances above
$\sim1500$\,km. Precipitation weighting, as seen from analysing the
$\delta^{18}\mathrm{O}^{\mathrm{(pw)}}$ field, results in overall even lower
correlation values, but it does not affect much the correlation structure
itself.

%f03
\begin{figure}[t]%
\centering
\includegraphics[width=8.3cm]{../plots/main/fig_03.pdf}
\caption{%
  Spatial correlation structure in the model data for the DML region. Shown is
  the average correlation as a function of distance between the interannual
  near-surface temperature at a target site and the spatial fields of
  interannual temperature ($T_{2\mathrm{m}}$, black), oxygen isotope
  ($\delta^{18}\mathrm{O}$, green) and precipitation-weighted oxygen isotope
  composition ($\delta^{18}\mathrm{O}^{\mathrm{(pw)}}$, blue). Averaging is
  performed in two steps: (1) For a given target site, the correlations between
  target site temperature and the respective fields are averaged across all grid
  cells that fall into consecutive rings of $250$\,km radial width around the
  target site (see Fig.~\ref{fig:concept} for a conceptual sketch of this
  approach). (2) We average these results across all target sites that lie in
  the DML region, i.e. $\pm17.5\degree$ longitude and $\pm5\degree$ latitude
  around the EDML drilling site ($-75\degree$\,S, $0\degree$\,E).}
\label{fig:avg.decor.lengths}%
\end{figure}%

\subsection{Picking optimal ice-core sites for temperature reconstructions}
\label{results:picking}

The above analyses suggest that isotope records of single ice cores only
capture a low portion of the local interannual temperature variability, since
additional processes that influence the isotopic signal lower the correlation
with the local temperature record. Interpreting these processes as noise raises
the question of whether the correlation with temperature can be improved by
averaging isotope records across sites. Assuming an ideal world, in which the
temperature data of the climate model are a perfect surrogate for the true
climate variations at each site, we can set up the simple experiment of
investigating which spatial array of $N$ ice cores, i.e. oxygen isotope records,
would optimise the temperature correlation with a target site by
random picking and averaging of $\delta^{18}\mathrm{O}^{\mathrm{(pw)}}$ grid
cells.

For the EDML drilling site as a target, we obtain the interesting result that
the optimal site for a single core is not the local grid cell, as one might
expect, but a site $\sim960$\,km away from the target towards the southeast
(Fig.~\ref{fig:picking}a). Choosing this site increases the correlation with the
target temperature from the local value of $0.26$ to a value of
$0.43$. Optimising the set of three and five cores
(Fig.~\ref{fig:picking}b--c) yields in both cases sites that scatter at
significant distances around the target, but with only slightly higher
correlations than for $N=1$ ($0.46$ for $N=3$ and $0.47$ for $N=5$). We obtain
similar results for choosing the drilling site of the Vostok ice core
($-78.47\degree$\,S, $106.83\degree$\,E) as a target
(Fig.~\ref{fig:picking}d--f). The optimal single core (correlation of $0.45$
compared to the local correlation of $0.34$) would be at a site $\sim300$\,km
north of Vostok. As for EDML, the optimal sites for three and five cores all lie
around the target without including it, but here the optimal three cores
(correlation of $0.56$) yield a significant increase in correlation compared to
the optimal single core, while the optimal five cores also show no further
increase (correlation $0.57$).

%f04
\begin{figure*}[t]%
\centering
\includegraphics[width=16cm]{../plots/main/fig_04.pdf}
\caption[Picking optimal sites]{%
  Picking ice core sites that optimally reconstruct interannual temperatures at
  EDML and Vostok. The maps show the correlation in the model data between the
  interannual temperature time series at the target sites (black crosses) EDML
  (\textbf{a}--\textbf{c}) and Vostok (\textbf{d}--\textbf{f}) with the fields
  of precipitation-weighted oxygen isotope composition. Filled black circles
  denote those grid cells that maximise the correlation with the target site
  temperature for choosing either a single grid cell ($N=1$; \textbf{a},
  \textbf{d}) or for averaging across $N=3$ (\textbf{b}, \textbf{e}) or $N=5$
  (\textbf{c}, \textbf{f}) grid cells.}
\label{fig:picking}%
\end{figure*}%

We generalise these findings by investigating how often the optimal correlation
with the target site is, for a single core, obtained from a non-local grid cell,
using all Antarctic model grid cells as target sites. Indeed, the majority
($\sim65$\,\%) of optimal sites lie at distances from the respective target site
of between $400$--$600$ and $800$--$1000$\,km.

% %f05
% \begin{figure}[t]%
% \centering
% \includegraphics[width=6.5cm]{../plots/main/fig_05.pdf}
% \caption{%
%   The distribution of distances between Antarctic target sites and the
%   respective optimal ice-core location. The optimal location is defined as that
%   grid cell of $\delta^{18}\mathrm{O}^{\mathrm{(pw)}}$ time series that
%   maximises the correlation with the target site temperature time series.}
% \label{fig:picking.hist}%
% \end{figure}%

\subsection{Optimal ice-core arrangements}
\label{results:optim-spacing}

The approach of picking optimal ice-core sites yields straightforward results,
but it also has several shortcomings. Firstly, the results will likely depend on
the used model simulation and the analysed time span and are therefore difficult
to generalise. Secondly, the variability between results for different target
sites and for averaging $N\ge2$ cores complicates the inference of typical
spatial scales that govern the correlation structures. As a next step, we thus
adapt our approach in order to overcome these issues and to learn in general
about the optimal arrangement of ice cores which maximises the correlation with
temperature. For this, we average across grid cells to reduce local variability
in the model data and performe this averaging step across concentric rings
around a target site (Fig.~\ref{fig:concept} and Sect.~\ref{methods:general}) in
order to avoid a preferred direction.

For a single core, the expected correlation with the target site temperature in
the DML region is readily obtained from Fig.~(\ref{fig:avg.decor.lengths}). For
the $T_{\mathrm{2m}}$ field itself, we find the maximum correlation with the
target site for sampling from the innermost ring, as expected. However, for
$\delta^{18}\mathrm{O}^{\mathrm{(pw)}}$ (and $\delta^{18}\mathrm{O}$) the decay
with distance of the expected correlation is much weaker, and, while the highest
correlation is also found for the innermost ring, nearly equally high
correlations are obtained for rings till a distance of $\lesssim1000$\,km.

%f05
\begin{figure*}[t]%
\centering
\includegraphics[width=16cm]{../plots/main/echam5_mpiom_wiso_two_core_correlation.png}
\caption{%
  The expected correlation with the target site temperature in the DML region as
  a function of distance for averaging two cores. Shown is the mean correlation
  across all possible single correlations from averaging two $T_{\mathrm{2m}}$
  (\textbf{a}), $T_{\mathrm{2m}}^{\mathrm{(pw)}}$ (\textbf{b}) and
  $\delta^{18}\mathrm{O}^{\mathrm{(pw)}}$ (\textbf{c}) time series of grid cells
  sampled from the same or from two different ring bins. The axes display the
  distance from the target of the inner radius of the sampled ring bin, where
  the $x$ ($y$) axis stands for the first (second) sampled ring. \#ADD labels to
  the figure: (a) is left, (b) middle, (c) right.}
\label{fig:two-core-correlation}%
\end{figure*}%

The difference between the fields of $T_{\mathrm{2m}}$ and
$\delta^{18}\mathrm{O}^{\mathrm{(pw)}}$ is even more pronounced when averaging
two cores. As one would expect, the maximum expected correlation for
$T_{\mathrm{2m}}$ is still found when sampling only from the innermost ring
(Fig.~\ref{fig:two-core-correlation}a). However, for
$\delta^{18}\mathrm{O}^{\mathrm{(pw)}}$ the optimal arrangement of two cores is
obtained for sampling one core from the innermost ring but the second one from
the fourth or fifth ring, i.e. between $\sim800$ and $1200$\,km away from the
target (Fig.~\ref{fig:two-core-correlation}c). Part of this structure is related
to the effect of precipitation weighting, as can be seen from the correlation
structure for sampling the $T_{\mathrm{2m}}^{\mathrm{(pw)}}$ field
(Fig.~\ref{fig:two-core-correlation}b). Here, the correlation is nearly as high
when the cores are sampled from the innermost and from one ring further away as
for sampling only from the innermost ring, which is not the case for the
$T_{\mathrm{2m}}$ field.

We obtain similar results for averaging $N=3$ and $N=5$
$\delta^{18}\mathrm{O}^{\mathrm{(pw)}}$ cores to reconstruct the EDML target
site temperature (Fig.~\ref{fig:binning}). Intriguingly, the optimal cores are
always arranged such that one core is placed in the innermost ring while the
others are distributed at distances between $\sim500$ and $1500$\,km from the
target.

\noindent
\#EDIT: Unclear whether to include the Vostok binning results here (see current
figure). If so, then I have the impression that we need to present Vostok
results also for the two previous figures.

\noindent
\#EDIT: What is missing:
\begin{itemize}
\item increase in correlation with number of averaged cores (available with the
  final runs of the ``binning'' results).
\item risk analysis: local correlation versus distribution of correlations from
  the optimal set of ring bins (for $N=3$ or 5 or...?). To be put either here in
  results or as part of the discussion.
\end{itemize}

%f06
\begin{figure*}[t]%
\centering
\includegraphics[width=16cm]{../plots/main/echam5_mpiom_wiso_binning.pdf}
\caption{%
  The optimal arrangement of averaging three or five
  $\delta^{18}\mathrm{O}^{\mathrm{(pw)}}$ cores to reconstruct the target site
  temperature at EDML (\textbf{a}, \textbf{c}) and Vostok (\textbf{b},
  \textbf{d}). Displayed are the optimal five of all possible combinations of
  ring bins, i.e. those which exhibit the highest mean correlation across $1000$
  (\#EDIT: to be increased) random trials of averaging $N=3$ (\textbf{c},
  \textbf{d}) or $N=5$ (\textbf{a}, \textbf{b}) grid cells from these
  rings. \#EDIT: the figure could also be done for the average across target
  sites in the \emph{regions} DML and Vostok.}
\label{fig:binning}%
\end{figure*}%

\section{Discussion}\label{discussion}

(\#ADD: compare local t2m-oxy.pw correlation with previous studies,
e.g. Goursaud2018/ECHAM5-wiso with ERA-Interim?)
























%\copyrightstatement{}

% \codedataavailability{TEXT} %% use this section when having data sets and software code available

% \appendix
% \section{}    %% Appendix A
% \subsection{}     %% Appendix A1, A2, etc.

% \noappendix       %% use this to mark the end of the appendix section

% \authorcontribution{TEXT} %% this section is mandatory
% \competinginterests{TEXT} %% this section is mandatory even if you declare that no competing interests are present
% \disclaimer{TEXT} %% optional section

% \begin{acknowledgements}
% \end{acknowledgements}

% \bibliographystyle{copernicus}
% \bibliography{muench-etal_optimalcores}

\end{document}
