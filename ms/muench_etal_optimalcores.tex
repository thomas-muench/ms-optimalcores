\documentclass[cp, manuscript]{copernicus}

\begin{document}

\title{[working title] Optimal ice-core arrangement}

\Author[1]{Thomas}{M\"{u}nch}
\Author[2]{Martin}{Werner}
\Author[1,3]{Thomas}{Laepple}

\affil[1]{Alfred-Wegener-Institut Helmholtz-Zentrum f{\"u}r Polar- und
Meeresforschung, Research Unit Potsdam, Telegrafenberg A45, 14473 Potsdam,
Germany}
\affil[2]{Alfred-Wegener-Institut Helmholtz-Zentrum f{\"u}r Polar- und
Meeresforschung, Bussestra{\ss}e 24, 27570 Bremerhaven, Germany}
\affil[3]{University of Bremen, MARUM~--~Center for Marine Environmental
  Sciences and Faculty of Geosciences, 28334 Bremen, Germany}

\correspondence{Thomas M\"unch (thomas.muench@awi.de)}
\runningtitle{Optimal ice-core arrangement}
\runningauthor{T. M\"unch et al.}

\received{}
\pubdiscuss{}
\revised{}
\accepted{}
\published{}

\firstpage{1}

\maketitle

\begin{abstract} Many proxies used in climate research share one complicating
property: they are not only driven by the climatic target variable of interest,
e.g. temperature, but also influenced by secondary effects which cause
additional variability, frequently termed \emph{noise}. Noise in individual
proxy records can be reduced generally by averaging them together, but the
effectiveness of this approach depends on the spatial correlation scales of the
involved noise-generating processes. Here, we review this concept in the context
of Antarctic ice cores and apply it in order to optimise local to regional scale
temperature reconstructions from ice-core isotope records. We use data from
climate model simulations to identify an optimal arrangement of ice-core arrays
which maximises the signal-to-noise ratio in the reconstruction with minimum
sampling effort. A first intriguing result is that, in order to optimise
temperature reconstructions at a certain target site, it is necessary to combine
local with more distant ice-core records. More specifically, for a target site
in Dronning Maud Land on the East Antarctic Plateau (\#EDIT generalise to EAP?)
our results suggest to use both local records and ice cores separated
$\sim500-1000$\,km away from the target. We show that these findings can be
largely explained by the interplay of the two spatial scales associated with the
correlation structure of the temperature field and of the noise generated by
precipitation intermittency. In summary, our study helps to maximise the
usability of existing ice cores as well as to optimally plan future drilling
campaigns. It also deepens our knowledge concerning the typical correlation
scales of the different processes that shape the isotopic record. Finally, the
presented method can be direcly extended to other palaeoclimate reconstruction
problems.
\end{abstract}

% \copyrightstatement{TEXT}

\introduction

The oxygen and hydrogen isotopic composition of firn and ice recovered from
polar ice cores is a key proxy for past near-surface air temperature changes
(\#REF). Although the physical mechanisms that link local changes in temperature
to the isotopic composition of precipitated snow are generally well understood
(\#REF) and can be modelled with general circulation models (\#REF), the
quantitative interpretation of ice-core isotope variabily in terms of
temperature variability is complicated by second order processes that
additionally influence the isotopic record, creating noise (\#REF). Analysing
the typical spatial scales on which the different processes act helps to
minimise variability, which is not related to temperature, by averaging ice-core
records from an array of sites.

The isotopic record that is derived from an ice core is the result of a chain of
processes from (1) atmospheric temperature changes to (2) the isotopic
fractionation during the pathway of the atmospheric moisture, (3) the effect of
variable and intermittent precipitation and finally (4) post-depositional
effects; each element of this chain can be associated with a typical spatial
scale.

Atmospheric temperature variations drive the fractionation of the isotopic
composition of the atmospheric moisture along its pathway to the final stage of
precipitation (\#REF). The spatial coherence of the temperature-related isotopic
signal in precipitation is hence given by the spatial coherence of the
atmospheric temperature field itself. Typical spatial decorrelation scales for
temperature are of the order of $\gtrsim1000$\,km (\#REF), which implies that
ice cores distributed on spatial scales below $\sim 1000$\,km should record a
similar, i.e. correlated, temperature signal. However, the temporal variability
of the isotopic composition in the local atmospheric moisture also depends on
the variability of the atmospheric circulation, since different air masses may
exhibit different source regions and distillation pathways (\#REF). In addition,
in an ice core, the isotopic composition in a deposited layer of snow will not
one-to-one reflect the temporal variability of the atmospheric isotope signal
due to the intermittent nature of precipitation (\#REF), which weights the
initial isotope signal with the amount of precipitation, thereby introducing
bias and adding additional variability to the isotopic record (\#REF). Both
processes are directly linked to the atmospheric dynamics, and their typical
scales are hence expected to range from mesoscales (i.e. tens of kilometres),
driven by topography and orographic effects, to synoptic scales of hundreds of
kilometres associated with cyclonic activity and the movement of high and low
presssure systems (\#REF). Finally, in polar conditions the precipitated snow
does not directly settle but is constantly eroded, blown away and
redeposited. It has been shown that this gives rise to stratigraphic noise in
the isotopic record (\#REF:fisher) which exhibits a small-scale decorrelation of
a few metres (\#REF:trench).%
\footnote{We note that the final isotopic record is influenced by further
  post-depositional processes within the snow and ice matrix, such as
  densification and diffusion, which are, however, not within the scope of this
  article.}

This hierarchy of process scales~--~from temperature to the atmospheric isotopic
composition, to the precipitation-weighted isotopic composition, and finally to
post-depositional effects~--, determines the effectiveness of reducing overall
noise by averaging isotope records, since the reduction of the noise level will
depend on the spatial correlation scale of the different noise sources. For
example, if we only considered temperature and stratigraphic noise, it would be
sufficient to average records spaced just some tens of metres apart, as this
would ensure highly correlated temperature signals but uncorrelated
stratigraphic noise between the records. However, the comparison of
correlation-based signal-to-noise ratios of nearby isotope records (\#REF) with
those estimated from analysing the record's temporal variability (\#REF) has
shown that reproducibility on a local scale does not necessarily imply a
climatic, i.e. temperature-driven, origin. Instead, circulation variability and
precipitation intermittency act as further noise sources which are expected to
exhibit larger decorrelation lengths (\#REF:cp18,cycles) than the stratigraphic
noise. Taking this into account, we expect some optimal length scale in between
the local and the temperature decorrelation scale which is a trade-off between
averaging out circulation and intermittency effects and ensuring a sufficient
coherence of the recorded temperature signal.

Here, we use data from a climate model equipped with stable isotope diagnostics
to learn about the different process scales and to link this to the optimal
arrangement of ice-core locations, with a focus on the East Antarctic Plateau
(EPA). For a target location in Dronning Maud Land (\#EDIT), our results suggest
a combination of local records with ice cores separated $\sim500-1000$\,km away
from the target in order to maximise the correlation with the target temperature
signal. This result also holds when assessing the range of correlations for an
arbitrary sampling of such combinations (\#riskanalysis). In general, optimal
sampling can increase the correlation with the target temperature by x-y\,\%
when averaging x-y cores as compared to a local reconstruction using only a
single record (\#EDIT).

\section{Data and methods}\label{methods}

\subsection{Climate model data}\label{methods:data}

We use data of the past-millennium simulation (800--2000\,CE; \#REF:Sjolte2018)
of the fully coupled ECHAM5/MPI-OM-wiso atmosphere--ocean general circulation
model equipped with stable isotope diagnostics (\#REF:Werner2016). This
simulation is forced by greenhouse gases, volcanic aerosols, total solar
irradiance and changes in land use and Earth's orbital parameters. The
atmospheric component of the model has a T31 spectral resolution
($3.75\degree\times3.75\degree$) with $19$ vertical levels
(\#REF:Sjolte2018). Compared to observations, the atmospheric model ECHAM5-wiso
generally reproduces the climatological relation between temperature and
precipitation isotopic composition well, but has, in the used T31 setup, a warm
bias and is not depleted enough in isotopic composition over Antarctica
(\#REF:Werner2011). This can be improved by a higher spatial resolution
(\#REF:Werner2011) but is not relevant for our study since we are mainly
interested in the relative variability between sites. The full atmosphere--ocean
model was evaluated against observations for equilibrium simulations under
pre-industrial and Last Glacial Maximum conditions (\#REF:Werner2016), and the
past-millennium simulation was used to reconstruct North Atlantic atmospheric
circulation in combination with ice-core isotope data (\#REF:Sjolte2018).

Here, we use the ECHAM5/MPI-OM-wiso time series for all model grid cells
on the Antarctic continent ($N_{\mathrm{grid}}=442$) of two-metre air
temperature ($T_{2\mathrm{m}}$), precipitation ($p$) and oxygen isotopic
composition in precipitation (relative abundance of oxygen-18 to oxygen-16
istopes, denoted as $\delta^{18}\mathrm{O}$).

\subsection{Data processing}\label{methods:prc}

The model simulation output has monthly temporal resolution, while ice-core
isotope records are typically analysed on annual or longer resolution achieved
by averaging the isotopic data across the respective amount of snow and
ice. The annual data of isotopic composition will then include a weighting
effect due to the intra-annual variability in the amount of precipitation. To
account for this, we produce two versions of annual data from the monthly model
output: (1) the two-metre temperature and oxygen isotopic composition are simply
averaged to annual resolution without any weighting ($T_{2\mathrm{m}}$ and
$\delta^{18}\mathrm{O}$ in the following), (2) the respective monthly data are
averaged to annual resolution including a weighting by the monthly amount of
precipitation (denoted as precipitation-weighted data
$T_{2\mathrm{m}}^{\mathrm{(pw)}}$ and $\delta^{18}\mathrm{O}^{\mathrm{(pw)}}$ in
the following).

\subsection{Data analyses}\label{methods:main}

\subsubsection{General approach}\label{methods:general}

We investigate the relationship between the various model variables by assessing
the Pearson correlation coefficient. To link model data to ice cores, we mimic
ice-core isotope records by associating them with the
$\delta^{18}\mathrm{O}^{\mathrm{(pw)}}$ time series at the model grid cells. We
thus neglect stratigraphic noise and any post-depositional effects on the
isotopic record since here we are interested in the upper limit of the extent to
which ice cores can reconstruct the climatic temperature signal. In order to
learn about the typical spatial scales that govern the temperature--isotope
relationship, we set up the following general scheme: For a given model grid
cell $\mathbf{R}_0$ of interest (target site in the following), we define
consecutive rings around this site of $250$\,km radial width until a maximum
distance of $2000$\,km (Fig.~\ref{fig:concept}). Then, we determine all grid
cells that fall into each of these rings and either sample these grid cells
directly or sample from the possibilities of combining model grid cells and
rings.

%f01
\begin{figure}[t]%
\centering
\includegraphics[width=6.5cm]{../plots/main/fig_01.pdf}
\caption[Conceptual approach]{%
  Conceptual sketch of the general approach. Around a given Antarctic target
  site (black cross), we define consecutive rings (red lines) of $250$\,km
  radial width and analyse all model grid cells that fall into each of the
  rings.}
\label{fig:concept}%
\end{figure}%

% \subsubsection{Estimation of decorrelation lengths and spatial correlation
%   structures}\label{methods:decor.model}

% We estimate spatial temperature decorrelation lengths by fitting an isotropic
% exponential model to the correlation--distance dependence of the form
% \begin{equation}\label{eq:decor.model}
% c(r) = \exp{\left(-\frac{r}{\tau}\right)},
% \end{equation}
% where $c(r)$ is the fitted correlation at some distance $r$ from the target, and
% $\tau$ is the estimated decorrelation length at which the correlation has
% declined to $1/e$ ($\sim37\,\%$). The input relationship between correlation and
% distance is obtained in two different ways: (1) For a single target site, we
% correlate the $T_{2\mathrm{m}}$ time series at the target site with the
% $T_{2\mathrm{m}}$ time series of every other grid cell on the Antarctic
% continent and record the respective distances between the sites. (2) To reduce
% the uncertainty of these individual correlation estimates, we average across the
% correlations obtained between the $T_{2\mathrm{m}}$ time series at a target site
% and the $T_{2\mathrm{m}}$ time series of grid cells sampled from rings
% (Fig.~\ref{fig:concept}). In a second averaging step, we compute the mean of the
% so obtained correlation--distance relationships for several target sites from a
% given Antarctic region and use that as input to Eq.~\eqref{eq:decor.model}.

% The spatial correlation structure between the target site temperature and other
% model variables is assessed in the same manner as for the average temperature
% decorrelation.

\subsubsection{Picking optimal sites}\label{methods:picking}

To obtain an optimal set of ice cores to reconstruct $T_{2\mathrm{m}}$ at a
given target site, we randomly pick without replacement $N$ of the
grid cells that lie within a circle of $1000$\,km radius around the target site
and correlate the average $\delta^{18}\mathrm{O}^{\mathrm{(pw)}}$ time series of
these $N$ grid cells with the target site temperature. The optimal set of cores
for each $N$ is then determined from the maximum correlation across all picking
trials. To ensure stable results, we set the maximum number of trials to
$n=5000$ for $N=1$ and to $n=10^5$ for $N\geq3$.

\subsubsection{Optimal sampling length scales}\label{methods:opt.sampling}

The optimal sites determined from the picking approach described above will
strongly depend on the used model simulation and do not offer insights into the
governing spatial scales. To overcome these issues, we generalise the approach
by using all climate model variables and by sampling $N$ sites from the defined
rings around a given target site (Fig.~\ref{fig:concept}). This is implemented
as a two-step process: (1) we determine all possible combinations of selecting
$N$ rings with replacement, and then, (2), identify for each ring combination
the possibilities of combining a single grid cell from each of the rings. For
each of these grid-cell combinations, we average the time series of the studied
variable and compute the correlation of this spatial average with the target
site temperature. Finally, we report the expected correlation for every ring
combination by averaging across all correlations of the analysed grid-cell
combinations. This gives insights into the average spatial correlation structure
depending on the number of averaged sites and the sampling length scale between
these sites.

To identify all possible grid-cell combinations in the second step above is
computationally feasible until $N=2$; for $N\geq3$, we resort to Monte Carlo
sampling instead. For this, we estimate a sufficient number of Monte Carlo
iterations from comparing the Monte Carlo sampling solution with the exact
solution for $N=2$. We find the correlation mismatch (root mean square
deviation) to be $\sim10^{-3}$ for $10^4$ iterations. Since a higher number of
possible ring combinations than available for $N=2$ involves many more possible
grid-cell combinations, we choose $10^5$ iterations for sampling $N\geq3$ sites.

\subsection{Selected study regions}\label{methods:regions}

We focus our analyses mainly on two subregions of the East Antarctic Plateau
which both cover existing deep ice-core drilling sites as well as large arrays
of shallower ice and firn cores.

For the first region, the Dronning Maud Land (DML) region in the following, we
choose all model grid cells ($N_{\mathrm{grid}}=26$) within a range of
$\pm17.5\degree$ longitude and $\pm5\degree$ latitude around the EPICA Dronning
Maud Land site (EDML; $-75\degree$\,S, $0\degree$\,E). This region covers the
locations of the deep EDML ice core (\#EDML) and of $>50$ firn and shallow ice
cores (\#ALTNAU). For the second region, referred to as the Vostok region, we
choose an identical latitudinal and longitudinal coverage
($N_{\mathrm{grid}}=30$) with respect to the Vostok station ($-78.47\degree$\,S,
$106.83\degree$\,E), covering the locations of the deep Vostok and Dome C ice
cores as well as of several shallower cores (\#STENNI), and including the new
deep drilling site for retrieving an ice core reaching back more than one
million years.

\section{Results}\label{results}

\subsection{Temperature decorrelation and temperature--isotope relationship}
\label{results:t2m-iso}

We first assess the extent to which a local ice-core record, i.e. the
interannual isotope time series of a single grid cell in the model simulation,
is representative for the interannual variability of near-surface atmospheric
temperatures.

The modelled temperature field over Antarctica exhibits generally large-scale
coherent variations (Fig.~\ref{fig:correlation.maps}a); however, with a clear
two-part structure roughly divided by the range of the Transantarctic Mountains:
For most regions of the East Antarctic Plateau, the temperature field has
typical decorrelation lengths of $\sim1500$ to $2500$\,km, while the
decorrelation lengths are significantly lower with values $\lesssim1000$\,km for
larger parts of the West Antarctic Ice Sheet and for the Antarctic Peninsula.
Hence, assuming perfect ice cores with an ideal temperature proxy only governed
by local temperature variations, a single ice core would still be enough in both
regions to capture the temperature variability across hundreds of kilometres.
% In order to reduce the uncertainty associated with the
% individual correlation estimates from single grid cells we analyse the expected
% correlation as a function of distance averaged across grid cells and across a
% region of target sites (Fig.~\ref{fig:avg.cor.structure}). Here, as an example
% for an existent ice-core drilling region, we choose as target sites all grid
% cells within a range of $\pm17.5\degree$ longitude and $\pm5\degree$ from the
% EPICA Dronning Maud Land (EDML) site ($-75\degree$\,S, $0\degree$\,E), referred
% to as the Dronning Maud Land (DML) region in the following. We find the
% temperature decorrelation in this region to clearly follow an exponential decay
% with a length scale of $\sim1900$\,km. 

%f02
\begin{figure*}[t]%
\centering
\includegraphics[width=16cm]{../plots/main/fig_02.png}
\caption{%
  Temperature decorrelation lengths and temperature--isotope
  relationship. (\textbf{a}) The temperature decorrelation lengths ($\tau$, in
  km) for each Antarctic model grid cell estimated by fitting an exponential
  model to the correlation--distance relationship (cf. Eq.~\ref{eq:t2m.decorr})
  obtained from correlating the local interannual near-surface $T_{2\mathrm{m}}$
  time series with the respective temperature time series from all other grid
  cells. Note that only continental grid cells are used for the fit.
  (\textbf{b}) The local correlation between the interannual near-surface
  temperature ($T_{2\mathrm{m}}$) and precipitation-weighted oxygen isotope
  composition ($\delta^{18}\mathrm{O}^{\mathrm{(pw)}}$) time series for each
  Antarctic model grid cell.}
\label{fig:correlation.maps}%
\end{figure*}%

However, actual single Antarctic ice-core isotope records only explain~--~as
simulated by the climate model~-- a low portion of the local temperature
variations and thus of the regional temperature field. This is seen from
analysing the modelled interannual precipitation-weighted field of
$\delta^{18}\mathrm{O}^{\mathrm{(pw)}}$, which most closely mimicks a real
ice-core record. From this we find the local correlation between the
$T_{2\mathrm{m}}$ and $\delta^{18}\mathrm{O}^{\mathrm{(pw)}}$ time series for
each model grid cell (Fig.~\ref{fig:correlation.maps}b) to be generally low,
ranging across all analysed grid cells from $<0.1$ up to $\sim0.53$ with
$\sim75\,\%$ of the correlations $\leq0.4$ and a mean correlation of $0.36$.

In the following, we will assess the extent to which the correlation with
temperature can be increased and how this is related to the spatial scales
studied.
% In the second step, we determine the average spatial
% strucure of the temperature--isotope correlation as a function of distance
% (Fig.~\ref{fig:avg.cor.structure}), assessed for the DML region in the same
% manner as the average temperature decorrelation structure presented above. In
% contrast to the exponential temperature decorrelation, the
% $\delta^{18}\mathrm{O}$ field in the DML region exhibits a much lower but also
% more stable structure of correlation with the target site temperature. Starting
% from a local value of $\sim0.4$, the correlations decrease only slightly with
% increasing distance up to $\sim1300$\,km, followed by a little steeper decrease
% for larger distances and constant levels of $\lesssim0.2$ for distances above
% $\sim1700$\,km. Precipitation weighting, as seen from analysing the
% $\delta^{18}\mathrm{O}^{\mathrm{(pw)}}$ field, results in overall even lower
% correlation values, but it does not affect much the correlation structure
% itself.

\subsection{Picking optimal ice-core sites for temperature reconstructions}
\label{results:picking}

The above analysis shows that isotope records of single ice cores only capture a
low portion of the local interannual temperature variability, suggesting that
additional processes influence the isotopic signal which lower the correlation
with the local temperature record. Interpreting these processes as noise raises
the question of whether the correlation with temperature can be improved by
averaging isotope records across sites. Here, we first assume an ideal world, in
which the climate model data are a perfect surrogate for the true climate and
proxy variations at each site. Doing so we can set up the simple experiment of
random picking and averaging of $\delta^{18}\mathrm{O}^{\mathrm{(pw)}}$ grid
cells to find that spatial array of $N$ ice cores, which optimises the
temperature correlation with a target site.

For our specific model simulation and the EDML drilling site as a target, we
obtain the interesting result that the optimal site for a single core is not the
local grid cell -- as one might expect -- but a site $\sim960$\,km away from the
target towards the southeast (Fig.~\ref{fig:picking}a). Choosing this site
increases the correlation with the target temperature from the local value of
$0.26$ to a value of $0.43$. Optimising the set of three and five cores
(Fig.~\ref{fig:picking}b--c) yields sites that in both cases scatter at
significant distances around the target, but with only slightly higher
correlations than for $N=1$ ($0.46$ for $N=3$ and $0.47$ for $N=5$). We obtain
similar results for choosing the Vostok drilling site as a target
(Fig.~\ref{fig:picking}d--f). The optimal single core (correlation of $0.45$
compared to the local correlation of $0.34$) would be in this model simulation
at a site $\sim300$\,km north of Vostok. As for EDML, the optimal sites for
three and five cores all lie around the target without including it, but here,
the optimal three cores (correlation of $0.56$) yield a significant increase in
correlation compared to the optimal single core, while the optimal five cores
likewise show no further increase (correlation $0.57$).

%f03
\begin{figure*}[t]%
\centering
\includegraphics[width=16cm]{../plots/main/fig_03.pdf}
\caption[Picking optimal sites]{%
  Picking ice core sites that optimally reconstruct interannual temperatures at
  the EDML and Vostok drilling sites. The maps show the correlation in the model
  data between the interannual temperature time series at the target sites
  (black crosses) EDML (\textbf{a}--\textbf{c}) and Vostok
  (\textbf{d}--\textbf{f}) with the fields of precipitation-weighted oxygen
  isotope composition. Filled black circles denote those grid cells that
  maximise the correlation with the target site temperature for choosing either
  a single grid cell ($N=1$; \textbf{a}, \textbf{d}) or for averaging across
  $N=3$ (\textbf{b}, \textbf{e}) or $N=5$ (\textbf{c}, \textbf{f}) grid cells.}
\label{fig:picking}%
\end{figure*}%

We generalise these findings by investigating for all Antarctic model grid cells
as target sites how often the optimal correlation for a single core is given by
a non-local grid cell. Indeed, the majority ($\sim65$\,\%) of optimal sites lie
at distances from the respective target site of between $400$--$600$ and
$800$--$1000$\,km.

\subsection{Optimal ice-core arrangements}
\label{results:optim-spacing}

The approach of picking optimal ice-core sites yields straightforward and
instructive results, but it also has several shortcomings. Firstly, the results
will likely depend on the used model simulation and are therefore difficult to
generalise. Secondly, the variability between results for different target sites
and for averaging $N\ge2$ cores complicates the inference of typical spatial
scales that govern the correlation structures. Thus, as a next step, we adapt
our approach in order to overcome these issues and to learn about the general
optimal spatial arrangement of ice cores maximising the correlation with
temperature. For this, we compute the mean of correlation results obtained
between a target site temperature and individual grid cells in order to reduce
local variability in the model data. We perform this averaging step across
combinations of $250$\,km wide concentric rings around a target site
(Fig.~\ref{fig:concept} and Sect.~\ref{methods:opt.sampling}) in order to avoid
preferring a direction. Additionally, if applicable, we average so obtained
results across the target sites which lie within our defined DML and Vostok
regions (Sect.~\ref{methods:regions}) to obtain regional estimates. Finally, we
analyse each of the model variables to highlight the differences between the
individual fields.

%f04
\begin{figure*}[t]%
\centering
\includegraphics[width=12cm]{../plots/main/fig_04.pdf}
\caption{%
  Spatial temperature correlation structures for the DML and Vostok regions.
  Shown is the average correlation as a function of distance between the
  interannual near-surface temperature ($T_{2\mathrm{m}}$) and the spatial
  fields of $T_{2\mathrm{m}}$ (black), oxygen isotope ($\delta^{18}\mathrm{O}$,
  green) and precipitation-weighted oxygen isotope composition
  ($\delta^{18}\mathrm{O}^{\mathrm{(pw)}}$, blue). Averaging is performed in two
  steps: first, correlations are averaged across grid cells falling in $250$\,km
  wide consecutive rings around a given target site, and secondly, these results
  are averaged across all respective target sites in the DML (\textbf{a}) and
  Vostok (\textbf{b}) region (see Methods). Dashed lines indicate an exponential
  fit to the $T_{2\mathrm{m}}$ data.}
\label{fig:avg.cor.structure}%
\end{figure*}%

For sampling a single site from the $T_{\mathrm{2m}}$ field in the DML and
Vostok region, we find the average spatial correlation structure to clearly
follow an exponential decay (Fig.~\ref{fig:avg.cor.structure}) with length
scales of $\sim1900$\,km in both cases, as already indicated by the results on
the local scale (Fig.~\ref{fig:correlation.maps}a). These results also imply
that the maximum expected correlation with the target site temperature
necessarily is obtained for sampling from the innermost ring only.

In contrast to the exponential temperature decorrelation, the
$\delta^{18}\mathrm{O}$ field in the DML region exhibits a much lower but also
more stable structure of correlation with the target site temperature
(Fig.~\ref{fig:avg.cor.structure}a). Starting from a local value of $\sim0.4$,
the correlations decrease only slightly with increasing distance up to
$\sim1300$\,km, yielding the highest correlation to also be found for sampling
from the innermost ring, but with nearly equally high correlations obtained for
rings till a distance of $\lesssim1000$\,km. This slight decrease is then
followed by a little steeper decrease for larger distances and constant levels
of $\lesssim0.2$ for distances above $\sim1700$\,km. For the Vostok region
(Fig.~\ref{fig:avg.cor.structure}b), the correlation structure of
$\delta^{18}\mathrm{O}$ exhibits a nearly constant linear decrease from an
initial value of $\gtrsim0.5$ to $\sim0.1$ in the final ring ($>2000$\,km).
Precipitation weighting induces, as seen from analysing the
$\delta^{18}\mathrm{O}^{\mathrm{(pw)}}$ fields, overall even lower correlation
values in both regions, but it does not affect much the spatial correlation
structure itself.

%f05
\begin{figure*}[t]%
\centering
\includegraphics[width=17cm]{../plots/main/fig_05.png}
\caption{%
  The expected correlation with the target site temperature for the average of
  two sites in the DML region. Shown is the mean correlation of all possible
  single correlations from averaging two grid cells of (\textbf{a})
  $T_{\mathrm{2m}}$, (\textbf{b}) $T_{\mathrm{2m}}^{\mathrm{(pw)}}$ and
  (\textbf{c}) $\delta^{18}\mathrm{O}^{\mathrm{(pw)}}$ time series sampled from
  the same or from two different rings, averaged over all target sites in
  the given region. The axes display the distance from the target, where the $x$
  ($y$) axis stands for the first (second) sampled ring and tickmarks
  indicate the radius of the midpoints of the rings.}
\label{fig:two-core-correlation}%
\end{figure*}%

The difference in correlation structure between the fields of $T_{\mathrm{2m}}$
and $\delta^{18}\mathrm{O}^{\mathrm{(pw)}}$ is even more pronounced when
averaging two sites. Here, we can assess either the possibility of sampling
sites from the same ring or from two different ones. As one would expect, the
maximum expected correlation for $T_{\mathrm{2m}}$ is still found when sampling
both sites from the innermost ring, as shown for the DML region
(Fig.~\ref{fig:two-core-correlation}a). However, for
$\delta^{18}\mathrm{O}^{\mathrm{(pw)}}$ the optimal arrangement of two sites is
obtained for sampling one site from the innermost ring but the second one from
the fifth ring, i.e. between $\sim1000$ and $1250$\,km from the target site
(Fig.~\ref{fig:two-core-correlation}c). Part of this structure is related to the
effect of precipitation intermittency, which can be seen from the correlation
structure for sampling the $T_{\mathrm{2m}}^{\mathrm{(pw)}}$ field
(Fig.~\ref{fig:two-core-correlation}b). Here, in contrast to $T_{\mathrm{2m}}$,
the correlation is about as high when we combine the innermost ring and one ring
further away, as when we sample both sites from the innermost ring.

The same effect of precipitation intermittency is also evident in the results
for averaging two sites from the Vostok study region
(Fig.~\ref{fig:two-core-correlation-vostok}), with a similar difference in
correlation structures between $T_{\mathrm{2m}}$ and
$T_{\mathrm{2m}}^{\mathrm{(pw)}}$ as for the DML region, and a similar structure
between $T_{\mathrm{2m}}^{\mathrm{(pw)}}$ and
$\delta^{18}\mathrm{O}^{\mathrm{(pw)}}$ for distances $\lesssim1000$\,km.
However, the results for the $\delta^{18}\mathrm{O}^{\mathrm{(pw)}}$ field
(Fig.~\ref{fig:two-core-correlation-vostok}c) do not display such a pronounced
maximum when sampling one site from the innermost ring and the second one from a
ring further away as it is observed for the DML region, which points at an
influence of the regional differences in the correlation structure of the
$\delta^{18}\mathrm{O}$ field (Fig.~\ref{fig:avg.cor.structure}).

Despite these regional differences, the general feature for sampling
$\delta^{18}\mathrm{O}^{\mathrm{(pw)}}$ is consistent throughout
Antarctica. When we fix the first site to the innermost ring and assess for all
available Antarctic target sites the optimal ring of the second site, for which
the expected correlation with the target site temperature is maximal, we find
that in $\sim77\,\%$ of all cases the optimal location is at least in the second
ring ($250$--$500$\,km), and in $\sim61\,\%$ of the cases it is within the
second to fourth ring ($250$--$1000$\,km).

Furthermore, we obtain similar results for averaging $N=3$ and $N=5$
$\delta^{18}\mathrm{O}^{\mathrm{(pw)}}$ sites to reconstruct the target site
temperature (Fig.~\ref{fig:binning}). While for EDML as a target site, the
optimal sites are arranged such that one site lies in the innermost ring and the
others are distributed at distances between $\sim500$ and $1500$\,km from the
target, for Vostok the sites are mostly distributed across the second to fourth
ring.

\noindent
\#EDIT: What is missing:
\begin{itemize}
\item increase in correlation with number of averaged cores (available with the
  final runs of the ``binning'' results).
\item risk analysis: local correlation versus distribution of correlations from
  the optimal set of ring bins (for $N=3$ or 5 or...?). To be put either here in
  results or as part of the discussion.
\end{itemize}

%f06
\begin{figure*}[t]%
\centering
\includegraphics[width=16cm]{../plots/main/fig_06.pdf}
\caption{%
  The optimal arrangement for averaging three or five
  $\delta^{18}\mathrm{O}^{\mathrm{(pw)}}$ ice cores to reconstruct the target
  site temperature at the EDML (\textbf{a}, \textbf{c}) and Vostok (\textbf{b},
  \textbf{d}) drilling sites. Displayed are the optimal five of all possible
  combinations of rings, i.e. those which exhibit the highest mean
  correlation across $10^5$ random trials of averaging $N=3$ (\textbf{c},
  \textbf{d}) or $N=5$ (\textbf{a}, \textbf{b}) grid cells from these rings.}
\label{fig:binning}%
\end{figure*}%

\section{Discussion}\label{discussion}

(\#ADD: compare local t2m-oxy.pw correlation with previous studies,
e.g. Goursaud2018/ECHAM5-wiso with ERA-Interim?)

% \codedataavailability{TEXT} %% use this section when having data sets and software code available

\appendix

\section{$N=2$ sampling results for Vostok region}
\label{app:vostok.n2}

For a clearer structure of the main text, we here provide the results for
sampling $N=2$ cores from rings in the Vostok study region
(Fig.~\ref{fig:two-core-correlation-vostok}).

%fA01
\begin{figure*}[t]%
\centering
\includegraphics[width=17cm]{../plots/main/fig_A01.png}
\caption{%
  The expected correlation with the target site temperature for the average of
  two cores in the Vostok region. Shown is the mean correlation of all possible
  single correlations from averaging two grid cells of (\textbf{a})
  $T_{\mathrm{2m}}$, (\textbf{b}) $T_{\mathrm{2m}}^{\mathrm{(pw)}}$ and
  (\textbf{c}) $\delta^{18}\mathrm{O}^{\mathrm{(pw)}}$ time series sampled from
  the same or from two different rings, averaged over all target sites in
  the given region. The axes display the distance from the target, where the $x$
  ($y$) axis stands for the first (second) sampled ring and tickmarks
  indicate the radius of the midpoints of the rings. Note that for
  $\delta^{18}\mathrm{O}^{\mathrm{(pw)}}$ the correlation maximum is, albeit
  marginal, located for combining the innermost ring with the ring between
  $500$--$750$\,km.}
\label{fig:two-core-correlation-vostok}%
\end{figure*}%

\section{Conceptual model of correlation structures}
\label{app:concept.model}

\subsection{General model}
\label{app:concept.model.general}

We set up a conceptual model for the correlation between a target temperature
time series and the spatial average of a climatic field, specifically for the
variables temperature and precipitation-weighted temperature. Our model builds
upon assuming simple isotropic and exponential decorrelation structures for the
involved climatic fields and on previous work which suggests that precipitation
intermittency partially converts the input time series into white noise.

For the model, we consider a temperature time series $T_0$ at some target site
$\mathbf{R}_0$ and a field $x$ of a given climate variable. Form this field, we
select $N$ time series $x_i$ at the sites $\mathbf{R}_i$, $i=1,\dotsc,N$, and
denote the spatial average of these time series by
$\overline{x}=\frac{1}{N}\sum_{i=1}^{N}{x_i}$. The distances of the $N$ sites
from the target site and the distances between these sites are given by
$r_i=|\mathbf{R}_i-\mathbf{R}_0|$ and by $d_{ij}=|{\mathbf{R}_i-\mathbf{R}_j}|$,
respectively. The correlation between $T_0$ and $\overline{x}$ follows from
%
\begin{equation}
\label{eq:corr.general}
\mathrm{cor}(T_0,\overline{x})=\frac
{\mathrm{cov}(T_0,\overline{x})}
{\sqrt{\mathrm{var}(T_0)\mathrm{var}(\overline{x})}},
\end{equation}
and it is essentially governed by the spatial covariance between temperature and
the climate field $x$,
%
\begin{equation}
\label{eq:cov.general}
\mathrm{cov}(T_0,\overline{x})=
\frac{1}{N}\sum_{i}^{N}{\mathrm{cov}(T_0,x_i)},
\end{equation}
%
and by the variance of the spatial average of the field,
\begin{equation}
\label{eq:var.general}
\mathrm{var}(\overline{x})=
\frac{1}{N^2}\left(
\sum_{i}^{N}{\mathrm{var}(x_i)} +
2\sum_{i}^{N-1}\sum_{j}^{N}{\mathrm{cov}(x_i,x_j)}
\right).
\end{equation}
%
In our model, as we will show in the following, these quantities depend on the
distances between the sites and on the type of the field $x$.

\subsection{Temperature}
\label{app:concept.model.t2m}

For the near-surface temperature field, $x \equiv T$, we assume a spatially
constant variance, $\mathrm{var}(T_0)=\mathrm{var}(T_i)\equiv\sigma_T^2$, and an
isotropic decorrelation following an exponential law with a decorrelation length
$\tau$; i.e. the covariance between sites is given by
%
\begin{align}
\label{eq:t2m.decorr}
\mathrm{cov}(T_0,T_i)&=\sigma_T^2\exp{\left(-\frac{r_i}{\tau}\right)},\\
\mathrm{cov}(T_i,T_j)&=\sigma_T^2\exp{\left(-\frac{d_{ij}}{\tau}\right)}.
\end{align}
%
For the correlation between the target site temperature and the spatial average
of $N$ temperature time series then immediately follows
%
\begin{equation}
\label{eq:t2m.corr}
\mathrm{cor}(T_0,\overline{T})=
\frac{\sum_{i=1}^{N}\exp{\left(-\frac{r_i}{\tau}\right)}}
{\sqrt{N+2\sum_{i=1}^{N-1}
\sum_{j=i+1}^{N}{\exp{\left(-\frac{d_{ij}}{\tau}\right)}}}}.
\end{equation}

\subsection{Precipitation-weighted temperature}
\label{app:concept.model.t2m.pw}

To model the effect of precipitation intermittency, we follow LAEPPLE et al and
assume that precipitation intermittency redistributes the energy of the
temperature time series constantly across frequencies, i.e. creating temporal
white noise without changing the total variance. Then, the
precipitation-weighted temperature time series at site $\mathbf{R}_i$ arises
from $T_i$ as
%
\begin{equation}
\label{eq:precip.weighting}
T_i^{\mathrm{(pw)}}=
\left(1-\xi\right)^{1/2}T_i + \xi^{1/2} \sigma_T \varepsilon_i(0,1),
\end{equation}
%
where $\varepsilon_i(0,1)$ are independent normally distributed random variables
with zero mean and standard deviation $1$, and the parameter $0\leq\xi\leq1$
determines the fraction of the input temperature time series which is turned
into white noise.

The covariance between the target site temperature and a precipitation-weighted
temperature time series is then simply
\begin{equation}
\label{eq:t2m.pw.decorr}
\mathrm{cov}(T_0,T_i^{\mathrm{(pw)}})=
(1-\xi)^{1/2}\sigma_T^2\exp{\left(-\frac{r_i}{\tau}\right)},
\end{equation}
%
which implies firstly that the spatial correlation structure between $T_0$ and
the precipitation-weighted temperature follows the same exponential law as
Eq.~\eqref{eq:t2m.decorr}, only scaled by the factor $(1-\xi)^{1/2}$. Secondly,
the factor $\xi$ can be estimated from the climate model data by analysing the
local correlation, i.e. at the same grid cell, between temperature and
precipitation-weighted temperature.

We further assume that the effect of precipitation intermittency is not
independent between sites but is related to the spatial coherence of the
precipitation fields, for which we assume an exponential decorrelation structure
with a decay length $\tau_{\mathrm{pw}}$. Following these assumptions, the
spatial covariance between sites of the white noise terms induced by the effect
of precipitation intermitteny has the form
%
\begin{equation}
\label{eq:noise.cov}
\mathrm{cov}(\varepsilon_i,\varepsilon_j)=
\exp{\left(-\frac{d_{ij}}{\tau_{\mathrm{pw}}}\right)}.
\end{equation}
%
Then, the correlation between the target site temperature and the spatial
average of $N$ precipitation-weighted temperature time series is governed by the
intermittency factor $\xi$ and by the two different spatial length scales $\tau$
and $\tau_{\mathrm{pw}}$,
%
\begin{equation}
\label{eq:t2m.pw.corr}
\mathrm{cor}\left(T_0,\overline{T}^{\mathrm{(pw)}}\right)=
\frac
{\sqrt{1-\xi}\sum_{i=1}^{N}\exp{\left(-\frac{r_i}{\tau}\right)}}
{\sqrt{N + 2\sum_{i=1}^{N-1}\sum_{j=i+1}^{N}
  g(d_{ij}; \tau, \tau_{\mathrm{pw}}, \xi)}},
\end{equation}
%
with
\begin{equation}
\label{eq:exp.fun}
g(z; \tau, \tau_{\mathrm{pw}}, \xi):=
(1-\xi)\exp{\left(-\frac{z}{\tau}\right)} +
\xi\exp{\left(-\frac{z}{\tau_{\mathrm{pw}}}\right)}.
\end{equation}

\noappendix

% \authorcontribution{TEXT} %% this section is mandatory
% \competinginterests{TEXT} %% this section is mandatory even if you declare that no competing interests are present
% \disclaimer{TEXT} %% optional section

% \begin{acknowledgements}
% \end{acknowledgements}

% \bibliographystyle{copernicus}
% \bibliography{muench-etal_optimalcores}

\end{document}
